\chapter{Haskell}

Для использования Yesod вам необходимо знать по крайней мере основы языка Haskell. Кроме того, в Yesod используются некоторые особенности Haskell, которые не описаны в большинстве руководств начального уровня. Хотя от читателя предполагается базовое владение Haskell, эта глава призвана заполнить возможные пробелы.

Если вы уже хорошо владеете Haskell, можете пропустить эту главу. Также, если вы предпочитаете начать непосредственно с погружения в Yesod, то вы всегда сможете потом к этой главе вернуться.

Если вам необходимо более полное введение в Haskell, я бы порекомендовал книги <<Real World Haskell>> или <<Learn You a Haskell>>.\marginpar{Возможно, стоит вставить ссылки на книги (в оригинале ссылок нет).}

\section{Терминология}

Даже у людей, хорошо знакомых с языком Haskell, порой может возникать путаница с терминологией. Сформулируем некоторые основные термины, которые мы будем использовать на протяжении всей книги.

\subsection{Тип данных}

Это один из основных строительных блоков языков со строгой типизацией, к которым относится и Haskell. Некоторые типы данных, такие как \lstinline'Int', можно считать элементарными, а остальные типы строятся на их основе для создания более сложных значений. Например, вы могли бы представить человека следующим типом:

\begin{lstlisting}
data Person = Person Text Int
\end{lstlisting}

Здесь \lstinline'Text' содержит имя человека, а \lstinline'Int'~--- его возраст. Благодаря простоте этого примера мы будем обращаться к нему на протяжении всей книги.

Существует три способа, с помощью которых вы можете создать новый тип данных:

\begin{itemize}
  \item Определение \lstinline'type', такое как \lstinline'type GearCount = Int', просто создаёт синоним существующего типа. Система типов не станет препятствовать вам использовать \lstinline'Int' везде, где требуется \lstinline'GearCount'. Использование такого типа может сделать ваш код более самодокументируемым.
  \item Определение \lstinline'newtype', например, \lstinline'newtype Make = Make Text'. В этом случае вы не можете случайно использовать \lstinline'Text' вместо \lstinline'Make', компилятор не даст вам этого сделать. Обёртка, создаваемая посредством \lstinline'newtype', всегда удаляется во время компиляции и не создаёт накладных расходов.
  \item Определение \lstinline'data', как в приведённом выше примере с \lstinline'Person'. Вы можете также создавать алгебраические типы данных (АТД), такие как \lstinline'data Vehicle = Bycicle GearCount | Car Make Model'.
\end{itemize}

\subsection{Конструктор данных}

В приведённых выше примерах \lstinline'Person', \lstinline'Make', Bicycle и Car~--- конструкторы данных.

\subsection{Конструктор типа}

В свою очередь \lstinline'Person', \lstinline'Make' и \lstinline'Vehicle'~--- это конструкторы типов.

\subsection{Переменные типов}

Рассмотрим тип данных \lstinline'data Maybe a = Just a | Nothing'. Здесь \lstinline'a'~--- переменная типа.

\section{Инструментарий}

Для разработки на Haskell вам понадобятся два основных инструмента. Glasgow Haskell Compiler (GHC)~--- стандартный компилятор Haskell, единственный официально поддерживаемый Yesod. Также вам понадобится Cabal~--- стандартное средство сборки для Haskell. Cabal используется не только для сборки локального кода, он также может автоматически скачивать и устанавливать зависимости из Hackage, репозитория пакетов Haskell.

Тем, кто работает в среде Windows или Mac, настоятельно рекомендуется скачать \href{http://hackage.haskell.org/platform/}{Haskell Platform}\footnote{\href{http://hackage.haskell.org/platform/}{\texttt{http://hackage.haskell.org/platform/}}}. Многие дистрибутивы Linux содержат Haskell Platform в своих репозиториях. Например, в дистрибутивах, основанных на Debian, стоит начать с выполнения команды \texttt{apt-get install haskell-platform}. Если в вашем дистрибутиве Haskell Platform недоступна, вы можете установить её вручную, следуя инструкциям на официальном сайте.

Ещё один важный инструмент, который вам нужно будет обновить~--- это \texttt{alex}. Haskell Platform включает версию 2, но используемой Yesod программе минификации JavaScript, \href{http://hackage.haskell.org/package/hjsmin}{hjsmin}\footnote{\href{http://hackage.haskell.org/package/hjsmin}{\texttt{http://hackage.haskell.org/package/hjsmin}}}, требуется версия 3. Не забудьте выполнить \texttt{cabal install alex} после установки Haskell Platform, иначе получите сообщения об ошибках в отношении пакета \href{http://hackage.haskell.org/package/language-javascript}{language-javascript}\footnote{\href{http://hackage.haskell.org/package/language-javascript}{\texttt{http://hackage.haskell.org/package/language-javascript}}}.

\begin{remark}
Некоторым нравится жить на острие прогресса и устанавливать последнюю версию GHC до того, как она станет доступна в Haskell Platform. Мы стараемся поддерживать совместимость Yesod со всеми современными версиями GHC, но официально мы поддерживаем только Haskell Platform. Если вы всё-таки решили пойти по пути ручной установки GHC, то имейте в виду:

\begin{itemize}
  \item Вам нужно будет установить некоторые дополнительные инструменты, в частности, \texttt{alex} и \texttt{happy}.
  \item Убедитесь, что установлены все \href{http://www.vex.net/~trebla/haskell/haskell-platform.xhtml}{требуемые C-библиотеки}\footnote{\href{http://www.vex.net/~trebla/haskell/haskell-platform.xhtml}{\texttt{http://www.vex.net/\~{}trebla/haskell/haskell-platform.xhtml}}}. В дистрибутивах, основанных на Debian, это можно сделать, выполнив команду:\\
 \texttt{apt-get install libedit-dev libbsd-dev lbgmp3-dev zlib1g-dev freeglut3-dev}.
\end{itemize}
\end{remark}

Независимо от способа установки инструментов вам нужно добавить путь к каталогу \texttt{bin} Cabal в переменную окружения \texttt{PATH}. В Mac и Linux это будет \texttt{\$HOME/.cabal/bin}, а в Windows~--- \texttt{\%APPDATA\%\textbackslash{}cabal\textbackslash{}bin}.

У программы \texttt{cabal} есть множество доступных опций, но для начала ознакомьтесь с двумя командами:

\begin{itemize}
  \item \texttt{cabal update} скачивает актуальный список пакетов с Hackage.
  \item \texttt{cabal install yesod-platform} устанавливает Yesod и его зависимости.
\end{itemize}

\begin{remark}
Многие члены сообщества предпочитают выполнять изолированную сборку пакетов Haskell, что предотвращает порчу существующих пакетов вследствие установки Yesod или порчу Yesod из-за установки каких-то пакетов в будущем. В этой книге я не буду углубляться в детали того, как это делается, но два наиболее часто используемых для этих целей инструмента~--- это \href{http://hackage.haskell.org/package/cabal-dev}{cabal-dev}\footnote{\href{http://hackage.haskell.org/package/cabal-dev}{\texttt{http://hackage.haskell.org/package/cabal-dev}}} и \href{http://hackage.haskell.org/package/virthualenv}{virthualenv}\footnote{\href{http://hackage.haskell.org/package/virthualenv}{\texttt{http://hackage.haskell.org/package/virthualenv}}}.
\end{remark}

\section{Указания компилятору}

По умолчанию GHC работает в режиме, очень близком к Haskell98. Однако с ним также поставляется большое число языковых расширений, предоставляющих более мощные классы типов, изменения в синтаксисе и прочее. Есть несколько способов заставить GHC включить эти расширения. В большинстве фрагментов кода, приведённых в этой книге, вы увидите указания компилятору, которые выглядят следующим образом:

\begin{lstlisting}
{-# LANGUAGE MyLanguageExtension #-}
\end{lstlisting}

Эти указания должны всегда находиться в самом начале исходного файла. Помимо этого, есть ещё два распространённых способа:
\begin{itemize}
\item при вызове GHC передать ему аргумент \lstinline'-XMyLanguageExtension';
\item добавить блок \texttt{extensions} в файл \texttt{cabal}.
\end{itemize}

Лично я никогда не использую подход с аргументами командной строки. Это личное предпочтение, но мне нравится, когда мои настройки явным образом указаны в файле. В общем случае также рекомендуется избегать помещения расширений в файл \texttt{cabal}. Впрочем, в сгенерированном Yesod шаблоне сайта \marginpar{Нужно добавить ссылку на главу <<Scaffolding and the Site Template>>} мы намеренно применяем этот подход, чтобы избежать необходимости задавать одни и те же указания компилятору в каждом исходном файле.

В конечном счёте мы будем использовать в этой книге немало языковых расширений (начальный каркас сайта использует 11). Мы не будем рассматривать назначение каждого из них. За этим, пожалуйста, обращайтесь к \href{http://www.haskell.org/ghc/docs/latest/html/users\_guide/ghc-language-features.html}{документации по GHC}\footnote{\href{http://www.haskell.org/ghc/docs/latest/html/users\_guide/ghc-language-features.html}{\texttt{http://www.haskell.org/ghc/docs/latest/html/users\_guide/ghc-language-features.html}}}.

\section{Перегруженные строки}

Какой тип имеет литерал \lstinline'"hello"'? Обычно это \lstinline'String', определённый как \lstinline'type String = [Char]'. К сожалению, у этого подхода есть ряд ограничений:
\begin{itemize}
\item Это очень неэффективный способ хранения текстовых данных. Мы должны выделить память под каждую cons-ячейку, плюс каждый символ занимает целое машинное слово.
\item Иногда у нас есть данные, напоминающие строку, но не являющиеся текстом, такие как \lstinline'ByteString' и \lstinline'HTML'.
\end{itemize}

Для преодоления этих ограничений в GHC есть языковое расширение, называемое \lstinline'OverloadedStrings'. Когда оно включено, строковые литералы будут иметь не мономорфный тип \lstinline'String', а тип \lstinline'IsString a => a', где \lstinline'IsString' определён следующим образом:

\begin{lstlisting}
class IsString a where
    fromString :: String -> a
\end{lstlisting}

Существуют экземпляры \lstinline'IsString' для ряда типов Haskell, таких как \lstinline'Text' (намного более эффективный тип для упакованного хранения строк), \lstinline'ByteString' и \lstinline'HTML'. Практически каждый пример в этой книге предполагает, что это языковое расширение включено.

К сожалению, у этого расширения есть один недостаток: иногда оно может сбить с толку проверку типов GHC. Представьте, что у нас есть следующий код:

\begin{lstlisting}
{-# LANGUAGE OverloadedStrings, TypeSynonymInstances, FlexibleInstances #-}

import Data.Text (Text)

class DoSomething a where
    something :: a -> IO ()

instance DoSomething String where
    something _ = putStrLn "String"

instance DoSomething Text where
    something _ = putStrLn "Text"

myFunc :: IO ()
myFunc = something "hello"
\end{lstlisting}

Что выведет эта программа: <<String>> или <<Text>>? Неясно. Поэтому вам придётся предоставить явную аннотацию типа, чтобы указать, должен ли литерал \lstinline'"hello"' трактоваться как \lstinline'String' или как \lstinline'Text'.

\section{Семейства типов}

Основная идея семейства типов~--- устанавливать некоторую ассоциацию между двумя различными типами. Предположим, мы хотим написать функцию, которая безопасным образом получает первый элемент списка. Но мы не хотим, чтобы она работала только на списках; нам бы хотелось, чтобы она трактовала \lstinline'ByteString' как список, состоящий из элементов типа \lstinline'Word8'. Чтобы этого добиться, нам нужно ввести некоторый ассоциированный тип, чтобы указать, каково содержимое определённого типа.

\lstinputlisting{../hs/02-type-families.hs}

Новый синтаксис выражается в возможности разместить тип внутри класса и экземпляра. Вместо этого мы также можем использовать \lstinline'data', что создаст новый тип данных вместо ссылки на уже существующий.

\begin{remark}
Есть и другие способы использования ассоциированных типов вне контекста класса типов. Однако в Yesod все наши ассоциированные типы являются, фактически, частью класса типов. Более подробную информацию о семействах типов можно найти на соответствующей \href{http://www.haskell.org/haskellwiki/GHC/Type\_families}{странице Haskell Wiki}\footnote{\href{http://www.haskell.org/haskellwiki/GHC/Type\_families}{\texttt{http://www.haskell.org/haskellwiki/GHC/Type\_families}}}.
\end{remark}

\section{Template Haskell}

Template Haskell (TH)~--- это способ \emph{генерации кода}. Мы используем его в ряде мест в Yesod для уменьшения объёма вспомогательного кода и для того, чтобы быть уверенными в корректности сгенерированного кода. Код на Template Haskell~--- это, в сущности, код на Haskell, который генерирует абстрактное синтаксическое дерево (АСД) кода на Haskell.

\begin{remark}
На самом деле, возможности TH этим не ограничиваются, так как он позволяет фактически выполнять интроспекцию кода. Однако эти возможности мы в Yesod не используем.
\end{remark}

Написание кода на TH может быть непростым делом, и, к сожалению, безопасность типов при этом помогает мало. Несложно написать на TH программу, генерирующую код, который не скомпилируется. Это является проблемой только для разработчиков Yesod, но не для его пользователей. В ходе разработки мы используем большой набор модульных тестов, чтобы убедиться в корректности сгенерированного кода. Как пользователю, всё, что вам нужно делать~--- вызывать уже существующие функции. Например, чтобы включить внешний шаблон Hamlet, вы можете написать:

\begin{lstlisting}
$(hamletFile "myfile.hamlet")
\end{lstlisting}
%$

(Hamlet обсуждается в главе о Shakespeare.) \marginpar{Вставить ссылку на главу <<Shakespearean Templates>>}Символ доллара, за которым следует открывающая скобка, сообщает GHC, что то, что следует далее, является функцией Template Haskell. Код внутри скобок затем запускается компилятором и генерирует АСД программы на Haskell, которое затем компилируется. И да, \href{http://www.yesodweb.com/blog/2010/09/yo-dawg-template-haskell}{метапрограммирование можно использовать даже здесь}\footnote{\href{http://www.yesodweb.com/blog/2010/09/yo-dawg-template-haskell}{\texttt{http://www.yesodweb.com/blog/2010/09/yo-dawg-template-haskell}}}.

Приятная особенность~--- код на TH может выполнять произвольные действия ввода-вывода, и, следовательно, мы можем разместить входные данные во внешних файлах, которые будут разбираться во время компиляции. Одним из примеров использования этой возможости является получение шаблонов HTML, CSS и JavaScript, проверяемых во время компиляции.

Если код на Template Haskell используется для генерации объявлений и располагается на верхнем уровне наших файлов, то мы можем избавиться от знаков доллара и скобок. Иными словами,

\begin{lstlisting}
{-# LANGUAGE TemplateHaskell #-}

-- Обычное определение функции, ничего особенного
myFunction = ...

-- Включить код на TH
$(myThCode)

-- Или, то же самое
myThCode
\end{lstlisting}
%$

Может быть полезно посмотреть, какой код генерирует для вас Template Haskell. Для этого вам нужно использовать опцию GHC \texttt{-ddump-splices}.

\begin{remark}
Есть ещё много других особенностей Template Haskell, которые здесь не рассматриваются. Более подробную информацию можете посмотреть на \href{http://www.haskell.org/haskellwiki/Template\_Haskell}{странице Haskell Wiki}\footnote{\href{http://www.haskell.org/haskellwiki/Template\_Haskell}{\texttt{http://www.haskell.org/haskellwiki/Template\_Haskell}}}.
\end{remark}

\subsection{QuasiQuotes}

QuasiQuotes (QQ)~--- это небольшое расширение Template Haskell, которое позволяет нам включать произвольные данные в файлы с исходным кодом на Haskell. Например, выше упоминалась функция Template Haskell \lstinline'hamletFile', которая считывает содержимое шаблона из внешнего файла. У нас также есть обработчик квазицитирования, называемый hamlet, который считывает содержимое, хранящееся непосредственно в исходном коде:

\begin{lstlisting}
{-# LANGUAGE QuasiQuotes #-}

[hamlet|<p>Это квазицитированный Hamlet.|]
\end{lstlisting}

Синтаксис выделяется квадратными скобками и символами вертикальной черты. Имя обработчика квазицитирования размещается между открывающей скобкой и первой вертикальной чертой, а содержимое приводится между двумя символами вертикальной черты.

На протяжении книги мы во многих случаях будем предпочитать внешним файлам QQ-подход, поскольку такой код проще скопировать из книги. Однако в реальных приложениях внешние файлы являются предпочтительными во всех случаях, кроме совсем коротких данных, так как это даёт нам хорошее разделение кода на Haskell и кода с иным синтаксисом.

\section{Выводы}

Вам не нужно быть экспертом в Haskell, чтобы использовать Yesod, достаточно базового знакомства. Надеюсь, эта глава дала вам достаточно информации, чтобы чувствовать себя более комфортно при изучении оставшейся части книги.
