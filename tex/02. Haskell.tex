Haskell

Чтобы использовать Yesod, вам необходимо иметь по крайней мере базовые знания языка Haskell. Помимо этого, в Yesod используются некоторые особенности Haskell, которые не описаны в большинстве руководств начального уровня. Хотя от читателя предполагается знакомство с основами Haskell, эта глава призвана заполнить возможные пробелы.

Если вы уже хорошо владеете Haskell, то вполне можете вовсе пропустить эту главу. Также, если вы предпочтете начать непосредственно с погружения в Yesod, то вы всегда сможете при необходимости вернуться к этой главе.

Если вам необходимо более полное руководство по языку Haskell, я бы порекомендовал "Real World Haskell" или "Learn You a Haskell".

Терминология

Даже у тех, кто хорошо знаком с Haskell, иногда может возникнуть путаница с терминологией. Сформулируем некоторые базовые термины, которые мы будем использовать на протяжении этой книги.

Тип данных

Это один из основных строительных блоков для строго типизированного языка, которым является Haskell. Некоторые типы данных, такие как Int, можно считать элементарными, а остальные типы строятся на их основе для создания более сложных значений. Например, вы могли бы представить человека следующим образом:

data Person = Person Text Int

Здесь Text содержит имя человека, а Int~--- его возраст. Из-за простоты этого примера мы будем обращаться к нему на протяжении всей книги.

Существует три способа, с помощью которых вы можете создать тип данных:

\begin{itemize}
  \item Определение типа, например, type GearCount = Int, просто создает синоним существующего типа. Система типов не станет препятствовать вам использовать Int везде, где требуется GearCount. Использование такого типа может сделать ваш код более самодокументирующим.
  \item Определение newtype, например, newtype Make = Make Text. В этом случае вы не можете случайно использовать Text вместо Make, компилятор не даст вам этого сделать. Обертка, создаваемая newtype, удаляется во время компиляции и не создает накладных расходов.
  \item Определение data, как в приведенном выше примере с Person. Вы можете также создавать алгебраические типы данных (АТД), такие как data Vehicle = Bycicle GearCount | Car Make Model.
\end{itemize}

Конструктор данных

В приведенных выше примерах Person, Make, Bicycle и Car~--- конструкторы данных.

Конструктор типа

В приведенных выше примерах Person, Make и Vehicle~--- конструкторы типов.

Переменные типов

Рассмотрим тип данных data Maybe a = Just a | Nothing. В этом случае a~--- переменная типа.

Инструментарий

Для разработки на Haskell вам понадобятся два основных инструмента. Во-первых, Glasgow Haskell Compiler (GHC), стандартный компилятор Haskell~--- и единственный официально поддерживаемый Yesod. Вам также понадобится Cabal, стандартное средство сборки для Haskell. Cabal используется не только для сборки локального кода~--- он также может автоматически скачивать и устанавливать зависимости из Hackage, репозитория пакетов Haskell.

Тем, кто работает в среде Windows или Mac, настоятельно рекомендуется скачать Haskell Platform. Многие дистрибутивы Linux содержат Haskell Platform в своих репозиториях. Например, в дистрибутивах, основанных на Debian, стоит начать с выполнения команды sudo apt-get install haskell-platform. Если в вашем дистрибутиве Haskell Platform недоступна, вы можете установить ее вручную, следуя инструкциям на официальном сайте.

Еще один важный инструмент, который вам нужно будет обновить,~--- это alex. Haskell Platform включает версию 2, но используемой Yesod программе минимизации JavaScript, hjsmin, требуется версия 3. Не забудьте выполнить cabal install alex после установки Haskell Platform, иначе получите сообщения об ошибках в отношении пакета language-javascript.

Некоторым нравится жить на острие прогресса и устанавливать последнюю версию GHC до того, как она станет доступна в Haskell Platform. Мы стараемся поддерживать совместимость Yesod со всеми современными версиями GHC, но официально мы поддерживаем только Haskell Platform. Если вы все-таки решили пойти по пути ручной установки GHC, то:

\begin{itemize}
  \item Вам нужно будет установить некоторые дополнительные инструменты, в частности, alex и happy.
  \item Убедитесь, что установлены все требуемые C-библиотеки. В дистрибутивах, основанных на Debian, это можно сделать, выполнив команду sudo apt-get install libedit-dev libbsd-dev lbgmp3-dev zlib1g-dev freeglut3-dev.
\end{itemize}

Независимо от способа установки инструментов вам нужно добавить путь к каталогу bin Cabal в переменную окружения PATH. В Mac и Linux это будет \$HOME/.cabal/bin, а в Windows~--- \%APPDATA\%\textbackslash{}cabal\textbackslash{}bin.

У Cabal есть множество доступных опций, но для начала ознакомьтесь с двумя:

\begin{itemize}
  \item cabal updata скачивает актуальный список пакетов с hackage.
  \item cabal install yesod-platform устанавливает Yesod и его зависимости.
\end{itemize}

Многие члены сообщества предпочитают выполнять изолированную сборку пакетов Haskell, что предотвращает порчу существующих пакетов вследствие установки Yesod или порчу Yesod из-за установки каких-то пакетов в будущем. В этой книге я не буду углубляться в детали того, как это делается, но два наиболее часто используемых для этих целей инструмента~--- это cabal-dev и virthualenv.
