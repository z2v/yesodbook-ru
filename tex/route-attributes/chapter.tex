\chapter{Атрибуты маршрутов}

Атрибуты маршрутов позволяют вам добавить некоторые метаданные для каждого
вашего маршрута непосредственно в описании маршрутов. Синтаксис тривиален:
просто восклицательный знак, за которым следует значение. Использование столь
же тривиально: просто используйте функцию~\lstinline'routeAttrs'.

Самый простой способ понять, как это всё совмещается и когда это вам может
потребоваться,~--- с помощью демонстрационного примера. Случай, когда я чаще
всего использую атрибуты маршрутов,~--- это обозначение административных
маршрутов. Представьте себе веб-сайт с 12-ю различными административными
действиями. Вы \textbf{могли бы} вручную добавить вызов
функции~\lstinline'requireAdmin' или нечто подобное в начало каждого действия,
но
\begin{itemize}
    \item Это утомительно.

    \item Чревато ошибками: вы легко можете забыть сделать это.

    \item Ещё хуже~--- не так просто заметить, что вы что-то упустили.
\end{itemize}

Добавление в ваш метод~\lstinline'isAuthorized' явного списка административных
маршрутов чуточку лучше, но по-прежнему трудно увидеть с первого взгляда, если
вы что-то упустили.

Вот поэтому я предпочитаю использовать для этого атрибуты маршрутов: вы
добавляете единственное слово в соответствующую часть описания маршрута и затем
просто проверяете наличие этого атрибута в функции~\lstinline'isAuthorized'.
Посмотрим на код!

\sourcecode{route-attributes}{route-attributes.hs}

Это было настолько очевидно, что, уверен, вы увидели брешь в безопасности
маршрута~\lstinline'Admin3R'.
