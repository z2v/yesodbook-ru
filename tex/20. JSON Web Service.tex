\chapter {JSON веб-сервис}\label{chap:json_web_service}

% FIXME \hyperref[chap:http_conduit]{http-conduit}
Давайте создадим простой веб-сервис: он будет принимать запрос в формате JSON и в нём же отдавать ответ. Мы напишем сервер на WAI/Warp, а клиент~--- с помощью {http-conduit}. Мы будем использовать \footnotehref{http://hackage.haskell.org/package/aeson}{aeson} для разбора JSON и его отображения. Мы также могли бы написать сервер и на Yesod, но для такого простого примера большие возможности Yesod немногое нам дают.

\section {Сервер}

WAI использует пакет \footnotehref{http://hackage.haskell.org/package/conduit}{conduit} для того, чтобы обрабатывать тела потоковых запросов, и эффективно формирует ответы при помощи \footnotehref{http://hackage.haskell.org/package/blaze-builder}{blaze-builder}. \footnotehref{http://hackage.haskell.org/package/aeson}{aeson} использует \footnotehref{http://hackage.haskell.org/package/attoparser}{attoparsec} для разбора текста; используя \footnotehref{http://hackage.haskell.org/package/attoparsec-conduit}{attoparsec-conduit} мы получаем простое взаимодействие с WAI. В итоге код выглядит следующим образом:

\includecode{20/server.hs}

\section {Клиент}

\footnotehref{http://hackage.haskell.org/package/http-conduit}{http-conduit} был написан как сопутствующий компонент к WAI. Он также использует \lstinline!conduit! и \lstinline!blaze-builder! повсеместно, что означает, что мы также получаем простое взаимодействие с \lstinline!aeson!. Несколько комментариев для тех, кто ещё не знаком с \lstinline!http-conduit!:

\begin{itemize}
  \item \lstinline!Manager! управляет открытыми соединениями таким образом, что повторные запросы к тому же серверу используют то же самое соединение. Чаще всего вам потребуется использовать функцию \lstinline!withManager! для создания и очистки \lstinline!Manager!, потому как она безопасна по отношению к исключениям.
  \item Нам необходимо знать размер тела запроса, который мы не можем определить напрямую из \lstinline!Builder!. Вместо этого мы преобразуем \lstinline!Builder! к ленивому \lstinline!ByteString! и берём размер из него.
  \item Есть несколько различных функций для инициализации запроса. Мы используем \lstinline!http!, которая позваляет нам получать прямой доступ к потоку данных. Есть и другие, более высокоуровневые функции (такие как \lstinline!httpLbs!), которые позволяют игнорировать вопросы с источниками и дают возможность получить напрямую тело целиком.
\end{itemize}

\includecode{20/client.hs}
