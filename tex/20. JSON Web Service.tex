\chapter {JSON веб-сервис}\label{chap:json_web_service}

% FIXME \hyperref[chap:http_conduit]{http-conduit}
Давайте создадим простой веб-сервис: он будет принимать запрос в формате JSON и в нём же отдавать ответ. Мы напишем сервер на WAI/Warp, а клиент~--- с помощью {http-conduit}. Мы будем использовать \footnotehref{http://hackage.haskell.org/package/aeson}{aeson} для разбора JSON и его отображения. Мы также могли бы написать сервер и на Yesod, но для такого простого примера большие возможности Yesod немногое нам дают.

\section {Сервер}

WAI использует пакет \footnotehref{http://hackage.haskell.org/package/conduit}{conduit} для того, чтобы обрабатывать тела потоковых запросов, и эффективно формирует ответы при помощи \footnotehref{http://hackage.haskell.org/package/blaze-builder}{blaze-builder}. \footnotehref{http://hackage.haskell.org/package/aeson}{aeson} использует \footnotehref{http://hackage.haskell.org/package/attoparser}{attoparsec} для разбора текста; используя \footnotehref{http://hackage.haskell.org/package/attoparsec-conduit}{attoparsec-conduit} мы получаем простое взаимодействие с WAI. В итоге код выглядит следующим образом:

\begin{lstlisting}
{-# LANGUAGE OverloadedStrings #-}
import Network.Wai (Response, responseLBS, Application, requestBody)
import Network.HTTP.Types (status200, status400)
import Network.Wai.Handler.Warp (run)
import Data.Aeson.Parser (json)
import Data.Conduit.Attoparsec (sinkParser)
import Control.Monad.IO.Class (liftIO)
import Data.Aeson (Value, encode, object, (.=))
import Control.Exception (SomeException)
import Data.ByteString (ByteString)
import Data.Conduit (ResourceT, ($$))
import Control.Exception.Lifted (handle)

main :: IO ()
main = run 3000 app

app :: Application
app req = handle invalidJson $ do
    value <- requestBody req $$ sinkParser json
    newValue <- liftIO $ modValue value
    return $ responseLBS
        status200
        [("Content-Type", "application/json")]
        $ encode newValue

invalidJson :: SomeException -> ResourceT IO Response
invalidJson ex = return $ responseLBS
    status400
    [("Content-Type", "application/json")]
    $ encode $ object
        [ ("message" .= show ex)
        ]

-- Бизнес-логика будет находиться в этой функции
modValue :: Value -> IO Value
modValue = return
\end{lstlisting}%$

\section {Клиент}

\footnotehref{http://hackage.haskell.org/package/http-conduit}{http-conduit} был написан как сопутствующий компонент к WAI. Он также использует \lstinline!conduit! и \lstinline!blaze-builder! повсеместно, что означает, что мы также получаем простое взаимодействие с \lstinline!aeson!. Несколько комментариев для тех, кто ещё не знаком с \lstinline!http-conduit!:

\begin{itemize}
  \item \lstinline!Manager! управляет открытыми соединениями таким образом, что повторные запросы к тому же серверу используют то же самое соединение. Чаще всего вам потребуется использовать функцию \lstinline!withManager! для создания и очистки \lstinline!Manager!, потому как она безопасна по отношению к исключениям.
  \item Нам необходимо знать размер тела запроса, который мы не можем определить напрямую из \lstinline!Builder!. Вместо этого мы преобразуем \lstinline!Builder! к ленивому \lstinline!ByteString! и берём размер из него.
  \item Есть несколько различных функций для инициализации запроса. Мы используем \lstinline!http!, которая позваляет нам получать прямой доступ к потоку данных. Есть и другие, более высокоуровневые функции (такие как \lstinline!httpLbs!), которые позволяют игнорировать вопросы с источниками и дают возможность получить напрямую тело целиком.
\end{itemize}

\begin{lstlisting}
{-# LANGUAGE OverloadedStrings #-}
import Network.HTTP.Conduit
    ( http, parseUrl, withManager, RequestBody (RequestBodyLBS)
    , requestBody, method, Response (..)
    )
import Data.Aeson (Value (Object, String))
import Data.Aeson.Parser (json)
import Data.Conduit (($$))
import Data.Conduit.Attoparsec (sinkParser)
import Control.Monad.IO.Class (liftIO)
import Data.Aeson (encode, (.=), object)

main :: IO ()
main = withManager $ \manager -> do
    value <- liftIO makeValue
    -- Нам надо знать размер тела запроса, поэтому преобразуем его к
    -- ByteString
    let valueBS = encode value
    req' <- liftIO $ parseUrl "http://localhost:3000/"
    let req = req' { method = "POST", requestBody = RequestBodyLBS valueBS }
    Response status version headers body <- http req manager
    resValue <- body $$ sinkParser json
    liftIO $ handleResponse resValue

-- Фунция бизнес-логики приложения, создающая значение для запроса
makeValue :: IO Value
makeValue = return $ object
    [ ("foo" .= ("bar" :: String))
    ]

-- Функция бизнес-логики приложения, обрабатывающая ответ от сервера
handleResponse :: Value -> IO ()
handleResponse = print
\end{lstlisting}
