\chapter{Интернационализация} % Internationalization

Пользователи ожидают от наших программ, что они будут разговаривать с ними на одном языке. К несчастью для нас, для этого скорее всего потребуется поддержка более, чем одного языка. В то время, как простая замена строк не представляет собой большой проблемы, корректное формирование фраз в соответствии со всеми правилами грамматики может оказаться нетривиальной задачей. В конце концов, кому из нас приятно видеть в выводе программы <<Список 1 файл(ов)>>?

Но от настоящей интернационализации (i18n, от англ. internationalization - прим. пер.) требуется не только возможность формировать корректный вывод. Также требуется сделать этот процесс простым как для программиста, так и для переводчика, а также оставить относительно простой проверку на ошибки. Решение, реализованное в Yesod, позволяет вам:

\begin{itemize}
  \item Угадывать язык пользователя основываясь на информации, переданной HTTP запросе, с возможностью перезаписи.
  \item Простой синтаксис для формирования переводов, не требующий знания Haskell. (В конце концов, не всякий переводчик является еще и программистом.)
  \item Возможность при необходимости использовать всю мощь языка Haskell для нетривиальных грамматических проблем, а также выбор хелперов по умолчанию для покрытия большинства случаев.
  \item Полное отсутсвие проблем с порядком слов. % Absolutely no issues at all with word order
\end{itemize}

\section{Краткое содержание} % Synopsis

\begin{lstlisting}
-- @messages/en.msg
Hello: Hello
EnterItemCount: I would like to buy: 
Purchase: Purchase
ItemCount count@Int: You have purchased #{showInt count} #{plural count "item" "items"}.
SwitchLanguage: Switch language to: 
Switch: Switch

-- @messages/he.msg
Hello: שלום
EnterItemCount: אני רוצה לקנות: 
Purchase: קנה
ItemCount count: קנית #{showInt count} #{plural count "דבר" "דברים"}.
SwitchLanguage: החלף שפה ל:
Switch: החלף

-- @i18n-synopsis.hs
{-# LANGUAGE OverloadedStrings, QuasiQuotes, TemplateHaskell, TypeFamilies,
    MultiParamTypeClasses #-}
import Yesod

data I18N = I18N

mkMessage "I18N" "messages" "en"

plural :: Int -> String -> String -> String
plural 1 x _ = x
plural _ _ y = y

showInt :: Int -> String
showInt = show

instance Yesod I18N

instance RenderMessage I18N FormMessage where
    renderMessage _ _ = defaultFormMessage

mkYesod "I18N" [parseRoutes|
/ RootR GET
/buy BuyR GET
/lang LangR POST
|]

getRootR :: Handler RepHtml
getRootR = defaultLayout [whamlet|
<h1>_{MsgHello}
<form action=@{BuyR}>
    _{MsgEnterItemCount}
    <input type=text name=count>
    <input type=submit value=_{MsgPurchase}>
<form action=@{LangR} method=post>
    _{MsgSwitchLanguage}
    <select name=lang>
        <option value=en>English
        <option value=he>Hebrew
    <input type=submit value=_{MsgSwitch}>
|]

getBuyR :: Handler RepHtml
getBuyR = do
    count <- runInputGet $ ireq intField "count"
    defaultLayout [whamlet|
<p>_{MsgItemCount count}
|]

postLangR :: Handler ()
postLangR = do
    lang <- runInputPost $ ireq textField "lang"
    setLanguage lang
    redirect RootR

main :: IO ()
main = warpDebug 3000 I18N
\end{lstlisting}

\section{Обзор} % Overview

Most existing i18n solutions out there, like gettext or Java message bundles, work on the principle of string lookups. Usually some form of printf-interpolation is used to interpolate variables into the strings. In Yesod, as you might guess, we instead rely on types. This gives us all of our normal advantages, such as the compiler automatically catching mistakes.

Let's take a concrete example. Suppose our application has two things it wants to say to a user: say hello, and state how many users are logged into the system. This can be modeled with a sum type:

\begin{lstlisting}
data MyMessage = MsgHello | MsgUsersLoggedIn Int
\end{lstlisting}

I can also write a function to turn this datatype into an English representation:

\begin{lstlisting}
toEnglish :: MyMessage -> String
toEnglish MsgHello = "Hello there!"
toEnglish (MsgUsersLoggedIn 1) = "There is 1 user logged in."
toEnglish (MsgUsersLoggedIn i) = "There are " ++ show i ++ " users logged in."
\end{lstlisting}

We can also write similar functions for other languages. The advantage to this inside-Haskell approach is that we have the full power of Haskell for addressing tricky grammar issues, especially pluralization.

\fbox{\begin{minipage}[!h]{.9\linewidth}
You may think pluralization isn't so complicated: you have one version for 1 item, and another for any other count. That might be true in English, but it's not true for every language. Russian, for example, has six different forms, and you need to use some modulus logic to determine which one to use.
\end{minipage}}

The downside, however, is that you have to write all of this inside of Haskell, which won't be very translator-friendly. To solve this, Yesod introduces the concept of message files. We'll cover that in a little bit.

Assuming we have this full set of translation functions, how do we go about using them? What we need is a new function to wrap them all up together, and then choose the appropriate translation function based on the user's selected language. Once we have that, Yesod can automatically choose the most relevant render function and call it on the values you provide.

In order to simplify things a bit, Hamlet has a special interpolation syntax, \lstinline'_{...}', which handles all the calls to the render functions. And in order to associate a render function with your application, you use the YesodMessage typeclass.

\section{Файлы сообщений} % Message files

The simplest approach to creating translations is via message files. The setup is simple: there is a single folder containing all of your translation files, with a single file for each language. Each file is named based on its language code, e.g. en.msg. And each line in a file handles one phrase, which correlates to a single constructor in your message data type.

\fbox{\begin{minipage}[!h]{.9\linewidth}
The scaffolded site already includes a fully configured message folder.
\end{minipage}}

So firstly, a word about language codes. There are really two choices available: using a two-letter language code, or a language-LOCALE code. For example, when I load up a page in my web browser, it sends two language codes: en-US and en. What my browser is saying is <<if you have American English, I like that the most. If you have English, I'll take that instead>>.

So which format should you use in your application? Most likely two-letter codes, unless you are actually creating separate translations by locale. This ensures that someone asking for Canadian English will still see your English. Behind the scenes, Yesod will add the two-letter codes where relevant. For example, suppose a user has the following language list:

\begin{lstlisting}
pt-BR, es, he
\end{lstlisting}

What this means is <<I like Brazilian Porteguese, then Spanish, and then Hebrew>>. Suppose your application provides the languages pt (general Porteguese) and English, with English as the default. Strictly following the user's language list would result in the user being served English. Instead, Yesod translates that list into:

\begin{lstlisting}
pt-BR, es, he, pt
\end{lstlisting}

In other words: unless you're giving different translations based on locale, just stick to the two-letter language codes.

Now what about these message files? The syntax should be very familiar after your work with Hamlet and Persistent. The line starts off with the name of the message. Since this is a data constructor, it must start with a capital letter. Next, you can have individual parameters, which must be given as lower case. These will be arguments to the data constructor.

The argument list is terminated by a colon, and then followed by the translated string, which allows usage of our typical variable interpolation syntax \lstinline'#{myVar}'. By referring to the parameters defined before the colon, and using translation helper functions to deal with issues like pluralization, you can create all the translated messages you need.

\section{Определяя типы} % Specifying types

Since we will be creating a datatype out of our message specifications, each parameter to a data constructor must be given a data type. We use a @-syntax for this. For example, to create the datatype \lstinline'data MyMessage = MsgHello | MsgSayAge Int', we would write:

\begin{lstlisting}
Hello: Hi there!
SayAge age@Int: Your age is: #{show age}
\end{lstlisting}

But there are two problems with this:

\begin{enumerate}
  \item It's not very DRY (don't repeat yourself) to have to specify this datatype in every file.
  \item Translators will be confused having to specify these datatypes.
\end{enumerate}

So instead, the type specification is only required in the main language file. This is specified as the third argument in the mkMessage function. This also specifies what the backup language will be, to be used when none of the languages provided by your application match the user's language list.

\section{Класс типов RenderMessage} % RenderMessage typeclass

Your call to mkMessage creates an instance of the RenderMessage typeclass, which is the core of Yesod's i18n. It is defined as:
\begin{lstlisting}
class RenderMessage master message where
    renderMessage :: master
                  -> [Text] -- ^ languages
                  -> message
                  -> Text 
\end{lstlisting}

Notice that there are two parameters to the RenderMessage class: the master site and the message type. In theory, we could skip the master type here, but that would mean that every site would need to have the same set of translations for each message type. When it comes to shared libraries like forms, that would not be a workable solution.

The renderMessage function takes a parameter for each of the class's type parameters: master and message. The extra parameter is a list of languages the user will accept, in descending order of priority. The method then returns a user-ready Text that can be displayed.

A simple instance of RenderMessage may involve no actual translation of strings; instead, it will just display the same value for every language. For example:

\begin{lstlisting}
data MyMessage = Hello | Greet Text
instance RenderMessage MyApp MyMessage where
    renderMessage _ _ Hello = "Hello"
    renderMessage _ _ (Greet name) = "Welcome, " <> name <> "!"
\end{lstlisting}
    
Notice how we ignore the first two parameters to renderMessage. We can now extend this to support multiple languages:

\begin{lstlisting}
renderEn Hello = "Hello"
renderEn (Greet name) = "Welcome, " <> name <> "!"
renderHe Hello = "שלום"
renderHe (Greet name) = "ברוכים הבאים, " <> name <> "!"
instance RenderMessage MyApp MyMessage where
    renderMessage _ ("en":_) = renderEn
    renderMessage _ ("he":_) = renderHe
    renderMessage master (_:langs) = renderMessage master langs
    renderMessage _ [] = renderEn
\end{lstlisting}

The idea here is fairly straight-forward: we define helper functions to support each language. We then add a clause to catch each of those languages in the renderMessage definition. We then have two final cases: if no languages matched, continue checking with the next language in the user's priority list. If we've exhausted all languages the user specified, then use the default language (in our case, English).

But odds are that you will never need to worry about writing this stuff manually, as the message file interface does all this for you. But it's always a good idea to have an understanding of what's going on under the surface.

\section{Интерполяция} % Interpolation

One way to use your new RenderMessage instance would be to directly call the renderMessage function. This would work, but it's a bit tedious: you need to pass in the foundation value and the language list manually. Instead, Hamlet provides a specialized i18n interpolation, which looks like \lstinline'_{...}'.

\fbox{\begin{minipage}[!h]{.9\linewidth}
Why the underscore? Underscore is already a well-established character for i18n, as it is used in the gettext library.
\end{minipage}}

Hamlet will then automatically translate that to a call to renderMessage. Once Hamlet gets the output Text value, it uses the toHtml function to produce an Html value, meaning that any special characters (\lstinline'<, &, >') will be automatically escaped.

\section{Фразы, а не слова} % Phrases, not words

As a final note, I'd just like to give some general i18n advice. Let's say you have an application for selling turtles. You're going to use the word <<turtle>> in multiple places, like <<You have added 4 turtles to your cart>> and <<You have purchased 4 turtles, congratulations>>. As a programmer, you'll immediately notice the code reuse potential: we have the phrase <<4 turtles>> twice. So you might structure your message file as:

\begin{lstlisting}
AddStart: You have added
AddEnd: to your cart.
PurchaseStart: You have purchased
PurchaseEnd: , congratulations!
Turtles count@Int: #{show count} #{plural count "turtle" "turtles"}
\end{lstlisting}

STOP RIGHT THERE! This is all well and good from a programming perspective, but translations are not programming. There are a many things that could go wrong with this, such as:

\begin{itemize}
  \item Some languages might put "to your cart" before "You have added."
  \item Maybe "added" will be constructed differently depending whether you added 1 or more turtles.
  \item There are a bunch of whitespace issues as well.
\end{itemize}

So the general rule is: translate entire phrases, not just words.
