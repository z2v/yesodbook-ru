\chapter{Формы}\label{chap:forms}
Я уже упоминал граничное условие: если данные приходят в приложение или
покидают его, мы должны их проверять. Вероятно, наиболее сложная проверка
происходит в формах.  Программировать формы не так просто; в идеальных
условиях мы хотели бы иметь решение, которое может следующее:

\begin{itemize}
    \item гарантировать, что данные корректны;
    \item преобразовывать строковые данные формы в типы данных Haskell;
    \item генерировать код HTML для отображения формы;
    \item генерировать Javascript, выполняющий валидацию на стороне клиента и
        предоставляющий более дружелюбные пользователю виджеты, например, для
        выбора даты;
    \item строить более сложные формы, объединяя вместе более простые;
    \item автоматически присваивать полям имена, которые гарантированно будут
        уникальными.
\end{itemize}

Пакет yesod-form предоставляет все эти возможности с помощью простого,
декларативного API. Он построен на базе виджетов Yesod для упрощения стилевого
оформления форм и применения Javascript надлежащим образом. И, как и везде в
Yesod, для обеспечения корректной работы используется система типов Haskell.

\section{Краткий обзор}
\lstinputlisting{../hs/08/synopsis.hs}

\section{Виды форм}
Погружение в непосредственное рассмотрение типов нам следует начать с обзора
различных видов форм. Их три категории:
\begin{description}
    \item[Аппликативные] \hfill \\
        Наиболее широко используемые (см. пример выше). Аппликативные формы
        позволяют нам объединять сообщения об ошибках друг с другом, сохраняя
        при этом декларативный высокоуровневый подход. Более детальную
        информацию об аппликативном подходе можно почерпнуть
        в~\footnotehref{http://www.haskell.org/haskellwiki/Applicative\_functor}%
        {Haskell-вики}.

    \item[Монадические] \hfill \\
        Более мощная альтернатива аппликативным. Гибкость достигается за счёт
        несколько большей многословности. Бывают полезны, если необходимо
        создать форму, которая не укладывается в стандартный двухстолбцовый
        формат.

    \item[Формы для ввода] \hfill \\
        Используются только для получения ввода. Не генерируют никакого HTML для
        получения ввода пользователя. Полезны для взаимодействия с уже
        существующими формами.
\end{description}

В дополнение, существует ряд различных переменных для каждого вида формы и поля,
которые вам захочется установить:
\begin{itemize}
    \item Обязательное поле или нет?
    \item Данные должны передаваться методом GET или POST?
    \item У поля есть значение по умолчанию или нет?
\end{itemize}

Главная цель~--- минимизировать число определений полей и позволить им
работать в максимально возможном количестве контекстов. Один из результатов
этого~--- мы пришли к добавлению нескольких дополнительных параметров для
каждого поля. В обзоре, вы, наверное, заметили такие слова как
\lstinline'areq' и дополнительный параметр \lstinline'Nothing'. В рамках
данной главы мы обсудим зачем они нужны, а сейчас представьте себе, что сделав
эти параметры явными, мы получили возможность повторно использовать отдельные
поля (например,~%
\footnotehref{http://hackage.haskell.org/packages/archive/yesod-form/latest/doc/html/Yesod-Form-Fields.html\#v:intField}%
{\lstinline'intField'}) большим количеством различных способов.

Короткое замечание по поводу именования. Каждая форма имеет однобуквенный
префикс (\texttt{A}, \texttt{M} или~\texttt{I}), который используется в
нескольких местах, как например в \lstinline'MForm'. Мы также будем
использовать~\lstinline'req' и~\lstinline'opt' для обозначения обязательных
(required) и необязательных (optional) полей соответственно. В итоге,
обязательное аппликативное поле мы создаём с помощью~\lstinline'areq', а
необязательное поле для ввода данных~--- с помощью~\lstinline'iopt'.

\section{Типы}
Модуль
\footnotehref{http://hackage.haskell.org/packages/archive/yesod-form/latest/doc/html/Yesod.Form.Types.html}{\texttt{Yesod.Form.Types}}
определяет набор типов. Давайте начнём с нескольких вспомогательных:

\begin{description}
    \item[Enctype] \hfill \\
        Тип кодировки: либо \lstinline'UrlEncoded', либо
        \lstinline'Multipart'. Этот тип данных объявляет экземпляр
        для~\lstinline'ToHtml', так что вы можете его использовать
        непосредственно в шаблонах~\texttt{Hamlet}.

    \item[Env] \hfill \\
        Отображает имя параметра в список значений.

    \item[FileEnv] \hfill \\
        Отображает имя параметра в соответствующий загруженный файл.

    \item[Ints] \hfill \\
        Как было сказано во введении, \texttt{yesod-form} автоматически
        присваивает уникальное имя каждому полю. \lstinline'Ints' используется
        для отслеживания следующего номера для присваивания.

    \item[FormResult] \hfill \\
        Имеет одно из трёх возможных состояний: \lstinline'FormMissing'~---
        если данные не были переданы, \lstinline'FormFailure'~--- если
        произошла ошибка при разборе формы (например, не заполнено
        обязательное поле или содержимое поля некорректно)
        или~\lstinline'FormSuccess', когда всё прошло гладко.
\end{description}

Три следующих типа данных используются для определения отдельных полей.

\begin{remark}
Поле~--- это отдельная порция информации, например, число, строка или адрес
электронной почты. Поля собирают вместе для построения форм.
\end{remark}

\begin{description}
    \item[Field] \hfill \\
        Определяют два вида функциональности: как преобразовать текстовый ввод
        пользователя в значение языка Haskell и как создавать виджет, который
        будет показан пользователю.  \texttt{yesod-form} определяет ряд
        отдельных полей в
        модуле~\footnotehref{http://hackage.haskell.org/packages/archive/yesod-form/latest/doc/html/Yesod.Form.Fields.html}{\lstinline'Yesod.Form.Fields'}.

    \item[FieldSettings] \hfill \\
        Основная информация о том, как поле следует отображать: отображаемое имя,
        необязательная подсказка, и, возможно, явно заданные атрибуты
        \lstinline'id' и~\lstinline'name'. (Если ничего не указано, то они
        генерируются автоматически.)

        \begin{remark}
            \lstinline'FieldSettings' предоставляет
            экземпляр~\lstinline'IsString', так что, когда вам понадобится
            указать значение типа~\lstinline'FieldSettings', вы можете
            обойтись просто строковым литералом. Так мы делали во введении.
        \end{remark}

    \item[FieldView] \hfill \\
        Промежуточный формат, содержащий кучу информации об отображении поля.
        Едва ли будет использоваться напрямую. Детали рассмотрим ниже.
\end{description}

И наконец, мы добрались до важного: сами формы. Есть три типа данных для форм:
\lstinline'MForm' для монадических, \lstinline'AForm' для аппликативных и
\lstinline'IForm' (определённая
в~\footnotehref{http://hackage.haskell.org/packages/archive/yesod-form/latest/doc/html/Yesod-Form-Input.html\#t:IForm}{\lstinline'IForm'})
для форм ввода данных.

\lstinline'MForm' на самом деле является синонимом типа для стека монад,
который предоставляет следующие возможности:
\begin{itemize}
    \item Монада \lstinline'Reader' предоставляет нам параметры
        (\lstinline'Env' и~\lstinline'FileEnv'), аргумент основного сайта
        (master site) и список языков, которые поддерживаются пользователем.
        Последние два используются для интернационализации (подробнее об этом
        ниже).

    \item Монада \lstinline'Writer' отслеживает \lstinline'Enctype'. Для формы
        значение всегда будет~\lstinline'UrlEncoded', за исключением случая,
        когда присутствует поле для загрузки файла. Тогда будет использовано
        значение~\lstinline'Multipart'.

    \item Монада \lstinline'State' хранит \lstinline'Ints' для отслеживания
        следующего генерируемого уникального имени.
\end{itemize}

Форма \lstinline'AForm' устроена сходно. Однако, есть несколько важных различий:
\begin{itemize}
    \item Она генерирует список \lstinline'FieldViews'. Это позволяет нам
        оперировать абстракцией отображения форм, выбирая соответствующую
        функцию для отрисовки страницы в самом конце. Во введении мы
        использовали \lstinline'renderDivs', которая создаёт связку тегов div.
        Другим вариантом может быть \lstinline'renderTable'.

    \item У неё нет экземпляра \lstinline'Monad'. Задача
        \lstinline'Applicative' заключается в том, чтобы позволить форме
        выполниться целиком, собрать как можно больше информации, и выдать
        результат. Это не работает в контексте \lstinline'Monad'.
\end{itemize}

\lstinline'IForm' ещё проще: она возвращает либо список ошибок, либо результат.

\section{Преобразование}
<<Минуточку>>,~--- скажете вы. <<Вы говорите, что во введении использовались
аппликативные формы, но я уверен, что сигнатура типа
указывает~\lstinline'MForm'. Не были ли они монадическими?>>. Да, верно,
итоговая форма, которую мы создали, была монадической. Но на самом деле
произошло вот что: мы преобразовали аппликативную форму в монадическую.

Повторюсь, наша цель заключается в том, чтобы повторно использовать как можно
больше кода и минимизировать число функций в API. А монадические формы
являются более мощными, чем аппликативные, поэтому, грубо говоря, всё, что
может быть выражено аппликативными формами, может быть выражено и
монадическими. Есть две основые функции, которые нас здесь выручают:
\lstinline'aformToForm' преобразует любую аппликативную форму в монадическую,
а \lstinline'formToAForm' преобразует определённые виды монадических форм в их
аппликативные варианты.

<<\textbf{Ещё} минутку>>,~--- настаиваете вы. <<Я не видел никаких
\lstinline'aformToForm'!>>. Это тоже верно.  Функция~\lstinline'renderDivs'
позаботилась об этом для нас.

\section{Создание аппликативных форм}
Теперь, когда я (надеюсь) убедил вас, что во введении мы действительно имели
дело с аппликативными формами, давайте попытаемся понять, как они создаются.
Начнём с простого примера:

\begin{lstlisting}
data Car = Car
    { carModel :: Text
    , carYear :: Int
    }
  deriving Show

carAForm :: AForm Synopsis Synopsis Car
carAForm = Car
    <$> areq textField "Модель" Nothing
    <*> areq intField "Год выпуска" Nothing

carForm :: Html -> MForm Synopsis Synopsis (FormResult Car, Widget)
carForm = renderTable carAForm

getCarR :: Handler RepHtml
getCarR = do
    ((result, widget), enctype) <- runFormGet carForm
    case result of
        FormSuccess car -> defaultLayout [whamlet|<p>#{show car}|]
        _ -> defaultLayout [whamlet|
<form method=get action=@{CarR} enctype=#{enctype}>
    <table>
        ^{widget}
    <input type=submit>
|]
\end{lstlisting}

Здесь мы явным образом разделили аппликативные и монадические формы.
В~\lstinline'carAForm' мы использовали операторы~\lstinline'<$>'
и~\lstinline'<*>'. Это не должно удивлять; они часто используются в коде,
использующем аппликативный стиль. Также у нас по одной строчке для каждой
записи нашего типа данных~\lstinline'Car'. Возможно не удивляет, что мы
использовали \lstinline'textField' для записи с типом~\lstinline'Text' и
\lstinline'intField' для записи с типом~\lstinline'Int'.

Давайте взглянем поближе на функцию~\lstinline'areq'. Её (упрощённая)
сигнатура типа~--- \lstinline'Field a -> FieldSettings -> Maybe a -> AForm a'.
Первый аргумент определяет тип данных поля, как его разбирать и как
отображать.  Следующий аргумент, \lstinline'FieldSettings', даёт нам метку,
подсказку, имя и идентификатор поля. В данном случае, мы используем вышу
упомянутый экземпляр~\lstinline'IsString' для~\lstinline'FieldSettings'.

А что с \lstinline'Maybe a'? Этот аргумент предоставляет необязательное
значение по умолчанию.  Например, если мы хотим, чтобы наша форма подставляла
для года машины значение по умолчанию <<2007>>, мы будем использовать
\lstinline'areq intField "Year" (Just 2007)'. Мы даже можем сделать это
уровнем выше, и получить форму, которая принимает необязательный аргумент,
задающий значения по умолчанию.

\begin{lstlisting}[caption={Формы со значениями по умолчанию}]
carAForm :: Maybe Car -> AForm Synopsis Synopsis Car
carAForm mcar = Car
    <$> areq textField "Модель" (carModel <$> mcar)
    <*> areq intField "Год выпуска" (carYear <$> mcar)

carForm :: Html -> MForm Synopsis Synopsis (FormResult Car, Widget)
carForm = renderTable (carAForm $ Just $ Car "Forte" 2010)

getCarR :: Handler RepHtml
getCarR = do
    ((result, widget), enctype) <- runFormGet carForm
    case result of
        FormSuccess car -> defaultLayout [whamlet|<p>#{show car}|]
        _ -> defaultLayout [whamlet|
<form method=get action=@{CarR} enctype=#{enctype}>
    <table>
        ^{widget}
    <input type=submit>
|]
\end{lstlisting}

\subsection{Необязательные поля}
Предположим, что нам нужно необязательное поле (например, цвет машины). Всё
что нужно~--- это воспользоваться функцией \lstinline'aopt'.

\begin{lstlisting}[caption={Необязательные поля}]
data Car = Car
    { carModel :: Text
    , carYear :: Int
    , carColor :: Maybe Text
    }
  deriving Show

carAForm :: AForm Synopsis Synopsis Car
carAForm = Car
    <$> areq textField "Модель" Nothing
    <*> areq intField "Год выпуска" Nothing
    <*> aopt textField "Цвет" Nothing

carForm :: Html -> MForm Synopsis Synopsis (FormResult Car, Widget)
carForm = renderTable carAForm

getCarR :: Handler RepHtml
getCarR = do
    ((result, widget), enctype) <- runFormGet carForm
    case result of
        FormSuccess car -> defaultLayout [whamlet|<p>#{show car}|]
        _ -> defaultLayout [whamlet|
<form method=get action=@{CarR} enctype=#{enctype}>
    <table>
        ^{widget}
    <input type=submit>
|]
\end{lstlisting}

Как и в случае обязательных полей, последний аргумент~--- необязательное
значение по умолчанию.  В итоге, тут получается двойное оборачивание
в~\lstinline'Maybe'. Это может показаться чрезмерным (и так и есть), но
существенно упрощает написание кода, который принимает в качестве
необязательного параметра форму со значениями по умолчанию, как в следующем
примере.

\begin{lstlisting}[caption={Необязательные поля со значениями по умолчанию}]
data Car = Car
    { carModel :: Text
    , carYear :: Int
    , carColor :: Maybe Text
    }
  deriving Show

carAForm :: Maybe Car -> AForm Synopsis Synopsis Car
carAForm mcar = Car
    <$> areq textField "Модель" (carModel <$> mcar)
    <*> areq intField  "Год выпуска"  (carYear  <$> mcar)
    <*> aopt textField "Цвет" (carColor <$> mcar)

carForm :: Html -> MForm Synopsis Synopsis (FormResult Car, Widget)
carForm = renderTable $ carAForm $ Just $ Car "Forte" 2010 $ Just "серый"

getCarR :: Handler RepHtml
getCarR = do
    ((result, widget), enctype) <- runFormGet carForm
    case result of
        FormSuccess car -> defaultLayout [whamlet|<p>#{show car}|]
        _ -> defaultLayout [whamlet|
<form method=get action=@{CarR} enctype=#{enctype}>
    <table>
        ^{widget}
    <input type=submit>
|]
\end{lstlisting}

\section{Валидация}
Как мы можем сделать, чтобы наша форма принимала только машины, созданные
после 1990 года?  Если вы помните, мы говорили выше, что \lstinline'Field' сам
по себе содержит информацию о том, что представляет собой корректное значение. То
есть, всё что нам надо сделать~--- это написать новый \lstinline'Field',
верно? Это будет несколько утомительно. Вместо этого, давайте изменим уже
существующее поле.

\begin{lstlisting}
carAForm :: Maybe Car -> AForm Synopsis Synopsis Car
carAForm mcar = Car
    <$> areq textField    "Модель" (carModel <$> mcar)
    <*> areq carYearField "Год выпуска"  (carYear  <$> mcar)
    <*> aopt textField    "Цвет" (carColor <$> mcar)
  where
    errorMessage :: Text
    errorMessage = "Ваша машина чересчур стара, купите новую!"

    carYearField = check validateYear intField

    validateYear y
        | y < 1990 = Left errorMessage
        | otherwise = Right y

carForm :: Html -> MForm Synopsis Synopsis (FormResult Car, Widget)
carForm = renderTable $ carAForm $ Just $ Car "Forte" 2010 $ Just "серый"

getCarR :: Handler RepHtml
getCarR = do
    ((result, widget), enctype) <- runFormGet carForm
    case result of
        FormSuccess car -> defaultLayout [whamlet|<p>#{show car}|]
        _ -> defaultLayout [whamlet|
<form method=get action=@{CarR} enctype=#{enctype}>
    <table>
        ^{widget}
    <input type=submit>
|]
\end{lstlisting}

Хитрость в функции~\lstinline'check'. Она принимает функцию
(\lstinline'validateYear'), которая возвращает либо сообщение об ошибке, либо
модифицированное значение поле. В этом примере, мы не изменяли значение
вообще. Это самый частый случай использования. И так как этот вид проверки
довольно распространён, у нас есть сокращение:

\begin{lstlisting}
carYearField = checkBool (>= 1990) errorMessage intField
\end{lstlisting}

\begin{lstlisting}
carForm :: Html -> MForm Synopsis Synopsis (FormResult Car, Widget)
carForm = renderTable $ carAForm $ Just $ Car "Forte" 2010 $ Just "серый"

getCarR :: Handler RepHtml
getCarR = do
    ((result, widget), enctype) <- runFormGet carForm
    case result of
        FormSuccess car -> defaultLayout [whamlet|<p>#{show car}|]
        _ -> defaultLayout [whamlet|
<form method=get action=@{CarR} enctype=#{enctype}>
    <table>
        ^{widget}
    <input type=submit>
|]
\end{lstlisting}

Функция \lstinline'checkBool' принимает два параметра: условие, которое должно
быть выполнено, и сообщение об ошибке, которое следует отобразить, если
условие не верно.

\begin{remark}
Вы могли заметить явное указание \lstinline'Text' в сигнатуре типа функции
\lstinline'errorMessage'. При использовании расширения OverloadedStrings это
необходимо. Чтобы поддерживать интернационализацию, сообщения могут иметь
множество различных типов, и у GHC нет возможности определить, какой
экземпляр~\lstinline'IsString' вам нужен.
\end{remark}

Это замечательно~--- гарантировать, что наша машина не очень старая. А если мы
хотим убедиться в том, что указанный год не из будущего? Чтобы получить
текущий год, нам надо будет выполнить немного \lstinline'IO'. Для таких
случаев, нам потребуется функция~\lstinline'checkM':

\begin{lstlisting}
carYearField = checkM inPast $ checkBool (>= 1990) errorMessage intField

inPast y = do
    thisYear <- liftIO getCurrentYear
    return $ if y <= thisYear
        then Right y
        else Left ("Да у вас машина времени!" :: Text)

getCurrentYear :: IO Int
getCurrentYear = do
    now <- getCurrentTime
    let today = utctDay now
    let (year, _, _) = toGregorian today
    return $ fromInteger year

carForm :: Html -> MForm Synopsis Synopsis (FormResult Car, Widget)
carForm = renderTable $ carAForm $ Just $ Car "Forte" 2010 $ Just "серый"

getCarR :: Handler RepHtml
getCarR = do
    ((result, widget), enctype) <- runFormGet carForm
    case result of
        FormSuccess car -> defaultLayout [whamlet|<p>#{show car}|]
        _ -> defaultLayout [whamlet|
<form method=get action=@{CarR} enctype=#{enctype}>
    <table>
        ^{widget}
    <input type=submit>
|]
\end{lstlisting}

Функция \lstinline'inPast' вернёт результат типа \lstinline'Either'. Однако,
она использует монаду~\lstinline'Handler'. Мы используем \lstinline'liftIO getCurrentYear',
чтобы получить текущий год, и затем сравниваем его с годом, указанным
пользователем. Также обратите внимание, как мы можем связывать вместе
несколько валидаторов.

\begin{remark}
Так как валидатор \lstinline'checkM' работает в монаде \lstinline'Handler', он
имеет доступ ко всем операциям, которые вы можете обычно выполнять в Yesod.
Это особенно полезно для выполнения операций с базой данных, которые мы
рассмотрим в главе~\hyperref[chap:persistent]{Persistent}.
\end{remark}

\section{Поля посложнее}
Наше поле для ввода цвета хорошее, но не совсем дружественное пользователю.
Что мы хотим на самом деле~--- это выпадающий список.

\begin{lstlisting}[caption={Выпадающие списки}]
data Car = Car
    { carModel :: Text
    , carYear :: Int
    , carColor :: Maybe Color
    }
  deriving Show

data Color = Red | Blue | Gray | Black
    deriving (Show, Eq, Enum, Bounded)

carAForm :: Maybe Car -> AForm Synopsis Synopsis Car
carAForm mcar = Car
    <$> areq textField "Модель" (carModel <$> mcar)
    <*> areq carYearField "Год выпуска" (carYear <$> mcar)
    <*> aopt (selectFieldList colors) "Цвет" (carColor <$> mcar)
  where
    colors :: [(Text, Color)]
    colors = [("Красный", Red), ("Синий", Blue), ("Серый", Gray), ("Чёрный", Black)]
    errorMessage :: Text
    errorMessage = "Ваша машина чересчур стара, купите новую!"

    carYearField = checkM inPast $ checkBool (>= 1990) errorMessage intField

    inPast y = do
        thisYear <- liftIO getCurrentYear
        return $ if y <= thisYear
            then Right y
            else Left ("Да у вас машина времени!" :: Text)

getCurrentYear :: IO Int
getCurrentYear = do
    now <- getCurrentTime
    let today = utctDay now
    let (year, _, _) = toGregorian today
    return $ fromInteger year

carForm :: Html -> MForm Synopsis Synopsis (FormResult Car, Widget)
carForm = renderTable $ carAForm $ Just $ Car "Forte" 2010 $ Just Black

getCarR :: Handler RepHtml
getCarR = do
    ((result, widget), enctype) <- runFormGet carForm
    case result of
        FormSuccess car -> defaultLayout [whamlet|<p>#{show car}|]
        _ -> defaultLayout [whamlet|
<form method=get action=@{CarR} enctype=#{enctype}>
    <table>
        ^{widget}
    <input type=submit>
|]
\end{lstlisting}

Функция \lstinline'selectFieldList' принимает список пар. Первый элемент
пары~--- это текст, который отобразится пользователю в выпадающем списке, а
вторым элементом является соответствующее значение Haskell. Конечно, код выше
выглядит повторяющимся; мы можем получить тот же результат воспользовавшись
экземплярами \lstinline'Enum' и \lstinline'Bounded', которые GHC автоматически
предоставляет нам.

\begin{lstlisting}[caption={Используя Enum и Bounded}]
data Car = Car
    { carModel :: Text
    , carYear :: Int
    , carColor :: Maybe Color
    }
  deriving Show

data Color = Red | Blue | Gray | Black
    deriving (Show, Eq, Enum, Bounded)

carAForm :: Maybe Car -> AForm Synopsis Synopsis Car
carAForm mcar = Car
    <$> areq textField "Модель" (carModel <$> mcar)
    <*> areq carYearField "Год выпуска" (carYear <$> mcar)
    <*> aopt (selectFieldList colors) "Цвет" (carColor <$> mcar)
  where
    colors = map (pack . show &&& id) $ [minBound..maxBound]
    errorMessage :: Text
    errorMessage = "Ваша машина чересчур стара, купите новую!"

    carYearField = checkM inPast $ checkBool (>= 1990) errorMessage intField

    inPast y = do
        thisYear <- liftIO getCurrentYear
        return $ if y <= thisYear
            then Right y
            else Left ("Да у вас машина времени!" :: Text)

getCurrentYear :: IO Int
getCurrentYear = do
    now <- getCurrentTime
    let today = utctDay now
    let (year, _, _) = toGregorian today
    return $ fromInteger year

carForm :: Html -> MForm Synopsis Synopsis (FormResult Car, Widget)
carForm = renderTable $ carAForm $ Just $ Car "Forte" 2010 $ Just Black

getCarR :: Handler RepHtml
getCarR = do
    ((result, widget), enctype) <- runFormGet carForm
    case result of
        FormSuccess car -> defaultLayout [whamlet|<p>#{show car}|]
        _ -> defaultLayout [whamlet|
<form method=get action=@{CarR} enctype=#{enctype}>
    <table>
        ^{widget}
    <input type=submit>
|]
\end{lstlisting}

Интервал \lstinline'[minBound..maxBound]' даёт нам список всех различных
значений типа~\lstinline'Color'. Мы затем применяем~\lstinline'map'
и~\lstinline'&&&' (так называемый, fan-out operator)\marginpar{требуется
адекватный перевод. Или оставить как есть}, превращая его в список пар.

Некоторые люди предпочитают выпадающим спискам переключатели.  К счастью, это
изменение в одно слово. Пример в листинге~\ref{lst:radio-buttons}.

\begin{lstlisting}[caption={Переключатели},label={lst:radio-buttons}]
data Car = Car
    { carModel :: Text
    , carYear :: Int
    , carColor :: Maybe Color
    }
  deriving Show

data Color = Red | Blue | Gray | Black
    deriving (Show, Eq, Enum, Bounded)

carAForm :: Maybe Car -> AForm Synopsis Synopsis Car
carAForm mcar = Car
    <$> areq textField "Модель" (carModel <$> mcar)
    <*> areq carYearField "Год выпуска" (carYear <$> mcar)
    <*> aopt (radioFieldList colors) "Цвет" (carColor <$> mcar)
  where
    colors = map (pack . show &&& id) $ [minBound..maxBound]
    errorMessage :: Text
    errorMessage = "Ваша машина чересчур стара, купите новую!"

    carYearField = checkM inPast $ checkBool (>= 1990) errorMessage intField

    inPast y = do
        thisYear <- liftIO getCurrentYear
        return $ if y <= thisYear
            then Right y
            else Left ("Да у вас машина времени!" :: Text)

getCurrentYear :: IO Int
getCurrentYear = do
    now <- getCurrentTime
    let today = utctDay now
    let (year, _, _) = toGregorian today
    return $ fromInteger year

carForm :: Html -> MForm Synopsis Synopsis (FormResult Car, Widget)
carForm = renderTable $ carAForm $ Just $ Car "Forte" 2010 $ Just Black

getCarR :: Handler RepHtml
getCarR = do
    ((result, widget), enctype) <- runFormGet carForm
    case result of
        FormSuccess car -> defaultLayout [whamlet|<p>#{show car}|]
        _ -> defaultLayout [whamlet|
<form method=get action=@{CarR} enctype=#{enctype}>
    <table>
        ^{widget}
    <input type=submit>
|]
\end{lstlisting}

\section{Выполнение форм}
В какой-то момент у нас возникнет необходимость воспользоваться нашими
прекрасными формами и получить некоторые результаты.  Для этого в нашем
распоряжении есть целый ряд различных функций, каждая из которых имеет своё
назначение. Я пройдусь по всем, начав с самых распространённых.

\begin{description}
    \item[runFormPost] \hfill \\
        Функция выполнит вашу форму с любыми предоставленными POST
        параметрами. Если запрос не является POST--запросом, вернёт
        \lstinline'FormMissing'. Функция автоматически вставляет признак
        безопасности (security token) в виде скрытого поля для предупреждения
        кросс-сайтовых~(\footnotehref{http://en.wikipedia.org/wiki/Cross-site\_request\_forgery}{CSRF})
        атак.

    \item[runFormGet] \hfill \\
        Аналог \lstinline'runFormPost' для GET-параметров. Чтобы различать
        обычную загрузку страницы через GET от отправки GET формы, включает в
        форму дополнительное скрытое поле~\lstinline'_hasdata'.

    \item[runFormPostNoNonce] \hfill \\
        То же что и \lstinline'runFormPost', но не включает (или не требует)
        признака безопасности CSRF.

    \item[generateFormPost] \hfill \\
        Вместо привязки к существующим POST параметрам, действует так, будто
        их нет. Может быть полезна, когда вам нужно сгенерировать новую форму
        после того, как предыдущая была отправлена, например, в мастере форм.

    \item[generateFormGet] \hfill \\
        То же что и \lstinline'generateFormPost', но для GET.
\end{description}

Тип возвращаемого значения первых трёх функций~---
\lstinline'((FormResult a, Widget), Enctype)'.  \lstinline'Widget' уже будет
включать любые ошибки валидации и ранее отправленные значения.

\section{Интернационализация}
В этой главе уже было несколько ссылок на интернационализацию. Эта тема
подробнее освещена в своей собственной~\hyperref[chap:i18n]{главе}, но так как
она имеет глубокое влияние на yesod-form, я хотел дать краткий обзор.  Идея в
основании поддержки интернационализации в Yesod состоит в том, чтобы иметь
типы данных для представления сообщений. Каждый сайт может иметь
экземпляр~\lstinline'RenderMessage' для конкретного типа данных, который будет
переводить сообщение в соответствии со списком языков, принимаемых клиентом.
Как результат, есть несколько моментов, о которых вам следует знать:
\begin{itemize}
    \item  Для каждого сайта автоматически создаётся
        экземпляр~\lstinline'RenderMessage' для типа~\lstinline'Text', так что
        вы можете просто использовать простые строки, если не заботитесь о
        поддержке интернационализации. Однако, вам изредка может потребоваться
        использовать явные сигнатуры типов.

    \item Все сообщения в yesod-form выражаются в терминах типа
        данных~\lstinline'FormMessage'.  Поэтому, для использования yesod-form
        вам нужен соответствующий экземпляр~\lstinline'RenderMessage'. Простая
        реализация, которая использует английский вариант перевода по
        умолчанию, будет выглядеть так:
\begin{lstlisting}
    instance RenderMessage MyApp FormMessage where
    renderMessage _ _ = defaultFormMessage
\end{lstlisting}
        Такой экземпляр автоматически предоставляется сгенерированным каркасом
        сайта.
\end{itemize}

\section{Монадические формы}
Зачастую, простая вёрстка форм достаточна, и тут аппликативные формы
превосходны. Иногда, однако, вы захотите получить более настраиваемый вид вашей
формы.
\begin{figure}[tbph]
  \centering
  \includegraphics[width=\textwidth]{08-forms-image-01.png}
  \caption{Нестандартная вёрстка формы}
\end{figure}

В таких случаях следует использовать монадические формы. Они несколько более
многословны, чем их аппликативные родственники, но эта многословность
позволяет вам получить полный контроль над тем, как форма будет выглядеть.
Чтобы получить такую же форму, как приведённая выше, мы могли бы написать
что-то подобное этому.
\lstinputlisting{../hs/08/monadic-forms.hs}

Подобно аппликативной функции~\lstinline'areq' мы используем \lstinline'mreq'
для монадических форм.  (И, конечно, есть \lstinline'mopt' для необязательных
полей).  Но здесь есть существенное различие: \lstinline'mreq' возвращает нам
пару значений.  Вместо скрытого значения типа
\footnotehref{http://hackage.haskell.org/packages/archive/yesod-form/latest/doc/html/Yesod-Form-Types.html\#t:FieldView}{FieldView}
и автоматической его вставки в виджет, мы получаем управление вставкой по
своему усмотрению.

Тип данных \lstinline'FieldView' содержит ряд записей с информацией о поле
формы. Самая важная из них~--- \lstinline'fvInput', которая и есть фактическое
поле формы. В этом примере мы также использовали~\lstinline'fvId', которая
возвращает нам HTML атрибут~\texttt{id} тегa~\texttt{input}. В нашем примере
мы её использовали, чтобы задать ширину поля.

Вас, наверное, интересует, что это за значения <<Это не используется>> и <<И это
тоже>>.  Функция~\lstinline'mreq' принимает вторым аргументом значение
типа~\lstinline'FieldSettings'.  Так как для \lstinline'FieldSettings'
существует экземпляр \lstinline'IsString', такие строки по сути раскрываются
компилятором в:

\begin{lstlisting}
fromString "Это не используется" == FieldSettings
    { fsLabel = "Это не используется"
    , fsTooltip = Nothing
    , fsId = Nothing
    , fsName = Nothing
    , fsClass = []
    }
\end{lstlisting}

В случае аппликативных форм, значения~\lstinline'fsLabel'
и~\lstinline'fsTooltip' использовались при построении HTML. В случае же
монадических форм, Yesod не генерирует для нас никаких HTML-обёрток, и поэтому
эти значения игнорируются. Однако, мы храним параметр
типа~\lstinline'FieldSettings', чтобы позволить вам при желании переопределять
атрибуты~\texttt{id} и~\texttt{name} ваших полей.

Ещё один интересный момент~--- значение~\lstinline{extra}. Формы типа GET
добавляют поле для обозначения отправки, а POST формы добавляют признак
безопасности для предупреждения кросс-сайтовых атак.  Если вы не включите это
дополнительное скрытое поле, то Yesod не примет вашу форму.

Всё остальное довольно-таки тривиально. Мы создаём наше
значение~\lstinline'personRes' из значений~\lstinline'nameRes'
и~\lstinline'ageRes', а затем возвращаем кортеж, включающий полученное
значение типа~\lstinline'Person' и виджет для формы. А в
функции~\lstinline'getRootR' всё также, как для аппликативной формы. В
действительности, вы могли бы заменить монадическую форму на
аппликативную, и код всё равно работал бы.

\section{Формы ввода данных}
Аппликативные и монадические формы выполняют и генерацию HTML, и разбор
данных.  Иногда вам нужно только последнее, например, когда где-то уже создана
форма, или если вы хотите генерировать форму динамически, используя
Javascript. В этих случаях вам пригодятся формы ввода данных.

Они аналогичны аппликативным и монадическим с некоторыми различиями:
\begin{itemize}
    \item Вы используете \lstinline'runInputPost' и \lstinline'runInputGet'.
    \item Вы используете \lstinline'ireq' и \lstinline'iopt'. Эти функции
        теперь принимают только 2 аргумента: тип рассматриваемого поля и его
        имя (т.е. значение HTML атрибута~\texttt{name}).
    \item После выполнения формы, она возвращает значение. Она не возвращает
        виджет или тип кодировки.
    \item В случае ошибок проверки данных, возвращается страница с сообщением
        об ошибке <<некорректные аргументы (invalid arguments)>>.
\end{itemize}
Вы можете переписать предыдущий пример с помощью форм ввода данных. Однако,
стоит заметить, что эта версия менее дружественна пользователю. Если вы
допустите ошибку при заполнении аппликативной или монадической формы, то вы
вернётесь на ту же страницу с ранее введёнными данными, а сообщение об ошибке
укажет, что надо исправить. При использовании форм ввода данных пользователь
получит просто сообщение об ошибке.

\lstinputlisting{../hs/08/input-forms.hs}

\section{Пользовательские поля}
Поля, которые встроены в Yesod, скорее всего покроют большую часть ваших
потребностей для работы с формами. Но иногда вам может понадобиться что-то
более специализированное.  К счастью, вы можете создавать новые поля в Yesod
самостоятельно. Тип данных~\lstinline'Field' имеет две записи:
\lstinline'fieldParse' принимает список значений, отправленных пользователем,
и возвращает один из трёх результатов:
\begin{itemize}
    \item Сообщение об ошибке в случае, если проверка данных провалилась.
    \item Разобранное значение.
    \item \lstinline'Nothing', сообщая, что данные не были предоставлены.
\end{itemize}

Последний случай может несколько удивить: разве Yesod не может автоматически
понять, что информации не было предоставлено, если входной список пуст? Не
совсем. Чекбоксы, например, сообщают о сброшенном состоянии, посылая пустой
список.

Почему список? Не лучше ли \lstinline{Maybe}? Не всегда. В случае группы
чекбоксов или списков с возможностью множественного выбора у вас будет
множество виджетов с одним именем. Мы используем этот трюк в нашем примере
ниже.

Вторая запись~--- \lstinline'fieldView'~--- формирует виджет для показа
пользователю. У этой функции четыре аргумента: атрибут~\texttt{id},
атрибут~\texttt{name}, результат и значение типа~\lstinline'Bool' для
обозначения, является ли поле обязательным.

Что мы подразумеваем под результатом? В действительности, это значение
типа~\lstinline'Either', возвращающее либо неразобранный ввод (когда разбор не
удался), либо успешно разобранное значение. Великолепный пример этому~--- поле
\lstinline'intField'.  Если вы введёте 42, то в результате будет
\lstinline'Right 42'. Но если вы введёте слово <<черепаха>>, результатом будет
\lstinline'Left "черепаха"'. Это позволит вам заполнить атрибут~\texttt{value}
вашего тега~\texttt{input} для организации единообразного пользовательского
интерфейса.

В качестве маленького примера мы создадим новый тип поля~--- поле для
подтверждения пароля. В нём будет два элемента для ввода текста (оба с одним и
тем же значением атрибута~\texttt{name}), и оно будет возвращать сообщение об
ошибке, если пароли не совпадают. Заметьте, что в отличие от большинства
полей, новое поле \emph{не} предоставляет атрибуты~\texttt{value} для
тегов~\texttt{input}, потому что вы \textbf{никогда} не захотите отправлять
обратно введённый пользователем пароль в своём HTML.

\begin{lstlisting}
passwordConfirmField :: Field sub master Text
passwordConfirmField = Field
    { fieldParse = \rawVals _ ->
        case rawVals of
            [a, b]
                | a == b -> return $ Right $ Just a
                | otherwise -> return $ Left "Пароли не совпадают"
            [] -> return $ Right Nothing
            _ -> return $ Left "Вы должны ввести два значения"
    , fieldView = \idAttr nameAttr _ eResult isReq -> [whamlet|
<input id=#{idAttr} name=#{nameAttr} type=password>
<div>Confirm:
<input id=#{idAttr}-confirm name=#{nameAttr} type=password>
|]
    , fieldEnctype = UrlEncoded
    }

getRootR :: Handler RepHtml
getRootR = do
    ((res, widget), enctype) <- runFormGet $ renderDivs
        $ areq passwordConfirmField "Пароль" Nothing
    defaultLayout [whamlet|
<p>Result: #{show res}
<form enctype=#{enctype}>
    ^{widget}
    <input type=submit value="Сменить пароль">
|]
\end{lstlisting}

\section{Выводы}
Формы в Yesod делятся на три вида. Аппликативные используются чаще всего, так
как предоставляют красивый интерфейс и простой для использования API.
Монадические формы дают больше возможностей, но их сложнее использовать. Формы
ввода данных полезны, когда вам надо просто принять данные пользователя, не
генерируя сложных виджетов.

Из коробки Yesod предоставляет несколько различных типов полей для форм. Для
их использования вам необходимо указать вид формы и является ли поле
обязательным или нет. Для этого есть шесть вспомогательных функций:
\lstinline'areq', \lstinline'aopt', \lstinline'mreq', \lstinline'mopt',
\lstinline'ireq' и~\lstinline'iopt'.

Формы обладают большой мощью. Они могут автоматически вставлять код на
Javascript, чтобы помочь вам получить элементы управления попривлекательнее,
как, например, диалог выбора даты из библиотеки jQuery UI. Формы также
полностью готовы для включения интернационализации, так что вы можете
поддерживать глобальное сообщество пользователей. А в случае более
специфических потребностей, вы можете привязать функции валидации данных для
конкретных полей или реализовать своё поле с чистого листа.

