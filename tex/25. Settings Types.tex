\chapter{Settings Types}
Let's say you're writing a webserver. You want the server to take a port to listen on, and an
application to run. So you create the following function:
\begin{lstlisting}
run :: Int -> Application -> IO ()
\end{lstlisting}

But suddenly you realize that some people will want to customize their timeout durations. So
you modify your API:
\begin{lstlisting}
run :: Int -> Int -> Application -> IO ()
\end{lstlisting}

So, which Int is the timeout, and which is the port? Well, you could create
some type aliases, or comment your code. But there's another problem creeping into our code: this
run function is getting unmanageable. Soon we'll need to take an extra
parameter to indicate how exceptions should be handled, and then another one to control which
host to bind to, and so on.

So a more extensible solution is to introduce a settings datatype:
\begin{lstlisting}
data Settings = Settings
    { settingsPort :: Int
    , settingsHost :: String
    , settingsTimeout :: Int
    }
\end{lstlisting}
And this makes the calling code almost self-documenting:
\begin{lstlisting}
run Settings
    { settingsPort = 8080
    , settingsHost = "127.0.0.1"
    , settingsTimeout = 30
    } myApp
\end{lstlisting}

Great, couldn't be clearer, right? True, but what happens when you have 50 settings to your
webserver. Do you really want to have to specify all of those each time? Of course not. So
instead, the webserver should provide a set of defaults:
\begin{lstlisting}
defaultSettings = Settings 3000 "127.0.0.1" 30
\end{lstlisting}
And now, instead of needing to write that long bit of code above, we can get away with:
\begin{lstlisting}
run defaultSettings { settingsPort = 8080 } myApp -- (1)
\end{lstlisting}

This is great, except for one minor hitch. Let's say we now decide to add an extra record to
Settings. Any code out in the wild looking like this:
\begin{lstlisting}
run (Settings 8080 "127.0.0.1" 30) myApp -- (2)
\end{lstlisting}
will be broken, since the Settings constructor now takes 4 arguments. The proper
thing to do would be to bump the major version number so that dependent packages don't get
broken. But having to change major versions for every minor setting you add is a nuisance. The
solution? Don't export the Settings constructor:
\begin{lstlisting}
module MyServer
    ( Settings
    , settingsPort
    , settingsHost
    , settingsTimeout
    , run
    , defaultSettings
    ) where
\end{lstlisting}

With this approach, no one can write code like (2), so you can freely add new records without
any fear of code breaking.

The one downside of this approach is that it's not immediately obvious from the Haddocks that
you can actually change the settings via record syntax. That's the point of this chapter: to
clarify what's going on in the libraries that use this technique.

I personally use this technique in a few places, feel free to have a look at the Haddocks to
see what I mean.


Warp:
  warp:   Network.Wai.Handler.Warp:Settings
  http-conduit: http-conduit:Network.HTTP.Conduit:Request and
  http-conduit: Network.HTTP.Conduit:ManagerSettings
  xml-conduit   Parsing: xml-conduit:Text.XML.Stream.Parse:ParseSettings
Rendering:
  xml-conduit:Text.XML.Stream.Render:RenderSettings

As a tangential issue, http-conduit and xml-conduit actually
create instances of the data-default:Data.Default:Default typeclass instead of
declaring a brand new identifier. This means you can just type def instead of
defaultParserSettings.
