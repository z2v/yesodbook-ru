\chapter{Создание дочернего сайта}\label{chap:subsite}

Сколько сайтов требуют систему аутентификации? Или функции управления данными (CRUD)? Или блог? Или вики?

Идея в том, что многие веб-сайты включают общие компоненты, которые можно использовать для нескольких сайтов. Однако часто бывает довольно трудно получить модульный код, который действительно был бы plug-and-play: такая компонента, вероятно, потребует включения в систему маршрутизации нескольких маршрутов, а также от нее потребуется соответствовать стилю основного сайта.

Решением в Yesod являются подсайты. Подсайт представляет собой набор маршрутов и их обработчиков, которые могут быть легко включены в основной сайт. Используя классы типов, легко убедиться, что основной сайт предоставляет определенные возможности. Также с их помощью несложно получить доступ к стандартной разметке сайта. В свою очередь типобезопасные URL позволяют с легкостью ссылаться с основного сайта на подсайты.

\section{Привет, мир!}

Создание подсайтов достаточно мудрёный процесс, вовлекающий ряд различных типов. Давайте начнем с простого приложения <<Привет, мир!>>:
\includecode{17/hellosub.hs}

Этот очень простой пример на самом деле демонстрирует большинство сложностей, связанных с созданием подсайта. Как и в обычном приложении Yesod, в подсайте всё сосредоточено вокруг основного типа данных, в нашем случае HelloSub. Затем мы используем mkYesodSub, во многом так же, как мы используем mkYesod, чтобы создать тип данных маршрута и функции деспетчеризации/рендеринга. (Мы еще вернемся к дополнительному параметру через секунду).

Что интересно, так это сигнатура типа getSubRootR. До сих пор мы старались не обращать внимания на тип данных GHandler, или, в случае необходимости, делать вид, что первые два аргумента типа всегда одинаковы. Теперь мы должны, наконец, узнать правду об этом странном типе данных.

У функции обработчика всегда есть два основных типа, связанных с ней: подсайт и основной сайт. Когда вы пишете обычное приложение, эти два типа данных одинаковы.  Однако, если вы работаете с подсайтом, они обязательно будут отличаться. Таким образом, сигнатура типа для getSubRootR использует HelloSub для первого аргумента и master для второго.

Функция DefaultLayout принадлежит классу типов Yesod. Таким образом, для того, чтобы её вызывать, аргумент типа master должен быть экземпляром Yesod.  Преимуществом такого подхода заключается в том, что любые изменения в методе defaultLayout основного сайта будут автоматически отражены в подсайтах.

Когда мы включаем подсайт в определение маршрутов нашего основного сайта, мы должны определить четыре вещи: маршрут, используемый подсайтом в качестве базового (в данном случае /subsite), конструктор для маршрутов подсайта (SubsiteR), основной тип данных (HelloSub) для подсайта и функцию, которая принимает основное значение основного сайта и возвращает основное значение подсайта (getHelloSub).

В определении getRootR, мы можем видеть, как используется конструктор маршрута. В некотором смысле, SubsiteR переводит любой маршрут подсайта в маршрут основного сайта, что позволяет безопасно ссылаться на него из любого шаблона основного сайта.
