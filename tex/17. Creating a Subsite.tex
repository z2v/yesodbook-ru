\chapter{Создание подсайта}\label{chap:subsite}

Сколько сайтов требуют систему аутентификации? Или функции управления данными (CRUD)? Или блог? Или вики?

Идея в том, что многие веб-сайты включают общие компоненты, которые можно использовать для нескольких сайтов. Однако часто бывает довольно трудно получить модульный код, который действительно был бы plug-and-play: такой компонент, вероятно потребует включения в систему маршрутизации нескольких маршрутов, а также от него потребуется соответствовать стилю основного сайта.

Решением в Yesod являются подсайты. Подсайт представляет собой набор маршрутов и их обработчиков, которые могут быть легко включены в основной сайт. Используя классы типов, легко убедиться, что основной сайт предоставляет определённые возможности. Также с их помощью несложно получить доступ к стандартной разметке сайта. В свою очередь типобезопасные URL позволяют с лёгкостью ссылаться с основного сайта на подсайты.

\section{Привет, мир!}
Возможно, главная сложность в создании подсайтов~--- это первоначальное знакомство. Поэтому давайте сразу начнём с простого приложения <<Привет, мир!>>. Нам потребуется сделать один модуль для данных подсайта, другой~--- для кода диспетчеризации, и затем ещё один модуль для приложения, которое использует наш подсайт.

\begin{remark}
Причина разделения на данные и код диспетчеризации~--- нечто, называемое <<ограничением стадий GHC>> (GHC stage restriction). Это требование делает маленькие демо программки многословнее, но на практике, разделение на множество модулей~--- хорошая практика для следования.
\end{remark}

\includecode{17/HelloSub/Data.hs}

\includecode{17/HelloSub.hs}

Этот очень простой пример на самом деле демонстрирует большинство сложностей, связанных с созданием подсайта. Как и в обычном приложении Yesod, в подсайте всё сосредоточено вокруг основного типа данных, в нашем случае \lstinline!HelloSub!. Затем мы используем \lstinline!mkYesodSubData!, чтобы создать тип данных маршрута и связанные функции диспетчеризации/рендеринга.

Что касается диспетчеризации, мы начинаем с определения функции-обработчика для маршрута~\lstinline'SubHomeR'. Обратите особое внимание на сигнатуру типа этой функции:
\begin{lstlisting}
getSubHomeR :: Yesod master
            => HandlerT HelloSub (HandlerT master IO) Html
\end{lstlisting}

Это суть всего, относящегося к подсайтам. Все наши действия находятся в этой многоуровневой монаде, в которой мы оборачиваем наш подсайт вокруг основного сайта. Наличие уровней естественным образом приводит к использованию функции~\lstinline'lift'. В нашем случае, подсайт использует функцию~\lstinline'defaultLayout' основного сайта для отображения виджета.

Функция \lstinline!defaultLayout! принадлежит классу типов Yesod. Таким образом, для того, чтобы её вызывать, аргумент типа \lstinline!master! должен быть экземпляром \lstinline!Yesod!.  Преимуществом такого подхода заключается в том, что любые изменения в методе \lstinline!defaultLayout! основного сайта будут автоматически отражены в подсайтах.

Когда мы включаем подсайт в определение маршрутов нашего основного сайта, мы должны определить четыре вещи: маршрут, используемый подсайтом в качестве базового (в данном случае \lstinline!/subsite!), конструктор для маршрутов подсайта (\lstinline!SubsiteR!), основной тип данных (\lstinline!HelloSub!) для подсайта и функцию, которая принимает основное значение основного сайта и возвращает основное значение подсайта (\lstinline!getHelloSub!).

В определении getRootR мы можем видеть, как используется конструктор маршрута. В некотором смысле, \lstinline!SubsiteR! переводит любой маршрут подсайта в маршрут основного сайта, что позволяет безопасно ссылаться на него из любого шаблона основного сайта.
