\chapter{Шекспировские шаблоны}\label{ch:shakespeare}
%FIXME: Shakespearean Templates -> Шекспировские шаблоны

Yesod использует Шекспировское семейство шаблонных языков как стандартный 
подход к созданию HTML, CSS и Javascript. Это семейство языков имеет похожий 
синтаксис и общепринятые принципы: 

\begin{itemize}
%FIXME: as well as overarching princeples -> и общепринятые принципы
\item Как можно меньшее вмешательство в язык, на которых шаблонные языки 
основываются, но в то же время использование преимуществ этого языка.
%FIXME: тавтология
\item Гарантии корректности контента обеспечиваются компилятором.
%XXX:compile-time guarantees on well-formed content -> Гарантии корректности контента обеспечиваются компилятором.
\item Предоставляемая статической типизацией безопасность, которая также 
предотвращает \texttt{XSS} (cross-site scripting) атаки.
\item Автоматическая проверка валидности URL-ов, где это возможно, с помощью 
типобезопасных URL-ов.
%FIXME:type-safe URLs -> типобезопасных URL-ов
\end{itemize}

По сути, ничего не связывает Yesod с этими языками, другими словами, и языки, 
и Yesod можно использовать по-отдельности. Данная глава будет рассматривать 
эти шаблонные языки сами по себе, в то время как оставшаяся часть книги 
будет их использовать для разработки приложений для Yesod.
%FIXME: to enhance Yesod application development -> для разработки приложений на Yesod

\section{Краткий обзор}
%FIXME: Synopsis -> Краткий обзор

Всего в игре 4 основных языка: Hamlet - это шаблонный язык HTML, 
Julius - для Javascript, Cassius и Lucius - оба для CSS. Hamlet и Cassius - 
%FIXME: Hamlet and Cassius are both whitespace sensitive formats -> два языка, чувствительные к форматированию
два языка, чувствительные к форматированию, использующие отступы для 
%XXX: use indentation to denote nesting -> использовать отступы для обозначения вложенных блоков
обозначения вложенных блоков. Lucius же, являясь подмножеством CSS, использует
%XXX: Lucius is a superset of CSS -> Lucius является подмножеством CSS
фигурные скобки для обозначения вложенных блоков. 
Julius - это простой однопроходный язык, который служит для 
%FIXME: simple passthrough language -> простой однопроходный язык
генерирования Javascript; единственная добавочная функциональность - это 
интерполяция переменных.
%XXX: variable interpolation -> интерполяция переменных

\subsection{Hamlet (HTML)}

\begin{lstlisting}
$doctype 5
<html>
    <head>
        <title>#{pageTitle} - My Site
        <link rel=stylesheet href=@{Stylesheet}>
    <body>
        <h1 .page-title>#{pageTitle}
        <p>Here is a list of your friends:
        $if null friends
            <p>Sorry, I lied, you don't have any friends.
        $else
            <ul>
                $forall Friend name age <- friends
                    <li>#{name} (#{age} years old)
        <footer>^{copyright}
\end{lstlisting}

\subsection{Cassius (CSS)}

\begin{lstlisting}
#myid
    color: #{red}
    font-size: #{bodyFontSize}
foo bar baz
    background-image: url(@{MyBackgroundR})
\end{lstlisting}

\subsection{Lucius (CSS)}

\begin{lstlisting}
section.blog {
    padding: 1em;
    border: 1px solid #000;
    h1 {
        color: #{headingColor};
    }
}
\end{lstlisting}

\subsection{Julius (Javascript)}

\begin{lstlisting}
$(function(){
    $("section.#{sectionClass}").hide();
    $("#mybutton").click(function(){document.location = "@{SomeRouteR}";});
    ^{addBling}
});
\end{lstlisting}
%FIXME: что-то с форматированием lstlisting в vim (мешает $)

\section{Types}

Прежде чем мы перейдем к синтаксису давайте взглянем на различные используемые 
типы данных. Мы уже обсуждали~\ref{ch:introduction} во вступлении, что типы 
помогают нам защищаться от \texttt{XSS} атак. К примеру, скажем, 
у нас есть HTML шаблон, который должен отображать чьë-то имя. Он может выглядеть
как-то так:

\begin{lstlisting}
<p>Hello, my name is #{name}
\end{lstlisting}


\lstinline'#{...}' - это способ интерполяции переменных в Shakespeare.

Что должно произойти с имененем, и какого оно должно быть типа данных?
Самое простое решение - использовать \textt{Text} переменную, и вставлять 
еë значение в код.  
%FIXME: naive approach - самое простое решение
%FIXME: use Text value and insert it verbatim  - использовать Text переменную и вставлять ее значение в код
Но это вызовет ряд проблем для 
\lstinline!name="<script src='http://nefarious.com/evil.js'></script>"!.
Что мы хотим - это иметь возможность кодировать имя, так что 
%FIXME: entity-encode - кодировать
\lstinline'<' становится \lstinline'&lt;'.

Таким же на наивным решением было бы просто кодировать каждый кусок текста,
который 





%TODO: for WORD in HTML, CSS, Javascript, Yesod, Hamlet, Julius, Casius, Lucius, Shakespeare: 
%TODO:       WORD -> \texttt{WORD}
%TODO: йо -> ë

