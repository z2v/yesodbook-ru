\chapter{Маршрутизация URL и обработчики}
Если мы посмотрим на Yesod как на веб-фреймворк, реализующий схему
Модель-Вид-Контроллер, маршрутизация URL и обработчики запросов
формируют часть <<Контроллер>>. Для сравнения, опишем два других
подхода к маршрутизации, используемых в других средах веб разработки:
\begin{itemize}
\item Диспетчеризация обработчиков, основанная на имени файла. Так,
  например, работают PHP и ASP.

\item Используется централизованная функция, которая обрабатывает
  маршруты, опираясь на регулярные выражения. Django и Rails
  следуют такому подходу.
\end{itemize}

Yesod ближе, в принципе, к последней технике. Однако, имеются
значительные отличия. Вместо использования регулярных выражений, Yesod
сопоставляет участки маршрута. А вместо однонаправленного отображения
маршрут-обработчик, Yesod использует промежуточный тип данных
(называемый маршрутный тип данных, или типобезопасный URL) и создаёт
функции для двухстороннего преобразования.

Ручная реализация этой более продвинутой системы утомительна и
подвержена ошибкам. Поэтому, Yesod определяет предметно-ориентированный
язык (Domain Specific Language (DSL)) для описания маршрутов и
предоставляет функции Template Haskell для преобразования этого DSL в
код на Haskell. В этой главе будет объяснён синтаксис объявлений
маршрута, дано некоторое представление, какой код генерируется для
вас, и описано взаимодействие между маршрутизацией и функциями
обработки запросов.

\section{Синтаксис записи маршрута}

Instead of trying to shoe-horn route declarations into an existing
syntax, Yesod's approach is to use a simplified syntax designed just
for routes. This has the advantage of making the code not only easy to
write, but simple enough for someone with no Yesod experience to read
and understand the sitemap of your application.

A basic example of this syntax is:

\begin{verbatim}
/             RootR     GET
/blog         BlogR     GET POST
/blog/#BlogId BlogPostR GET POST

/static       StaticR   Static getStatic
\end{verbatim}

The next few sections will explain the full details of what goes on in
the route declaration.

\subsection{Участки пути URL}

One of the first thing Yesod does when it gets a request is split up
the requested path into pieces. The pieces are tokenized at all
forward slashes. For example:

\begin{lstlisting}
toPieces "/" = []
toPieces "/foo/bar/baz/" = ["foo", "bar", "baz", ""]
\end{lstlisting}

You may notice that there are some funny things going on with trailing
slashes, or double slashes ("/foo//bar//"), or a few other
things. Yesod believes in having canonical URLs; if someone requests a
URL with a trailing slash, or with a double slash, they automatically
get a redirect to the canonical version. This ensures you have one URL
for one resource, and can help with your search rankings.

What this means for you is that you needn't concern yourself with the
exact structure of your URLs: you can safely think about pieces of a
path, and Yesod automatically handles intercalating the slashes and
escaping problematic characters.

If, by the way, you want more fine-tuned control of how paths are
split into pieces and joined together again, you'll want to look at
the cleanPath and joinPath methods in the Yesod typeclass chapter.

\subsubsection{Типы участков}

When you are declaring your routes, you have three types of pieces at
your disposal:

Static

This is a plain string that must be matched against precisely in the
URL.

Dynamic single

This is a single piece (ie, between two forward slashes), but can be a
user-submitted value. This is the primary method of receiving extra
user input on a page request. These pieces begin with a hash (\#) and
are followed by a data type. The datatype must be an instance of
PathPiece.

Dynamic multi

The same as before, but can receive multiple pieces of the URL. This
must always be the last piece in a resource pattern. It is specified
by an asterisk (*) followed by a datatype, which must be an instance
of PathMultiPiece. Multi pieces are not as common as the other two,
though they are very important for implementing features like static
trees representing file structure or wikis with arbitrary
hierarchies.

Let us take a look at some standard kinds of resource patterns you may
want to write. Starting simply, the root of an application will just
be /. Similarly, you may want to place your FAQ at /page/faq.

Now let's say you are going to write a Fibonacci website. You may
construct your URLs like /fib/\#Int. But there's a slight problem with
this: we do not want to allow negative numbers or zero to be passed
into our application. Fortunately, the type system can protect us:

\begin{lstlisting}
newtype Natural = Natural Int
    deriving (Show, Read, Eq, Num, Ord)
instance PathPiece Natural where
    toPathPiece (Natural i) = T.pack $ show i
    fromPathPiece s =
        case reads $ T.unpack s of
            (i, ""):_
                | i < 1 -> Nothing
                | otherwise -> Just $ Natural i
            [] -> Nothing
mkYesod "Fibs" [parseRoutes|
/fibs/#Natural FibsR GET
|]
instance Yesod Fibs
fibs = 1 : 1 : zipWith (+) fibs (tail fibs)
getFibsR :: Natural -> GHandler Fibs Fibs RepPlain
getFibsR (Natural i) = return $ RepPlain $ toContent $ show $ fibs !! (i - 1)
main = warpDebug 3000 Fibs
\end{lstlisting}

On line 1 we define a simple newtype wrapper around Int to protect
ourselves from invalid input. We can see that PathPiece is a typeclass
with two methods. toPathPiece does nothing more than convert to a
Text. fromPathPiece attempts to convert a Text to our datatype,
returning Nothing when this conversion is impossible. By using this
datatype, we can ensure that our handler function is only ever given
natural numbers, allowing us to once again use the type system to
battle the boundary issue.

In a real life application, we would also want to ensure we never
accidently constructed an invalid Natural value internally to our
app. To do so, we could use an approach like smart constructors. For
the purposes of this example, we've kept the code simple.

Defining a PathMultiPiece is just as simple. Let's say we want to have a Wiki with at least two levels of hierarchy; we might define a datatype such as:

\begin{lstlisting}
data Page = Page Text Text [Text] -- 2 or more
instance PathMultiPiece Page where
    toPathMultiPiece (Page x y z) = x : y : z
    fromPathMultiPiece (x:y:z) = Just $ Page x y z
    fromPathMultiPiece _ = Nothing
main = return ()
\end{lstlisting}

\subsection{Наименование ресурса}

Each resource pattern also has a name associated with it. That name
will become the constructor for the type safe URL datatype associated
with your application. Therefore, it has to start with a capital
letter. By convention, these resource names all end with a capital
R. There is nothing forcing you to do this, it is just common
practice.

The exact definition of our constructor depends upon the resource
pattern it is attached to. Whatever datatypes are included in single
and multi pieces of the pattern become arguments to the datatype. This
gives us a 1-to-1 correspondence between our type safe URL values and
valid URLs in our application.

This doesn't necessarily mean that every value is a working page, just
that it is is a potentially valid URL. As an example, that value
PersonR "Michael" may not resolve to a valid page if there is no
Michael in the database.

Let's get some real examples going here. If you had the resource
patterns /person/\#Text named PersonR, /year/\#Int named YearR and
/page/faq named FaqR, you would end up with a route data type roughly
looking like:

\begin{lstlisting}
data MyRoute = PersonR Text
             | YearR Int
             | FaqR
\end{lstlisting}

If a user requests the relative URL of /year/2009, Yesod will convert
it into the value YearR 2009. /person/Michael becomes PersonR
"Michael" and /page/faq becomes FaqR. On the other hand,
/year/two-thousand-nine, /person/michael/snoyman and /page/FAQ would
all result in 404 errors without ever seeing your code.

\section{Спецификация обработчика}

The last piece of the puzzle when declaring your resources is how they
will be handled. There are three options in Yesod:

* A single handler function for all request methods on a given route.
* A separate handler function for each request method on a given route. Any other request method will generate a 405 Bad Method response.
* You want to pass off to a subsite.

The first two can be easily specified. A single handler function will
be a line with just a resource pattern and the resource name, such as
/page/faq FaqR. In this case, the handler function must be named
handleFaqR.

A separate handler for each request method will be the same, plus a
list of request methods. The request methods must be all capital
letters. For example, /person/\#String PersonR GET POST DELETE. In this
case, you would need to define three handler functions: getPersonR,
postPersonR and deletePersonR.

Subsites are a very useful— but complicated— topic in Yesod. We will
cover writing subsites later, but using them is not too difficult. The
most commonly used subsite is the static subsite, which serves static
files for your application. In order to serve static files from
/static, you would need a resource line like:

\begin{verbatim}
/static StaticR Static getStatic
\end{verbatim}

In this line, /static just says where in your URL structure to serve
the static files from. There is nothing magical about the word static,
you could easily replace it with /my/non-dynamic/files.

The next word, StaticR, gives the resource name. The next two words
are what specify that we are using a subsite. Static is the name of
the subsite foundation datatype, and getStatic is a function that gets
a Static value from a value of your master foundation datatype.

Let's not get too caught up in the details of subsites now. We will
look more closely at the static subsite in the scaffolded site
chapter.

%FIXME: уточнить перевод для dispatch
\section{Отправка}
Once you have specified your routes, Yesod will take care of all the
pesky details of URL dispatch for you. You just need to make sure to
provide the appropriate handler functions. For subsite routes, you do
not need to write any handler functions, but you do for the other
two. We mentioned the naming rules above (MyHandlerR GET becomes
getMyHandlerR, MyOtherHandlerR becomes handleMyOtherHandlerR). Now we
need the type signature.

\subsection{Возвращаемый тип}

Let's look at a simple handler function:

\begin{lstlisting}
mkYesod "Simple" [parseRoutes|
/ HomeR GET
|]

getHomeR :: Handler RepHtml
getHomeR = defaultLayout [whamlet|<h1>This is simple
|]
instance Yesod Simple
main = warpDebug 3000 Simple
\end{lstlisting}

Look at the type signature of getHomeR. The first component is
Handler. Handler is a special monad that all handler functions live
in. It provides access to request information, let's you send
redirects, and lots of other stuff we'll get to soon.

Next we have RepHtml. When we discuss representations we will explore
the why of things more; for now, we are just interested in the how.

As you might guess, RepHtml is a datatype for HTML responses. And as
you also may guess, web sites need to return responses besides
HTML. CSS, Javascript, images, XML are all necessities of a
website. Therefore, the return value of a handler function can be any
instance of HasReps.

HasReps is a powerful concept that allows Yesod to automatically
choose the correct representation of your data based on the client
request. For now, we will focus just on simple instances such as
RepHtml, which only provide one representation.

\subsection{Аргументы}

Not every route is as simple as the HomeR we just defined. Take for
instance our PersonR route from earlier. The name of the person needs
to be passed to the handler function. This translation is very
straight-forward, and hopefully intuitive. For example:

\begin{lstlisting}
mkYesod "Args" [parseRoutes|
/person/#Text PersonR GET
/year/#Integer/month/#Text/day/#Int DateR
/wiki/*Texts WikiR GET
|]

getPersonR :: Text -> Handler RepHtml
getPersonR name = defaultLayout [whamlet|<h1>Hello #{name}!|]

handleDateR :: Integer -> Text -> Int -> Handler RepPlain -- text/plain
handleDateR year month day =
    return $ RepPlain $ toContent $
        T.concat [month, " ", T.pack $ show day, ", ", T.pack $ show year]

getWikiR :: [Text] -> Handler RepPlain
getWikiR = return . RepPlain . toContent . T.unwords
instance Yesod Args
main = warpDebug 3000 Args
\end{lstlisting}

The arguments have the types of the dynamic pieces for each route, in
the order specified. Also, notice how we are able to use both RepHtml
and RepPlain.

\section{Монада Handler}

The vast majority of code you write in Yesod sits in the Handler
monad. If you are approaching this from an MVC (Model-View-Controller)
background, your Handler code is the Controller. Some important points
to know about Handler:

* It is an instance of MonadIO, so you can run any IO action in your
handlers with liftIO. By the way, liftIO is exported by the Yesod
module for your convenience.
* Like Widget, Handler is a fake-monad-transformer. It wraps around a
ResourceT IO monad. We discuss this type at length in the conduits
appendix, but for now, we'll just say it let's you safely allocate
resources.
* By "fake", I mean you can't use the standard lift function provided
by the transformers package, you must use the Yesod-supplied one (just
like with widgets).
* Handler is just a type synonym around GHandler. GHandler let's you
specify exactly which subsite and master site you're using. The
Handler synonym says that the sub and master sites are your
application's type.
* Handler provides a lot of different functionality, such as:
** Providing request information.
** Keeping a list of the extra response headers you've added.
** Allowing you to modify the user's session.
** Short-circuiting responses, for redirecting, sending static files,
** or reporting errors.

The remainder of this chapter will give a brief introduction to some
of the most common functions living in the Handler monad. I am
specifically not covering any of the session functions; that will be
addressed in the sessions chapter.

\subsection{Информация о приложении}

There are a number of functions that return information about your
application as a whole, and give no information about individual
requests. Some of these are:

getYesod

Returns your applicaton foundation value. If you store configuration
values in your foundation, you will probably end up using this
function a lot.

getYesodSub

Get the subsite foundation value. Unless you are working in a subsite,
this will return the same value as getYesod.

getUrlRender

Returns the URL rendering function, which converts a type-safe URL
into a Text. Most of the time- like with Hamlet- Yesod calls this
function for you, but you may occassionally need to call it directly.

getUrlRenderParams

A variant of getUrlRender that converts both a type-safe URL and a
list of query-string parameters. This function handles all
percent-encoding necessary.

\subsection{Информация о запросе}

The most common information you will want to get about the current
request is the requested path, the query string parameters and POSTed
form data. The first of those is dealt with in the routing, as
described above. The other two are best dealt with using the forms
module.

That said, you will sometimes need to get the data in a more raw
format. For this purpose, Yesod exposes the Request datatype along
with the getRequest function to retrieve it. This gives you access to
the full list of GET parameters, cookies, and preferred
languages. There are some convenient functions to make these lookups
easier, such as lookupGetParam, lookupCookie and languages. For raw
access to the POST parameters, you should use runRequest.

If you need even more raw data, like request headers, you can use
waiRequest to access the Web Application Interface (WAI) request
value. See the WAI appendix for more details.

\subsection{Неполная обработка запроса}
%FIXME: Вариант перевода: `Обработка запроса по короткой схеме`

The following functions immediately end execution of a handler
function and return a result to the user.

redirect

Sends a redirect response to the user (a 303 response). If you want to
use a different response code (e.g., a permanent 301 redirect), you
can use redirectWith.

Yesod uses a 303 response for HTTP/1.1 clients, and a 302 response for
HTTP/1.0 clients. You can read up on this sordid saga in the HTTP
spec.

notFound

Return a 404 response. This can be useful if a user requests a
database value that doesn't exist.

permissionDenied

Return a 403 response with a specific error message.

invalidArgs

A 400 response with a list of invalid arguments.

sendFile

Sends a file from the filesystem with a specified content type. This
is the preferred way to send static files, since the underlying WAI
handler may be able to optimize this to a sendfile system call. Using
readFile for sending static files should not be necessary.

sendResponse

Send a normal HasReps response with a 200 status code. This is really
just a convenience for when you need to break out of some deeply
nested code with an immediate response.

sendWaiResponse

When you need to get low-level and send out a raw WAI response. This
can be especially useful for creating streaming responses or a
technique like server-sent events.

\subsection{Заголовки ответа}

setCookie

Set a cookie on the client. Instead of taking an expiration date, this
function takes a cookie duration in minutes. Remember, you won't see
this cookie using lookupCookie until the following request.

deleteCookie

Tells the client to remove a cookie. Once again, lookupCookie will not
reflect this change until the next request.

setHeader

Set an arbitrary response header.

setLanguage

Set the preferred user language, which will show up in the result of
the languages function.

cacheSeconds

Set a Cache-Control header to indicate how many seconds this response
can be cached. This can be particularly useful if you are using
varnish on your server.

neverExpires

Set the Expires header to the year 2037. You can use this with content
which should never expire, such as when the request path has a hash
value associated with it.

alreadyExpired

Sets the Expires header to the past.

expiresAt

Sets the Expires header to the specified date/time.

\section{Выводы}

Routing and dispatch is arguably the core of Yesod: it is from here
that our type-safe URLs are defined, and the majority of our code is
written within the Handler monad. This chapter covered some of the
most important and central concepts of Yesod, so it is important that
you properly digest it.

This chapter also hinted at a number of more complex Yesod topics that
we will be covering later. But you should be able to write some very
sophisticated web applications with just the knowledge you have
learned up until here.'
