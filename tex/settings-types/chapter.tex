\chapter{Типы для настроек} \label{chap:settings_types}
Представим, что вы пишете веб-сервер. Вы хотите, чтобы вашему серверу
передавали порт, который он будет слушать, и приложение для запуска. И вы
создаёте следующую функцию:
\begin{lstlisting}
run :: Int -> Application -> IO ()
\end{lstlisting}

Но внезапно вы понимаете, что некоторые люди захотят настроить
продолжительность ожидания тайм-аута. И тогда вы меняете свой API:
\begin{lstlisting}
run :: Int -> Int -> Application -> IO ()
\end{lstlisting}

Какой \lstinline!Int! в итоге для времени ожидания, а какой для
номера порта? Да, вы могли бы создать какие-нибудь синонимы для типов
или прокомментировать ваш код. Но здесь есть ещё одна проблема,
пробирающаяся в ваш код: такая функция \lstinline!run! становится
неподдерживаемой. Вскоре потребуется получать дополнительный параметр
для обозначения, как следует обрабатывать исключения, а затем ещё один
для управления, к какому хосту привязываться, и так далее.

Более масштабируемое решение~--- ввести тип данных для
настроек:
\begin{lstlisting}
data Settings = Settings
    { settingsPort :: Int
    , settingsHost :: String
    , settingsTimeout :: Int
    }
\end{lstlisting}
И это делает вызывающий код почти самодокументированным:
\begin{lstlisting}
run Settings
    { settingsPort = 8080
    , settingsHost = "127.0.0.1"
    , settingsTimeout = 30
    } myApp
\end{lstlisting}

Великолепно, яснее быть не может, не так ли? Да, так, но что случится, когда у вас
будет 50 настроек для вашего веб-сервера? Вы действительно хотите
задавать их все каждый раз? Конечно, нет. Так что вместо этого
веб-сервер должен предоставлять набор значений по умолчанию:
\begin{lstlisting}
defaultSettings = Settings 3000 "127.0.0.1" 30
\end{lstlisting}
И теперь, вместо необходимости писать длинный кусок кода как выше,
мы можем отделаться лишь этим:
\begin{lstlisting}
run defaultSettings { settingsPort = 8080 } myApp -- (1)
\end{lstlisting}

Это замечательно, за исключением одного маленького
препятствия. Допустим, мы решаем добавить дополнительную запись в
\lstinline!Settings!. Любой код во внешнем мире, выглядящий как
\begin{lstlisting}
run (Settings 8080 "127.0.0.1" 30) myApp -- (2)
\end{lstlisting}
станет нерабочим, так как теперь конструктор~\lstinline!Settings!
принимает четыре аргумента. Правильной вещью стало бы увеличение номера
основной версии (major version), чтобы не нарушить работоспособность
зависимых пакетов. Но менять номер основной версии для каждого
минимального изменения настроек~--- это неудобство. Решение? Не
экспортировать конструктор~\lstinline!Settings!:
\begin{lstlisting}
module MyServer
    ( Settings
    , settingsPort
    , settingsHost
    , settingsTimeout
    , run
    , defaultSettings
    ) where
\end{lstlisting}

С таким подходом никто не сможет написать код наподобие (2), поэтому
вы можете свободно добавлять новые настройки, не боясь поломать код.

Единственный недостаток такого подхода~-- не сразу очевидно из
документации Haddock, что вы фактически можете менять настройки,
используя синтаксис записей. Основная идея этой главы: прояснить, что
происходит в библиотеках, использующих такую технику.

Я сам использую эту технику в нескольких местах, не стесняйтесь
заглянуть в Haddocks, чтобы понять, что я имею в виду.
\begin{itemize}
\item Warp: \footnotehref{http://hackage.haskell.org/packages/archive/warp/latest/doc/html/Network-Wai-Handler-Warp.html\#t:Settings}{\lstinline!Settings!}
\item http-conduit: \footnotehref{http://hackage.haskell.org/packages/archive/http-conduit/latest/doc/html/Network-HTTP-Conduit.html\#t:Request}{\lstinline!Request!}
  и \footnotehref{http://hackage.haskell.org/packages/archive/http-conduit/latest/doc/html/Network-HTTP-Conduit.html\#t:ManagerSettings}{\lstinline!ManagerSettings!}
\item xml-conduit:
  \begin{itemize}
  \item Разбор: \footnotehref{http://hackage.haskell.org/packages/archive/xml-conduit/latest/doc/html/Text-XML-Stream-Parse.html\#t:ParseSettings}{\lstinline!ParseSettings!}
  \item Представление: \footnotehref{http://hackage.haskell.org/packages/archive/xml-conduit/latest/doc/html/Text-XML-Stream-Render.html\#t:RenderSettings}{\lstinline!RenderSettings!}
  \end{itemize}
\end{itemize}

Кстати, \lstinline!http-conduit! и \lstinline!xml-conduit! фактически
создают экземпляры класса типов \footnotehref{http://hackage.haskell.org/packages/archive/data-default/latest/doc/html/Data-Default.html\#t:Default}{Default} из пакета Data.Default вместо объявления
своего нового идентификатора. Это означает, что вы можете просто
печатать \lstinline!def! вместо \lstinline!defaultParserSettings!.
