\chapter{Создание каркаса сайта}\label{chap:scaffolding}

Вы уже устали запускать маленькие примеры и готовы написать настоящий сайт?
Тогда вы читаете нужную главу. Даже при наличии библиотеки Yesod \marginpar{at your fingertips},
остаётся ещё много шагов, которые нужно пройти, чтобы получить production-quolity 
конфигурацию сайта:

\begin{itemize}
  \item разбор файлов конфигурации
  \item поддержка сигналов (*nix)
  \item более эффективная поддержка статических файлов
  \item хорошее расположение файлов \marginpar{layout}
\end{itemize}

Каркас сайта является комбинацией лучших техник собранных вместе в готовый к 
использованию сайт. И он (highly) очень рекомендуется для всех сайтов.
Эта глава описывает структуру каркаса, как использовать его, и некоторые 
неочевидные возможности.

Основная часть данной главы не содержит примеров кода. Рекомендуется 
просматривать собранный каркас. \marginpar{переписать предложение}

\begin{remark}
 Т.к. nature of каркаса сайта, это наиболее изменяемый компонент Yesod, и 
 он может изменяться от версии к версии. Поэтому информация в этой главе
 может быть немного устаревшей.
\end{remark}

\section{Как создавать каркас}

Пакет Yesod содержит  себе библиотеку и исполняемый файл (conveniently 
названый \lstinline!yesod!). Этот файл предоставляет небольшой набор 
команд (запустите \lstinline!yesod! чтобы увидеть весь список). Команда
\lstinline!yesod init! генерировать каркас сайта. Она запустит процесс
вопрос-ответ в котором вы должны будете указать минимальную информацию
(ваше имя, название проекта, и т.д.). После ответов на вопросы создастся 
шаблон проекта в подкаталоге с названием проекта.\marginpar{не точный перевод}

Наиболее важный вопрос о backend базы данных. У вас будет четыре варианта 
выбора: \texttt{SQLite}, \texttt{PostgreSQL}, \texttt{MongoDB}, и tiny. 
Tiny это не движок базы данных, это указание, что вы не хотите использовать
какую-либо базу данных. Этот выбор так же отключит некоторые дополнительные
зависимости и вы получите, leaner overall site. Далее в этой главе мы предполагаем,
что выбран один из движков баз данных. В случае с отсутсвующим движком 
существуют минимальные отличия.

После создания файлов, scrappfolder напишет команды для запуска сайта.
Существует два набора опций команд: одни для использования \texttt{cabal},
другие для использования \texttt{cabal-dev}. \texttt{cabal-dev} это обёртка
вокруг \texttt{cabal}, собирающая все зависимости в песочнице. Использование
его good way, чтобы убедиться, что установка других пакетов, не сломает сайт.
Его использование рекомендуется. Если у вас нет \texttt{cabal-dev}, то вы можете
установить его командой \lstinline!cabal install cabal-dev!.

Заметьте, что вам нужно использовать команду \lstinline!cabal install --only-dependencies! 
или \lstinline!cabal-dev install --only-dependencies!. Вероятно, что у вас не 
установлены все зависимости требуемые для сайта. Например, ни одного backend
базы данных, ни минификатора Javascript (\footnotehref{http://hackage.haskell.org/package/hjsmin}{hjsmin})
обычно не при установке пакета \texttt{yesod}.

И наконец, для того, чтобы запустить разработческий сайт, вам нужно использовать
\lstinline!yesod devel! (или \lstinline!yesod --dev devel!). Сайт автоматически
пересоберётся и будет перезагружаться при всех изменениях кода.

\section{Структура файлов}

Каркас сайта является полностью "кабализированным" приложением. В дополнение
к исходным файлам, конфигурационным файлам, шаблонам так же создаются статические файлы.

\subsection{Кабал файл}

Независимо от того используете ли вы cabal или yesod devel, сборка вашего кода всегда 
использует cabal файл. Если вы откроете файл, то увидите что там есть два блока для
библиотеки и для исполняемого файла. При построении используется только одна, в зависимости
от того, утановлен ли флаг library-only. Если флаг library-only включен, то собирается только
библиотека, это скособ которым yesod devel запускает приложение. Иначе собирается исполняемый файл

Флаг library-only должен использоваться только yesod devel; вы никогда не должны передавать его в 
cabal явно. Так же есть дополнительный флаг dev, который позволяет cabal-у собирать исполняемый файл,
но включает некоторые возможности library-only флага, т.е. отключение оптимизаций перезапуск
верский Shakespearean шаблонов

В общем случае вы будете использовать следующую сборку:
\begin{itemize}
    \item при разработке использовать только yesod devel
    \item при боевой сборке вызывайте \lstinline!cabal clean && cabal configure && cabal build!. И вы
        создадите оптимизованный исполняемый файл в директории \texttt{dist}. Раньше использовался флаг
        \lstinline!-fproduction!, поэтому если вы создали каркас сайта ранее вы должны использовать этот
        флаг для боевой сборки.
\end{itemize}

Вы также заметите, что мы указали все расширения языка в кабал файле. Расширения указаны дважды:
один раз для исполняемого файла, и другой для библиотеки. Если вы будете добавлять расширения, то
добавляйте их в оба блока.

Вы можете быть удивлены увидев расширение \texttt{NoImplicitPredude}. Мы его включили поскольку сайт 
включет свой специальный модуль \lstinline!Import!, с некоторыми изменениями в Prelude, это позволяет
сделать работу с Yesod более convenient.

И напоследок, следует отметить список экспортируемых модулей. Если вы добавите какой-либо модуль в 
ваше приложение, то вы должны обновить список для корректной работы yesod devel. К сожалению,
ни Cabal, ни GHC не выдадут предупреждение о том, что вы забыли сделать это обновление, и вместо этого
вы получите жутко выглядющую ошибку от yesod devel. Одно из планируемых улучшений это позволять yesod
devel проверять наличие неуказанных модулей.

\section{Маршруты и entities}

Множество раз на протяжении книги вы видели комментарии вида ,,Мы определяем пути/entities квазицитированием для простоты. 
На рабочем сайте, вам следует использовать внешний файл``. Каркас сайта, как раз и использует внешние файлы.
Маршруты определяются в файле \texttt{config/routes}, а entities в \texttt{config/models} соотвественно. Эти файлы
имеют в точности такой же синтаксис как и квазицитирование, которое вы видели на протяжении книги, и yesod devel 
автоматически пересобирает соотвествующие модули при изменении этих файлов.

The models files is referenced by \texttt{Model.hs}. В этом файле вы можете определять все, что угодно,
вот некоторые guidelines:
\begin{itemize}
    \item все типы данных использующиеся в entities должны быть импортированы, декларированы в \texttt{Model.hs}, 
        перед вызовом \lstinline!persistFile!.
    \item вспомогательные утилиты должны быть определены в \texttt{Import.hs} или, если они относятся к модели,
        то в фале в каталоге \texttt{Model} и импортированы в \texttt{Import.hs}.
\end{itemize}

\section{Основание и модули приложения}

Функция \lstinline!mkYesod! которую мы использовали на протяжении книги определяют несколько вещей:

\begin{itemize}
  \item тип маршрута
  \item функцию отображения
  \item функцию диспетчечризации
\end{itemize}

Функиця диспетчеризации refers ко всем остальным функциям обработки, поэтому все они должны быть 
или определены в том же файле, что и функция диспетчеризации или импортированы этой функцией (?). %TODO wtf?
Междутем, функция обработчих практически обязательно будет refer на маршрутный тип. Поэтому,
они должны быть или в одном файле, где определен тип маршрута или импротировать это файл.
Если you follow the logic here, то всё приложение должно обязательно находиться в одном файле!

Очевидно, что это не то, что мы хотим. Поэтому вместо использования \lstinline!mkYesod!, каркас 
сайта использует разделенные (decomposed) версии этой функции. Основание вызывает \lstinline!mkYesodData!,
который определяет маршрутный тип и функцию обработки. Так как он не определят функцию диспетчеризации,
функции обработчики не должны быть в той же области видимости. \texttt{Import.hs} импортирует
\texttt{Foundation.hs}, и все модули обработчики импортируют \texttt{Import.hs}.

В \texttt{Application.hs} мы вызываем \lstinline!mkYesodDispatch!, который создаёт функцию диспетчеризации.
Чтобы это заработало, все функции обрабочики должны быть в той же области видимости, так что убедитесь, в 
том, чтобы добавить импортирование всех модулей обработчиков, которые вы создаёте.

В остальном \texttt{Application.hs} достаточно прост. Он предоставляет две функции:
\lstinline!withDevelAppPort! используемую \lstinline!yesod devel! для запуска приложения
и \lstinline!getApplication! используемую исполняемым файлом для запуска.

\texttt{Foundation.hs} гораздо более интересен. Он:
\begin{itemize}
      \item определяет тип основание;
      \item определяет instance, такие как \lstinline!Yesod!, \lstinline!YesodAuth! и \lstinline!YesodPersist!;
      \item импортирует файлы сообщений
\end{itemize}

Если вы взгляните на строку запускающую \lstinline!mkMessage!, вы увидите, что она определяет директорию 
в которой находятся сообщения (\lstinline!messages!) и язык по умолчанию (\lstinline!en! для английского).
Это правильный файл для добавления дополнительных instances для приложени, таких как \lstinline!YesodAuthEmail! 
или \lstinline!YesodBreadcrumbs!. Мы будем ещё обратимся к этому файлу позже, когда будем ,
as we discussed some of the special implementations of Yesod typeclass methods.

\section{Import.hs}

Модуль \texttt{Import} появился из нескольких часто используемых шаблонов
Я хотел определить несколько вспомогательных функций (может быть оператор \lstinline!<> = mappend!) 
для того, чтобы использовать их во всех обработчиках. Я всегда добавляют один и те же мять
импортов  (\lstinline!Data.Text!, \lstinline!Control.Applicative!, прочие) в каждый модуль обработчик.
Мне бы хотелось гарантировать, что я никогда не использую некоторые (evil) функции (\lstinline!head!,
\lstinline!readFile!,\ldots) из \lstinline!Prelude!. Да evil это преувеличение. Если вам интересно,
почему я причислил эти функции к плохим: \lstinline!head! partial, и выбрасывает исключение на пустом
списке, а \lstinline!readFile! использует ленивый ввод-вывод, что приводит к тому, что хэндлер файла,
не закрывается достаточно быстро. Также, \lstinline!readFile! использует \lstinline!String! вместо
\lstinline!Text!. Решение заключается в том, чтобы включить расширение \texttt{NoImplicitPrelude} и
ре-экспортировать интересующие нас части \lstinline!Prelude!, и всё остальное, то нам нужно, так же
определить наши функции и инпортировать этот файл во все обработчики.

\section{Модули обработчиков}

Модули обработчиков должны находиться в директории \texttt{Handler}. Шаблон сайта включает один
модуль: \texttt{Handler/Home.hs}. Как разделять функции обрабочиков на модули зависит от вас, но
хорошее правило thumb:
\begin{itemize}
    \item методы относящиеся к одному маршруту должны находиться в одном файле, напр. \lstinline!getBlogR!
          и \lstinline!postBlogR!;
    \item связанные маршруты обычно находятся в одном файле, напр \lstinline!getPeopleR! и \lstinline!getPersonR!.
\end{itemize}

Естественно, решение зависит только от вас. Когда вы добавляете новый файл обработчик, убедитесь,
что вы сделали следующее:
\begin{itemize}
  \item добавили файл в систему контроля версий (вы ведь используете систему контроля версий?);
  \item добавили модуль в cabal файл;
  \item добавили его в файл \texttt{Application.hs};
  \item добавили \lstinline!module! statement вначале файла, и \lstinline!import Import! на следующей строке.
\end{itemize}

Одним из планируемых улучшений Yesod является автоматизация этих четырёх шагов.

\section{widgetFile}

It's very common to want to include CSS and Javascript specific to a page. You don't want to have to remember to include those Lucius and Julius files manually every time you refer to a Hamlet file. For this, the site template provides the widgetFile function.
If you have a handler function:
\begin{lstlisting}
getHomeR = defaultLayout $(widgetFile "homepage")
\end{lstlisting}
, Yesod will look for the following files:
\begin{lstlisting}
templates/homepage.hamlet
templates/homepage.lucius
templates/homepage.cassius
templates/homepage.julius
\end{lstlisting}

If any of those files are present, they will be automatically included in the output.
Due to the nature of how this works, if you launch your app with yesod devel, and then create a new file (e.g., templates/homepage.julius), the contents will not be included until the file calling widgetFile is recompiled. In such a case, you may need to force a save of that file to get yesod devel to recompile.
defaultLayout

One of the first things you're going to want to customize is the look of your site. The layout is actually broken up into two files:
templates/default-layout-wrapper.hamlet contains just the basic shell of a page. This file is interpreted as plain Hamlet, not as a Widget, and therefore cannot refer to other widgets, embed i18n strings, or add extra CSS/JS.
templates/default-layout.hamlet is where you would put the bulk of your page. You must remember to include the widget value in the page, as that contains the per-page contents. This file is interpreted as a Widget.
Also, since default-layout is included via the widgetFile function, any Lucius, Cassius, or Julius files named default-layout.* will automatically be included as well.
\section{Static files}

The scaffolded site automatically includes the static file subsite, optimized for serving files that will not change over the lifetime of the current build. What this means is that:
When your static file identifiers are generated (e.g., static/mylogo.png becomes mylogo\_png), a query-string parameter is added to it with a hash of the contents of the file. All of this happens at compile time.
When yesod-static serves your static files, it sets expiration headers far in the future, and incldues an etag based on a hash of your content.
Whenever you embed a link to mylogo\_png, the rendering includes the query-string parameter. If you change the logo, recompile, and launch your new app, the query string will have changed, causing users to ignore the cached copy and download a new version.
Additionally, you can set a specific static root in your Settings.hs file to serve from a different domain name. This has the advantage of not requiring transmission of cookies for static file requests, and also lets you offload static file hosting to a CDN or a service like Amazon S3. See the comments in the file for more details.
Another optimization is that CSS and Javascript included in your widgets will not be included inside your HTML. Instead, their contents will be written to an external file, and a link given. This file will be named based on a hash of the contents as well, meaning:
Caching works properly.
Yesod can avoid an expensive disk write of the CSS/Javascript file contents if a file with the same hash already exists.
Finally, all of your Javascript is automatically minified via hjsmin.
\section{Заключение}

The purpose of this chapter was not to explain every line that exists in the scaffolded site, but instead to give a general overview to how it works. The best way to become more familiar with it is to jump right in and start writing a Yesod site with it.


