\chapter{Persistent}\label{chap:persistent}

Формы представляют собой границу между пользователем и приложением. Другая граница, с которой нам приходится иметь дело, находится между приложением и хранилищем. Является ли это хранилище SQL базой данных, YAML- или бинарным файлом, вам с большой вероятностью придется попотеть, чтобы хранилище принимало типы данных вашего приложения. Persistent представляет собой ответ Yesod'а на проблему хранения данных. Это универсальный типобезопасный интерфейс к хранилищу данных для Haskell.

Haskell предлагает множество различных биндингов к базам данных. Однако большинство из них имеют малое представление о схеме базы данных и потому не обеспечивают полезных статических проверок. Кроме того, они вынуждают программиста использовать API и типы данных, зависящие от конкретной базы данных. Чтобы избавиться от этих проблем, программистами на Haskell была предпринята попытка пойти более революционным путем и создать хранилище данных, специфичное для Haskell, тем самым получив возможность с легкостью хранить любой тип данных Haskell. Эта возможность действительно прекрасна в некоторых случаях, но она делает программиста зависимым от техники хранения данных и используемой библиотеки, плохо взаимодейтсвует с другими языками, а также для обеспечения гибкости может требовать от программиста написания кучи кода, запрашивающего данные. В отличии от Persistent, который предоставляет выбор среди множества баз данных, каждая из которых оптимизирована для различных случаев, позволяет взаимодействовать с другими языками, а также использовать безопасный и производительный интерфейс запросов.

Persistent следует принципам безопасности типов и краткого, декларативного синтексиса. Среди других возможностей следует отметить:

\begin{itemize}
  \item Независимость от базы данных. Имеется первоклассная поддержка PostgreSQL, SQLite и MongoDB, а также эксперементальная поддержка CouchDB и находящаяся в разработке поддержка MySQL.
  \item Будучи нереляционным по своей природе, Persistent позволяет одновременно поддерживать множество слоев хранения данных и не обременен проблемами производительности, связанными с использованием JOIN'ов.
  \item Основным источником разочарования при использовании SQL баз данных является попытка изменения схемы базы данных. Persistent позволяет автоматически выполнять переход между базами данных. % схемами БД?
\end{itemize}

\section{Краткое содержание} % Synopsis

\begin{lstlisting}
{-# LANGUAGE QuasiQuotes, TemplateHaskell, TypeFamilies, OverloadedStrings #-}
{-# LANGUAGE GADTs, FlexibleContexts #-}

import Database.Persist
import Database.Persist.Sqlite
import Database.Persist.TH
import Control.Monad.IO.Class (liftIO)

share [mkPersist sqlSettings, mkMigrate "migrateAll"] [persist|
Person
    name String
    age Int Maybe
    deriving Show
BlogPost
    title String
    authorId PersonId
    deriving Show
|]

main :: IO ()
main = withSqliteConn ":memory:" $ runSqlConn $ do
    runMigration migrateAll

    johnId <- insert $ Person "John Doe" $ Just 35
    janeId <- insert $ Person "Jane Doe" Nothing

    insert $ BlogPost "My fr1st p0st" johnId
    insert $ BlogPost "One more for good measure" johnId

    oneJohnPost <- selectList [BlogPostAuthorId ==. johnId] [LimitTo 1]
    liftIO $ print (oneJohnPost :: [Entity BlogPost])

    john <- get johnId
    liftIO $ print (john :: Maybe Person)

    delete janeId
    deleteWhere [BlogPostAuthorId ==. johnId]
\end{lstlisting}%$

\section{Решение <<пограничной>> проблемы} % Solving the boundary issue

Допустим, вы храните информацию о людях в SQL базе данных. Соответствующая таблица может выглядеть как-то так:

\begin{lstlisting}
CREATE TABLE Person(id SERIAL PRIMARY KEY, name VARCHAR NOT NULL, age INTEGER)
\end{lstlisting}

И если вы используете такую СУБД, как PostgreSQL, вы можете быть уверены, что СУБД никогда не сохранит какой-нибудь дополнительный текст в поле age. (Нельзя сказать то же самое в отношении SQLite, однако пока что забудем об этом). Для отображения этой таблицы вы можете захотеть создать примерно такой тип данных:

\begin{lstlisting}
data Person = Person
    { personName :: Text
    , personAge :: Int
    }
\end{lstlisting}

Все выглядит вполне типобезопасно --- схема базы данных соответствует типу данных в Haskell, СУБД гарантирует, что некорректные данные никогда не будут сохранены в таблице, и все в целом выглядит прикрасно. Ну, до поры до времени.

Вы хотите получить данные из базы данных, а слой базы данных отдает вам данные в нетипизированном формате.

\begin{itemize}
  \item Вы хотите найти все людей, старше 32-х лет, но по ошибке пишете <<тридцать два>> в SQL-запросе. И значете что? Все прекрасно скомпилируется и вы не узнаете о проблеме до тех пор, пока не запустите программу.
  \item Вы решили найти первых десятерых человек в алфавитном порядке. Нет проблем... до тех пор, пока вы не сделаете опечатку в SQL-запросе. И снова, вы не узнаете об этом до тех пор, пока не запустите программу.
  \item В языках с динамической типизацией ответом на эти проблемы является модульное тестирование. Проверьте, что для всего, что может пойти не так, вы не забили написать тест. Но как, я полагаю, вы уже знаете, это не очень согласуется с подходом, принятом в Yesod. Мы предпочитаем использовать преимущества статической типизации языка Haskell для нашей собственной защиты, насколько это возможно, и хранение данных не является исключением.
\end{itemize}

Итак, вопрос остается открытым: как мы можем использовать систему типов языка Haskell, чтобы исправить положение?

\section{Типы} % Types

Как и в случае с маршрутами, нет ничего невероятно сложного в типобезопасном доступе к данным. Он всего лишь требует написания монотонного, подверженного ошибкам избыточного шаблонного кода. Как обычно, это означает, что мы можем использовать систему типов для того, чтобы избежать лишних ошибок. А чтобы не заниматься нудной работой, мы вооружимся Template Haskell.

\begin{remark}
В ранних версиях Persistent очень активно использовался Template Haskell. Начиная с версии 0.6 используется новая архитектура, позаимстованная из пакета groundhog. Благодаря новому подходу существенная часть нагрузки была переложена на фантомные типы.
\end{remark}

PersistValue является основным строительным блоком в Persistent. Этот тип представляет данные, посылаемые базе данных или получаемые от нее. Вот его определение:

\begin{lstlisting}
data PersistValue = PersistText Text
                  | PersistByteString ByteString
                  | PersistInt64 Int64
                  | PersistDouble Double
                  | PersistBool Bool
                  | PersistDay Day
                  | PersistTimeOfDay TimeOfDay
                  | PersistUTCTime UTCTime
                  | PersistNull
                  | PersistList [PersistValue]
                  | PersistMap [(T.Text, PersistValue)]
                  | PersistForeignKey ByteString -- ^ предназначен прежде всего для MongoDB
\end{lstlisting}

Каждый из бэкэндов Persistent должен знать, как переводить соответствующие значения во что-то понятное базе данных. Однако было бы неудобно выражать все данные через эти базовые типы. Следующим слоем является класс типов PersistField, определяющий, как произвольный тип может быть преобразован в тип PersistValue или обратно. PersistField соответствует столбцами в SQL базах данных. В приведенном ранее примере с людьми name и age будут нашими PersistField'ами.

Чтобы связать пользовательский код, нам понадобится последний класс типов --- PersistEntity. Экземпляр класса типов PersistEntity соответствует таблице в SQL базе данных. Этот класс типов определеяет несколько функций и связанные с ними типы. Таким образом, имеет место следующее соответствие между Persistent и SQL:

\begin{center}
\begin{tabular}{ | l | l |}
\hline
SQL	& Persistent \\ \hline
Тип (VARCHAR, INTEGER и тд)	& PersistValue \\ \hline
Столбец & PersistField \\ \hline
Таблица	& PersistEntity \\ \hline
\end{tabular}
\end{center}

\section{Генерация кода} % Code Generation

Дабы убедиться, что экземпляры класса PersistEntity корректно соответствуют вашим типам данных, Persistent берет на себя ответственность и за тех, и за других. Это хорошо и с точки зрения принципа DRY (Не повторяйтесь, Don't Repeat Yourslef): от вас требуется объявить сущности только один раз. Рассмотрим следующий пример:

\begin{lstlisting}
{-# LANGUAGE QuasiQuotes, TypeFamilies, GeneralizedNewtypeDeriving, TemplateHaskell, OverloadedStrings, GADTs #-}

import Database.Persist
import Database.Persist.TH
import Database.Persist.Sqlite
import Control.Monad.IO.Class (liftIO)

mkPersist sqlSettings [persist|
Person
    name String
    age Int
    deriving Show
|]
main = return ()
\end{lstlisting}

Здесь мы используем комбинацию из Template Haskell и квазицитирования (как при определении маршрутов): persist является обработчиком квазицитирования, который преобразует чувствительный к пробелам синтаксис в список определений сущностей. (Также вы можете вынести определение сущностей в отдельный файл и воспользоваться persistFile.) mkPersist принимает список этих сущностей и определяет:

\begin{itemize}
  \item По одному типу данных языка Haskell на сущность;
  \item Экземпляр класса PersistEntity для каждого определенного типа данных;
\end{itemize}

Приведенный выше пример генерирует код, который выглядит примерно следующим образом:

\begin{lstlisting}
{-# LANGUAGE TypeFamilies, GeneralizedNewtypeDeriving, OverloadedStrings, GADTs #-}

import Database.Persist
import Database.Persist.Store
import Database.Persist.Sqlite
import Database.Persist.EntityDef
import Control.Monad.IO.Class (liftIO)
import Control.Applicative

data Person = Person
    { personName :: String
    , personAge :: Int
    }
  deriving (Show, Read, Eq)

type PersonId = Key SqlPersist Person

instance PersistEntity Person where
    -- Обобщенный алгебраический тип данных.
    -- Это дает нам типобезопасный подход к сопоставлению
    -- полей с их типами данных
    data EntityField Person typ where
        PersonId   :: EntityField Person PersonId
        PersonName :: EntityField Person String
        PersonAge  :: EntityField Person Int

    type PersistEntityBackend Person = SqlPersist

    toPersistFields (Person name age) =
        [ SomePersistField name
        , SomePersistField age
        ]

    fromPersistValues [nameValue, ageValue] = Person
        <$> fromPersistValue nameValue
        <*> fromPersistValue ageValue
    fromPersistValues _ = Left "Invalid fromPersistValues input"

    -- Информация о каждом поле для внутреннего использования при генерации SQL выражений
    persistFieldDef PersonId = FieldDef
        (HaskellName "Id")
        (DBName "id")
        (FTTypeCon Nothing "PersonId")
        []
    persistFieldDef PersonName = FieldDef
        (HaskellName "name")
        (DBName "name")
        (FTTypeCon Nothing "String")
        []
    persistFieldDef PersonAge = FieldDef
        (HaskellName "age")
        (DBName "age")
        (FTTypeCon Nothing "Int")
        []
    data Unique Person typ = IgnoreThis
    entityDef = undefined
    halfDefined = undefined
    persistUniqueToFieldNames = undefined
    persistUniqueToValues = undefined
    persistUniqueKeys = undefined
    persistIdField = undefined
main :: IO ()
main = return ()
\end{lstlisting}%$

Как и следовало ожидать, определение типа данных Person очень близко к определению, данному в оригинальной версии кода, где использовался Template Haskell. Мы также имеем обощенный алгебраический тип данный (ОАТД), предоставляющий отдельный конструктор для каждого поля. Этот ОАТД кодирует как тип сущности, так и тип поля. Мы используем его конструкторы через модуль Persistent, например, чтобы убедиться, что когда мы применяем фильтр, типы фильтруемого значения и поля совпадают.

Мы можем использовать сгенерированный тип Person как и любой другой тип языка Haskell, а затем передать его в одну из функций модуля Persistent.

\begin{lstlisting}
main = withSqliteConn ":memory:" $ runSqlConn $ do
    michaelId <- insert $ Person "Michael" 26
    michael <- get michaelId
    liftIO $ print michael
\end{lstlisting}%$

Мы начнем со стандартного кода, работающего с базой данных. В данном случае мы использовали функции для работы с одним соединением. Модуль Persistent также предоставляет функции для работы с пулом соединений, которые использовать в боевом окружении обычно предпочтительнее.

В приведенном примере мы видим две функции. Функция \lstinline'insert' создает новую запись в базе данных и возвращает ее ID. Как и все остальное в модуле Persistent, ID являются типобезопасными. Более подробно о том, как работают эти ID, мы узнаем позже. Итак, код \lstinline'insert \$ Person "Michael" 25', возвращает значение типа PersonId.

Следующая функция, которую мы видим --- это \lstinline'get'. Она пытается загрузить из базы данных значение, используя заданный ID. При использовании Persistent вам никогда не придется беспокоиться, что вы, возможно, используете ключ не от той таблицы. Код, который пытается получить другую сущность (например, House), используя PersonId, никогда не будет скомпилирован.

\section{PersistStore}

One last detail is left unexplained from the previous example: what are those withSqliteConn and runSqlConn functions doing, and what is that monad that our database actions are running in?

All database actions need to occur within an instance of PersistStore. As its name implies, every data store (PostgreSQL, SQLite, MongoDB) has an instance of PersistStore. This is where all the translations from PersistValue to database-specific values occur, where SQL query generation happens, and so on.

As you can imagine, even though PersistStore provides a safe, well-typed interface to the outside world, there are a lot of database interactions that could go wrong. However, by testing this code automatically and thoroughly in a single location, we can centralize our error-prone code and make sure it is as bug-free as possible.

withSqliteConn creates a single connection to a database using its supplied connection string. For our test cases, we will use :memory:, which uses an in-memory database. runSqlConn uses that connection to run the inner action. Both SQLite and PostgreSQL share the same instance of PersistStore: SqlPersist.

There are actually a few other typeclasses: PersistUpdate and PersistQuery. Different typeclasses provide different functionality, which allows us to write backends that use simpler data stores (e.g., Redis) even though they can't provide us all the high-level functionality available in Persistent.

One important thing to note is that everything which occurs inside a single call to runSqlConn runs in a single transaction. This has two important implications:

* For many databases, committing a transaction can be a costly activity. By putting multiple steps into a single transaction, you can speed up code dramatically.
* If an exception is thrown anywhere inside a single call to runSqlConn, all actions will be rolled back (assuming your backend has rollback support).

\section{Migrations}

I'm sorry to tell you, but so far I have lied to you a bit: the example from the previous section does not actually work. If you try to run it, you will get an error message about a missing table.

For SQL databases, one of the major pains can be managing schema changes. Instead of leaving this to the user, Persistent steps in to help, but you have to ask it to help. Let's see what this looks like:

\begin{lstlisting}
{-# LANGUAGE QuasiQuotes, TypeFamilies, GeneralizedNewtypeDeriving, TemplateHaskell,
             OverloadedStrings, GADTs, FlexibleContexts #-}

import Database.Persist
import Database.Persist.TH
import Database.Persist.Sqlite
import Control.Monad.IO.Class (liftIO)

share [mkPersist sqlSettings, mkSave "entityDefs"] [persist|
Person
    name String
    age Int
    deriving Show
|]

main = withSqliteConn ":memory:" $ runSqlConn $ do
    runMigration $ migrate entityDefs (undefined :: Person) -- this line added: that's it!
    michaelId <- insert $ Person "Michael" 26
    michael <- get michaelId
    liftIO $ print michael
\end{lstlisting}%$

With this one little code change, Persistent will automatically create your Person table for you. This split between runMigration and migrate allows you to migrate multiple tables simultaneously.

This works when dealing with just a few entities, but can quickly get tiresome once we are dealing with a dozen entities. Instead of repeating yourself, Persistent provides a helper function, mkMigrate:

\begin{lstlisting}
{-# LANGUAGE QuasiQuotes, TypeFamilies, GeneralizedNewtypeDeriving, TemplateHaskell,
             OverloadedStrings, GADTs, FlexibleContexts #-}

import Database.Persist
import Database.Persist.Sqlite
import Database.Persist.TH

share [mkPersist sqlSettings, mkMigrate "migrateAll"] [persist|
Person
    name String
    age Int
    deriving Show
Car
    color String
    make String
    model String
    deriving Show
|]

main = withSqliteConn ":memory:" $ runSqlConn $ do
    runMigration migrateAll
\end{lstlisting}%$

mkMigrate is a Template Haskell function which creates a new function that will automatically call migrate on all entities defined in the persist block. The share function is just a little helper that passes the information from the persist block to each Template Haskell function and concatenates the results.

Persistent has very conservative rules about what it will do during a migration. It starts by loading up table information from the database, complete with all defined SQL datatypes. It then compares that against the entity definition given in the code. For the following cases, it will automatically alter the schema:

\begin{itemize}
	\item The datatype of a field changed. However, the database may object to this modification if the data cannot be translated.
	\item A field was added. However, if the field is not null, no default value is supplied (we'll discuss defaults later) and there is already data in the database, the database will not allow this to happen.
	\item A field is converted from not null to null. In the opposite case, Persistent will attempt the conversion, contingent upon the database's approval.
	\item A brand new entity is added.
\end{itemize}

However, there are some cases that Persistent will not handle:

\begin{itemize}
	\item Field or entity renames: Persistent has no way of knowing that "name" has now been renamed to "fullName": all it sees is an old field called name and a new field called fullName.
	\item Field removals: since this can result in data loss, Persistent by default will refuse to perform the action (you can force the issue by using runMigrationUnsafe instead of runMigration, though it is not recommended).
\end{itemize}

runMigration will print out the migrations it is running on stderr (you can bypass this by using runMigrationSilent). Whenever possible, it uses ALTER TABLE calls. However, in SQLite, ALTER TABLE has very limited abilities, and therefore Persistent must resort to copying the data from one table to another.

Finally, if instead of performing a migration, you want Persistent to give you hints about what migrations are necessary, use the printMigration function. This function will print out the migrations which runMigration would perform for you. This may be useful for performing migrations that Persistent is not capable of, for adding arbitrary SQL to a migration, or just to log what migrations occurred.

\section{Уникальность} % Uniqueness

In addition to declaring fields within an entity, you can also declare uniqueness constraints. A typical example would be requiring that a username be unique.

% TODO: WTF?

Unique Username

User
    username Text
    UniqueUsername username

While each field name must begin with a lowercase letter, the uniqueness constraints must begin with an uppercase letter.

\begin{lstlisting}
{-# LANGUAGE QuasiQuotes, TypeFamilies, GeneralizedNewtypeDeriving, TemplateHaskell,
             OverloadedStrings, GADTs, FlexibleContexts #-}
import Database.Persist
import Database.Persist.Sqlite
import Database.Persist.TH
import Data.Time
import Control.Monad.IO.Class (liftIO)

share [mkPersist sqlSettings, mkMigrate "migrateAll"] [persist|
Person
    firstName String
    lastName String
    age Int
    PersonName firstName lastName
    deriving Show
|]

main = withSqliteConn ":memory:" $ runSqlConn $ do
    runMigration migrateAll
    insert $ Person "Michael" "Snoyman" 26
    michael <- getBy $ PersonName "Michael" "Snoyman"
    liftIO $ print michael
\end{lstlisting}%$

To declare a unique combination of fields, we add an extra line to our declaration. Persistent knows that it is defining a unique constructor, since the line begins with a capital letter. Each following word must be a field in this entity.

The main restriction on uniqueness is that it can only be applied non-null fields. The reason for this is that the SQL standard is ambiguous on how uniqueness should be applied to NULL (e.g., is NULL=NULL true or false?). Besides that ambiguity, most SQL engines in fact implement rules which would be contrary to what the Haskell datatypes anticipate (e.g., PostgreSQL says that NULL=NULL is false, whereas Haskell says Nothing == Nothing is True).

In addition to providing nice guarantees at the database level about consistency of your data, uniqueness constraints can also be used to perform some specific queries within your Haskell code, like the getBy demonstrated above. This happens via the Unique associated type. In the example above, we end up with a new constructor:

\begin{lstlisting}
PersonName :: String -> String -> Unique Person
\end{lstlisting}

\section{Запросы} % Queries

Depending on what your goal is, there are different approaches to querying the database. Some commands query based on a numeric ID, while others will filter. Queries also differ in the number of results they return: some lookups should return no more than one result (if the lookup key is unique) while others can return many results.

Persistent therefore provides a few different query functions. As usual, we try to encode as many invariants in the types as possible. For example, a query that can return only 0 or 1 results will use a Maybe wrapper, whereas a query returning many results will return a list.

\section{Выборка по ID} % Fetching by ID

The simplest query you can perform in Persistent is getting based on an ID. Since this value may or may not exist, its return type is wrapped in a Maybe.

Using get % TODO: title? image?

\begin{lstlisting}
    personId <- insert $ Person "Michael" "Snoyman" 26
    maybePerson <- get personId
    case maybePerson of
        Nothing -> liftIO $ putStrLn "Just kidding, not really there"
        Just person -> liftIO $ print person
\end{lstlisting}%$

This can be very useful for sites that provide URLs like /person/5. However, in such a case, we don't usually care about the Maybe wrapper, and just want the value, returning a 404 message if it is not found. Fortunately, the get404 function helps us out here. We'll go into more details when we see integration with Yesod.

\section{Выборка по уникальному ключу} % Fetching by unique constraint

getBy is almost identical to get, except it takes a uniqueness constraint instead of an ID it takes a Unique value.

Using getBy % TODO: title? image?

\begin{lstlisting}
    personId <- insert $ Person "Michael" "Snoyman" 26
    maybePerson <- getBy $ UniqueName "Michael" "Snoyman"
    case maybePerson of
        Nothing -> liftIO $ putStrLn "Just kidding, not really there"
        Just person -> liftIO $ print person
\end{lstlisting}

Like get404, there is also a getBy404 function.

\section{Select functions}

Most likely, you're going to want more powerful queries. You'll want to find everyone over a certain age; all cars available in blue; all users without a registered email address. For this, you need one of the select functions.

All the select functions use a similar interface, with slightly different outputs:

% TODO: table!

Function	Returns

selectSource	A Source containing all the IDs and values from the database. This allows you to write streaming code.
                We cover Sources in detail in the conduits appendix. Additionally, there's another function called selectSourceConn which allows you more control of connection allocation. We use this in the Sphinx case study.

selectList	A list containing all the IDs and values from the database. All records will be loaded into memory.

selectFirst	Takes just the first ID and value from the database, if available

selectKeys	Returns only the keys, without the values, as a Source.

selectList is the most commonly used, so we will cover it specifically. Understanding the others should be trivial after that.

selectList takes two arguments: a list of Filters, and a list of SelectOpts. The former is what limits your results based on characteristics; it allows for equals, less than, is member of, and such. SelectOpts provides for three different features: sorting, limiting output to a certain number of rows, and offsetting results by a certain number of rows.

The combination of limits and offsets is very important; it allows for efficient pagination in your webapps.

Let's jump straight into an example of filtering, and then analyze it.

\begin{lstlisting}
    people <- selectList [PersonAge >. 25, PersonAge <=. 30] []
    liftIO $ print people
\end{lstlisting}%$

As simple as that example is, we really need to cover three points:

% TODO: list!

1. PersonAge is a constructor for an associated phantom type. That might sound scary, but what's important is that it uniquely identifies the "age" column of the "person" table, and that it knows that the age field is an Int. (That's the phantom part.)
2. We have a bunch of Persistent filtering operators. They're all pretty straight-forward: just tack a period to the end of what you'd expect. There are three gotchas here, I'll explain below.
3. The list of filters is ANDed together, so that our constraint means "age is greater than 25 AND age is less than or equal to 30". We'll describe ORing later.

The one operator that's surprisingly named is "not equals." We use !=., since /=. is used for updates (for "divide-and-set", described later). Don't worry: if you use the wrong one, the compiler will catch you. The other two surprising operators are the "is member" and "is not member". They are, respectively, <-. and /<-. (both end with a period).

And regarding ORs, we use the ||. operator. For example:

\begin{lstlisting}
    people <- selectList
        (       [PersonAge >. 25, PersonAge <=. 30]
            ||. [PersonFirstName /<-. ["Adam", "Bonny"]]
            ||. ([PersonAge ==. 50] ||. [PersonAge ==. 60])
        )
        []
    liftIO $ print people
\end{lstlisting}%$

This (completely nonsensical) example means: find people who are 26-30, inclusive, OR whose names are neither Adam or Bonny, OR whose age is either 50 or 60.

\section{SelectOpt}

All of our selectList calls have included an empty list as the second parameter. That specifies no options, meaning: sort however the database wants, return all results, and don't skip any results. A SelectOpt has four constructors that can be used to change all that.

Asc

Sort by the given column in ascending order. This uses the same phantom type as filtering, such as PersonAge.

Desc

Same as Asc, in descending order.

LimitTo

Takes an Int argument. Only return up to the specified number of results.

OffsetBy

Takes an Int argument. Skip the specified number of results.

The following code defines a function that will break down results into pages. It returns all people aged 18 and over, and then sorts them by age (oldest person first). For people with the same age, they are sorted alphabetically by last name, then first name.

\begin{lstlisting}
resultsForPage pageNumber = do
    let resultsPerPage = 10
    selectList
        [ PersonAge >=. 18
        ]
        [ Desc PersonAge
        , Asc PersonLastName
        , Asc PersonFirstName
        , LimitTo resultsPerPage
        , OffsetBy $ (pageNumber - 1) * resultsPerPage
        ]

main = withSqliteConn ":memory:" $ runSqlConn $ do
    runMigrationSilent migrateAll
    personId <- insert $ Person "Michael" "Snoyman" 26
    resultsForPage 1 >>= liftIO . print
\end{lstlisting}%$

Manipulation % TODO: title?

Querying is only half the battle. We also need to be able to add data to and modify existing data in the database.

\section{Вставка} % Insert 

It's all well and good to be able to play with data in the database, but how does it get there in the first place? The answer is the insert function. You just give it a value, and it gives back an ID.

At this point, it makes sense to explain a bit of the philosophy behind Persistent. In many other ORM solutions, the datatypes used to hold data are opaque: you need to go through their defined interfaces to get at and modify the data. That's not the case with Persistent: we're using plain old Algebraic Data Types for the whole thing. This means you still get all the great benefits of pattern matching, currying and everything else you're used to.

However, there are a few things we can't do. For one, there's no way to automatically update values in the database every time the record is updated in Haskell. Of course, with Haskell's normal stance of purity and immutability, this wouldn't make much sense anyway, so I don't shed any tears over it.

However, there is one issue that newcomers are often bothered by: why are IDs and values completely separate? It seems like it would be very logical to embed the ID inside the value. In other words, instead of having:

\begin{lstlisting}
data Person = Person { name :: String }
\end{lstlisting}

have

\begin{lstlisting}
data Person = Person { personId :: PersonId, name :: String }
\end{lstlisting}

Well, there's one problem with this right off the bat: how do we do an insert? If a Person needs to have an ID, and we get the ID by inserting, and an insert needs a Person, we have an impossible loop. We could solve this with undefined, but that's just asking for trouble.

OK, you say, let's try something a bit safer:

\begin{lstlisting}
data Person = Person { personId :: Maybe PersonId, name :: String }
\end{lstlisting}

I definitely prefer insert \$ Person Nothing "Michael" to insert \$ Person undefined "Michael". And now our types will be much simpler, right? For example, selectList could return a simple [Person] instead of that ugly [Entity SqlPersist Person].

Entity is a datatype that ties together both the ID and value of an entity. Since IDs can be different based on backend, we also need to provide the Persistent backend we're using. The datatype Entity SqlPersist Person can be read as "the ID and value of a person stored in a SQL database."

The problem is that the "ugliness" is incredibly useful. Having Entity SqlPersist Person makes it obvious, at the type level, that we're dealing with a value that exists in the database. Let's say we want to create a link to another page that requires the PersonId (not an uncommon occurrence as we'll discuss later). The Entity SqlPersist Person form gives us unambiguous access to that information; embedding PersonId within Person with a Maybe wrapper means an extra runtime check for Just, instead of a more error-proof compile time check.

Finally, there's a semantic mismatch with embedding the ID within the value. The Person is the value. Two people are identical (in the context of a database) if all their fields are the same. By embedding the ID in the value, we're no longer talking about a person, but about a row in the database. Equality is no longer really equality, it's identity: is this the same person, as opposed to an equivalent person.

In other words, there are some annoyances with having the ID separated out, but overall, it's the right approach, which in the grand scheme of things leads to better, less buggy code.

\section{Обновление} % Update

Now, in the context of that discussion, let's think about updating. The simplest way to update is:

\begin{lstlisting}
let michael = Person "Michael" 26
    michaelAfterBirthday = michael { personAge = 27 }
\end{lstlisting}

But that's not actually updating anything, it's just creating a new Person value based on the old one. When we say update, we're not talking about modifications to the values in Haskell. (We better not be of course, since Haskell data types are immutable.)

Instead, we're looking at ways of modifying rows in a table. And the simplest way to do that is with the update function.

\begin{lstlisting}
    personId <- insert $ Person "Michael" "Snoyman" 26
    update personId [PersonAge =. 27]
    resultsForPage 1 >>= liftIO . print
\end{lstlisting}%$

update takes two arguments: an ID, and a list of Updates. The simplest update is assignment, but it's not always the best. What if you want to increase someone's age by 1, but you don't have their current age? Persistent has you covered:

\begin{lstlisting}
haveBirthday personId = update personId [PersonAge +=. 1]

main = withSqliteConn ":memory:" $ runSqlConn $ do
    runMigrationSilent migrateAll
    personId <- insert $ Person "Michael" "Snoyman" 26
    update personId [PersonAge =. 27]
    haveBirthday personId
    resultsForPage 1 >>= liftIO . print
\end{lstlisting}%$

And as you might expect, we have all the basic mathematical operators: +=., -=., *=., and /=. (full stop). These can be convenient for updating a single record, but they are also essential for proper ACID guarantees. Imagine the alternative: pull out a Person, increment the age, and update the new value. If you have two threads/processes working on this database at the same time, you're in for a world of hurt (hint: race conditions).

Sometimes you'll want to update many fields at once (give all your employees a 5\% pay increase, for example). updateWhere takes two parameters: a list of filters, and a list of updates to apply.

\begin{lstlisting}
    updateWhere [PersonFirstName ==. "Michael"] [PersonAge *=. 2] -- it's been a long day
\end{lstlisting}

Occassionally, you'll just want to completely replace the value in a database with a different value. For that, you use (surprise) the replace function.

\begin{lstlisting}
    personId <- insert $ Person "Michael" "Snoyman" 26
    replace personId $ Person "John" "Doe" 20
    update personId [PersonAge =. 27]
    haveBirthday personId
    updateWhere [PersonFirstName ==. "Michael"] [PersonAge *=. 2] -- it's been a long day
    resultsForPage 1 >>= liftIO . print
\end{lstlisting}%$

\section{Удаление} % Delete

As much as it pains us, sometimes we must part with our data. To do so, we have three functions:

delete

Delete based on an ID

deleteBy

Delete based on a unique constraint

deleteWhere

Delete based on a set of filters

\begin{lstlisting}
    personId <- insert $ Person "Michael" "Snoyman" 26
    delete personId
    deleteBy $ UniqueName "Michael" "Snoyman"
    deleteWhere [PersonFirstName ==. "Michael"]
\end{lstlisting}%$

We can even use deleteWhere to wipe out all the records in a table, we just need to give some hints to GHC as to what table we're interested in:

\begin{lstlisting}
    deleteWhere ([] :: [Filter Person])
\end{lstlisting}

\section{Атрибуты} % Attributes

So far, we have seen a basic syntax for our persist blocks: a line for the name of our entities, and then an indented line for each field with two words: the name of the field and the datatype of the field. Persistent handles more than this: you can assign an arbitrary list of attributes after the first two words on a line.

Suppose we want to have a Person entity with an (optional) age and the timestamp of when he/she was added to the system. For entities already in the database, we want to just use the current date-time for that timestamp.

\begin{lstlisting}
{-# LANGUAGE QuasiQuotes, TypeFamilies, GeneralizedNewtypeDeriving, TemplateHaskell,
             OverloadedStrings, GADTs, FlexibleContexts #-}

import Database.Persist
import Database.Persist.Sqlite
import Database.Persist.TH
import Data.Time
import Control.Monad.IO.Class

share [mkPersist sqlSettings, mkMigrate "migrateAll"] [persist|
Person
    name String
    age Int Maybe
    created UTCTime default=now()
    deriving Show
|]

main = withSqliteConn ":memory:" $ runSqlConn $ do
    time <- liftIO getCurrentTime
    runMigration migrateAll
    insert $ Person "Michael" (Just 26) time
    insert $ Person "Greg" Nothing time
\end{lstlisting}%$

Maybe is a built in, single word attribute. It makes the field optional. In Haskell, this means it is wrapped in a Maybe. In SQL, it makes the column nullable.

The default attribute is backend specific, and uses whatever syntax is understood by the database. In this case, it uses the database's built-in now() function. Suppose that we now want to add a field for a person's favorite programming language:

\begin{lstlisting}
{-# LANGUAGE QuasiQuotes, TypeFamilies, GeneralizedNewtypeDeriving, TemplateHaskell,
             OverloadedStrings, GADTs, FlexibleContexts #-}

import Database.Persist
import Database.Persist.Sqlite
import Database.Persist.TH
import Data.Time

share [mkPersist sqlSettings, mkMigrate "migrateAll"] [persist|
Person
    name String
    age Int Maybe
    created UTCTime default=now()
    language String default='Haskell'
    deriving Show
|]

main = withSqliteConn ":memory:" $ runSqlConn $ do
    runMigration migrateAll
\end{lstlisting}%$

The default attribute has absolutely no impact on the Haskell code itself; you still need to fill in all values. This will only affect the database schema and automatic migrations.

We need to surround the string with single quotes so that the database can properly interpret it. Finally, Persistent can use double quotes for containing white space, so if we want to set someone's default home country to be El Salvador:

\begin{lstlisting}
{-# LANGUAGE QuasiQuotes, TypeFamilies, GeneralizedNewtypeDeriving, TemplateHaskell,
             OverloadedStrings, GADTs, FlexibleContexts #-}

import Database.Persist
import Database.Persist.Sqlite
import Database.Persist.TH
import Data.Time

share [mkPersist sqlSettings, mkMigrate "migrateAll"] [persist|
Person
    name String
    age Int Maybe
    created UTCTime default=now()
    language String default='Haskell'
    country String "default='El Salvador'"
    deriving Show
|]

main = withSqliteConn ":memory:" $ runSqlConn $ do
    runMigration migrateAll
\end{lstlisting}%$
 
One last trick you can do with attributes is to specify the names to be used for the SQL tables and columns. This can be convenient when interacting with existing databases.

\begin{lstlisting}
share [mkPersist sqlSettings, mkMigrate "migrateAll"] [persist|
Person sql=the-person-table
    firstName String sql=first_name
    lastName String sql=fldLastName
    age Int Gt Desc "sql=The Age of the Person"
    UniqueName firstName lastName
    deriving Show
|]

resultsForPage pageNumber = do
    let resultsPerPage = 10
    selectList
        [ PersonAge >=. 18
        ]
        [ Desc PersonAge
        , Asc PersonLastName
        , Asc PersonFirstName
        , LimitTo resultsPerPage
        , OffsetBy $ (pageNumber - 1) * resultsPerPage
        ]

main = withSqliteConn ":memory:" $ runSqlConn $ do
    runMigration migrateAll
    personId <- insert $ Person "Michael" "Snoyman" 26
    resultsForPage 1 >>= liftIO . print
\end{lstlisting}%$

\section{Отношения} % Relations

Persistent allows references between your data types in a manner that is consistent with supporting non-SQL databases. We do this by embedding an ID in the related entity. So if a person has many cars:

\begin{lstlisting}
{-# LANGUAGE QuasiQuotes, TypeFamilies, GeneralizedNewtypeDeriving, TemplateHaskell,
             OverloadedStrings, GADTs, FlexibleContexts #-}

import Database.Persist
import Database.Persist.Sqlite
import Database.Persist.TH
import Control.Monad.IO.Class (liftIO)
import Data.Time

share [mkPersist sqlSettings, mkMigrate "migrateAll"] [persist|
Person
    name String
    deriving Show
Car
    ownerId PersonId Eq
    name String
    deriving Show
|]

main = withSqliteConn ":memory:" $ runSqlConn $ do
    runMigration migrateAll
    bruce <- insert $ Person "Bruce Wayne"
    insert $ Car bruce "Bat Mobile"
    insert $ Car bruce "Porsche"
    -- this could go on a while
    cars <- selectList [CarOwnerId ==. bruce] []
    liftIO $ print cars
\end{lstlisting}%$
 
Using this technique, you can define one-to-many relationships. To define many-to-many relationships, we need a join entity, which has a one-to-many relationship with each of the original tables. It is also a good idea to use uniqueness constraints on these. For example, to model a situation where we want to track which people have shopped in which stores:

\begin{lstlisting}
{-# LANGUAGE QuasiQuotes, TypeFamilies, GeneralizedNewtypeDeriving, TemplateHaskell,
             OverloadedStrings, GADTs, FlexibleContexts #-}

import Database.Persist
import Database.Persist.Sqlite
import Database.Persist.TH
import Data.Time

share [mkPersist sqlSettings, mkMigrate "migrateAll"] [persist|
Person
    name String
Store
    name String
PersonStore
    personId PersonId
    storeId StoreId
    UniquePersonStore personId storeId
|]

main = withSqliteConn ":memory:" $ runSqlConn $ do
    runMigration migrateAll

    bruce <- insert $ Person "Bruce Wayne"
    michael <- insert $ Person "Michael"

    target <- insert $ Store "Target"
    gucci <- insert $ Store "Gucci"
    sevenEleven <- insert $ Store "7-11"

    insert $ PersonStore bruce gucci
    insert $ PersonStore bruce sevenEleven

    insert $ PersonStore michael target
    insert $ PersonStore michael sevenEleven
\end{lstlisting}%$

\section{Closer look at types}

So far, we've spoken about Person and PersonId without really explaining what they are. In the simplest sense, for a SQL-only system, the PersonId could just be type PersonId = Int64. However, that means there is nothing binding a PersonId at the type level to the Person entity. As a result, you could accidently use a PersonId and get a Car. In order to model this relationship, we use phantom types. So, our next naive step would be:

\begin{lstlisting}
newtype Key entity = Key Int64
type PersonId = Key Person
\end{lstlisting}

Prior to Persistent 0.6, we used associated types instead of phantom types. You could solve the problem in that direction as well, but phantoms worked out better.

And that works out really well, until you get to a backend that doesn't use Int64 for its IDs. And that's not just a theoretical question; MongoDB uses ByteStrings instead. So what we need is a key value that can contain an Int and a ByteString. Seems like a great time for a sum type:

\begin{lstlisting}
data Key entity = KeyInt Int64 | KeyByteString ByteString
\end{lstlisting}

But that's just asking for trouble. Next we'll have a backend that uses timestamps, so we'll need to add another constructor to Key. This could go on for a while. Fortunately, we already have a sum type intended for representing arbitrary data: PersistValue:

\begin{lstlisting}
newtype Key entity = Key PersistValue
\end{lstlisting}

But this has another problem. Let's say we have a web application that takes an ID as a parameter from the user. It will need to receive that parameter as Text and then try to convert it to a Key. Well, that's simple: write a function to convert a Text to a PersistValue, and then wrap the result in the Key constructor, right?

Wrong. We tried this, and there's a big problem. We end up getting Keys that could never be. For example, if we're dealing with SQL, a key must be an integer. But the approach described above would allow arbitrary textual data in. The result was a bunch of 500 server errors as the database choked on comparing an integer column to text.

So what we need is a way to convert text to a Key, but have it dependent on the rules of the backend in question. And once phrased that way, the answer is simple: just add another phantom. The real, actual definition of Key in Persistent is:

\begin{lstlisting}
newtype Key backend entity = Key { unKey :: PersistValue }
\end{lstlisting}

This works great: we can have a Text -> Key MongoDB entity function and a Text -> Key SqlPersist entity function, and everything runs smoothly. But now we have a new problem: relations. Let's say we want to represent blogs and blog posts. We would use the entity definition:

Blog
    title Text
Post
    title Text
    blogId BlogId

But what would that look like in terms of our Key datatype?

\begin{lstlisting}
data Blog = Blog { blogTitle :: Text }
data Post = Post { postTitle :: Text, postBlogId :: Key <what goes here?> Blog }
\end{lstlisting}

We need something to fill in as the backend. In theory, we could hardcode this to SqlPersist, or Mongo, but then our datatypes will only work for a single backend. For an individual application, that might be acceptable, but what about libraries defining datatypes to be used by multiple applications, using multiple backends?

So things got a little more complicated. Our types are actually:

\begin{lstlisting}
data BlogGeneric backend = Blog { blogTitle :: Text }
data PostGeneric backend = Post { postTitle :: Text, postBlogId :: Key backend (BlogGeneric backend) }
\end{lstlisting}

Notice that we still keep the short names for the constructors and the records. Finally, to give a simple interface for normal code, we define some type synonyms:

\begin{lstlisting}
type Blog = BlogGeneric SqlPersist
type BlogId = Key SqlPersist Blog
type Post = PostGeneric SqlPersist
type PostId = Key SqlPersist Post
\end{lstlisting}

And no, SqlPersist isn't hard-coded into Persistent anywhere. That sqlSettings parameter you've been passing to mkPersist is what tells us to use SqlPersist. Mongo code will use mongoSettings instead.

This might be quite complicated under the surface, but user code hardly ever touches this. Look back through this whole chapter: not once did we need to deal with the Key or Generic stuff directly. The most common place for it to pop up is in compiler error messages. So it's important to be aware that this exists, but it shouldn't affect you on a day-to-day basis.

\section{Произвольные поля} % Custom Fields

Occassionally, you will want to define a custom field to be used in your datastore. The most common case is an enumeration, such as employment status. For this, Persistent provides a helper Template Haskell function:

\begin{lstlisting}
{-# LANGUAGE QuasiQuotes, TypeFamilies, GeneralizedNewtypeDeriving, TemplateHaskell,
             OverloadedStrings, GADTs, FlexibleContexts #-}

import Database.Persist
import Database.Persist.Sqlite
import Database.Persist.TH

data Employment = Employed | Unemployed | Retired
    deriving (Show, Read, Eq)
derivePersistField "Employment"

share [mkPersist sqlSettings, mkMigrate "migrateAll"] [persist|
Person
    name String
    employment Employment
|]

main = withSqliteConn ":memory:" $ runSqlConn $ do
    runMigration migrateAll

    insert $ Person "Bruce Wayne" Retired
    insert $ Person "Peter Parker" Unemployed
    insert $ Person "Michael" Employed
\end{lstlisting}%$

derivePersistField stores the data in the database using a string field, and performs marshaling using the Show and Read instances of the datatype. This may not be as efficient as storing via an integer, but it is much more future proof: even if you add extra constructors in the future, your data will still be valid.

\section{Persistent: Raw SQL}

The Persistent package provides a type safe interface to data stores. It tries to be backend-agnostic, such as not relying on relational features of SQL. My experience has been you can easily perform 95\% of what you need to do with the high-level interface. (In fact, most of my web apps use the high level interface exclusively.)

But occassionally you'll want to use a feature that's specific to a backend. One feature I've used in the past is full text search. In this case, we'll use the SQL "LIKE" operator, which is not modeled in Persistent. We'll get all people with the last name "Snoyman" and print the records out.

Actually, you can express a LIKE operator directly in the normal syntax due to a feature added in Persistent 0.6, which allows backend-specific operators. But this is still a good example, so let's roll with it.

\begin{lstlisting}
{-# LANGUAGE OverloadedStrings, TemplateHaskell, QuasiQuotes, TypeFamilies #-}
{-# LANGUAGE GeneralizedNewtypeDeriving, GADTs, FlexibleContexts #-}

import Database.Persist.Sqlite (withSqliteConn)
import Database.Persist.TH (mkPersist, persist, share, mkMigrate, sqlSettings)
import Database.Persist.GenericSql (runSqlConn, runMigration, SqlPersist)
import Database.Persist.GenericSql.Raw (withStmt)
import Data.Text (Text)
import Database.Persist
import Database.Persist.Store (PersistValue)
import Control.Monad.IO.Class (liftIO)
import qualified Data.Conduit as C
import qualified Data.Conduit.List as CL

share [mkPersist sqlSettings, mkMigrate "migrateAll"] [persist|
Person
    name Text
|]

main :: IO ()
main = withSqliteConn ":memory:" $ runSqlConn $ do
    runMigration migrateAll
    insert $ Person "Michael Snoyman"
    insert $ Person "Miriam Snoyman"
    insert $ Person "Eliezer Snoyman"
    insert $ Person "Gavriella Snoyman"
    insert $ Person "Greg Weber"
    insert $ Person "Rick Richardson"

    -- Persistent does not provide the LIKE keyword, but we'd like to get the
    -- whole Snoyman family...
    let sql = "SELECT name FROM Person WHERE name LIKE '%Snoyman'"
    C.runResourceT $ withStmt sql []
                C.$$ CL.mapM_ $ liftIO . print
\end{lstlisting}%$

There is also higher-level support that allows for automated data marshaling. Please see the Haddock API docs for more details.

\section{Интеграция с Yesod} % Integration with Yesod

So you've been convinced of the power of Persistent. How do you integrate it with your Yesod application? If you use the scaffolding, most of the work is done for you already. But as we normally do, we'll build up everything manually here to point out how it works under the surface.

The yesod-persistent package provides the meeting point between Persistent and Yesod. It provides the YesodPersist typeclass, which standardizes access to the database via the runDB method. Let's see this in action.

\begin{lstlisting}
{-# LANGUAGE QuasiQuotes, TypeFamilies, GeneralizedNewtypeDeriving, FlexibleContexts #-}
{-# LANGUAGE TemplateHaskell, OverloadedStrings, GADTs, MultiParamTypeClasses #-}

import Yesod
import Database.Persist.Sqlite

-- Define our entities as usual
share [mkPersist sqlSettings, mkMigrate "migrateAll"] [persist|
Person
    firstName String
    lastName String
    age Int Gt Desc
    deriving Show
|]

-- We keep our connection pool in the foundation. At program initialization, we
-- create our initial pool, and each time we need to perform an action we check
-- out a single connection from the pool.
data PersistTest = PersistTest ConnectionPool

-- We'll create a single route, to access a person. It's a very common
-- occurrence to use an Id type in routes.
mkYesod "PersistTest" [parseRoutes|
/person/#PersonId PersonR GET
|]

-- Nothing special here
instance Yesod PersistTest

-- Now we need to define a YesodPersist instance, which will keep track of
-- which backend we're using and how to run an action.
instance YesodPersist PersistTest where
    type YesodPersistBackend PersistTest = SqlPersist

    runDB action = do
        PersistTest pool <- getYesod
        runSqlPool action pool

-- We'll just return the show value of a person, or a 404 if the Person doesn't
-- exist.
getPersonR :: PersonId -> Handler RepPlain
getPersonR personId = do
    person <- runDB $ get404 personId
    return $ RepPlain $ toContent $ show person

openConnectionCount :: Int
openConnectionCount = 10

main :: IO ()
main = withSqlitePool "test.db3" openConnectionCount $ \pool -> do
    runSqlPool (runMigration migrateAll) pool
    runSqlPool (insert $ Person "Michael" "Snoyman" 26) pool
    warpDebug 3000 $ PersistTest pool
\end{lstlisting}%$

There are two important pieces here for general use. runDB is used to run a DB action from within a Handler. Within the runDB, you can use any of the functions we've spoken about so far, such as insert and selectList.

The type of runDB is runDB :: YesodDB sub master a -> GHandler sub master a. YesodDB is defined as:

\begin{lstlisting}
type YesodDB sub master = YesodPersistBackend master (GHandler sub master)
\end{lstlisting}

Since it is built on top of the YesodPersistBackend associated type, it uses the appropriate database backend based on the current site.

The other new feature is get404. It works just like get, but instead of returning a Nothing when a result can't be found, it returns a 404 message page. The getPersonR function is a very common approach used in real-world Yesod applications: get404 a value and then return a response based on it.

\section{Заключение} % Summary

Persistent brings the type safety of Haskell to your data access layer. Instead of writing error-prone, untyped data access, or manually writing boilerplate marshal code, you can rely on Persistent to automate the process for you.

The goal is to provide everything you need, most of the time. For the times when you need something a bit more powerful, Persistent gives you direct access to the underlying data store, so you can write whatever 5-way joins you want.

Persistent integrates directly into the general Yesod workflow. Not only do helper packages like yesod-persistent provide a nice layer, but packages like yesod-form and yesod-auth also leverage Persistent's features as well.
