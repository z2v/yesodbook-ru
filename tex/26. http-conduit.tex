\chapter{http-conduit}

Most of Yesod is about serving content over HTTP. But that's only half
the story: someone has to receive it. And even when you're writing a
web app, sometimes that someone will be you. If you want to consume
content from other services or interact with RESTful APIs, you'll need
to write client code. And the recommended approach for that is
http-conduit.

This chapter is not directly connected to Yesod, and will be generally
useful for anyone wanting to make HTTP requests.

\section{Synopsis}
\begin{lstlisting}
  {-# LANGUAGE OverloadedStrings #-}
import Network.HTTP.Conduit -- the main module

-- The streaming interface uses conduits
import Data.Conduit
import Data.Conduit.Binary (sinkFile)

import qualified Data.ByteString.Lazy as L
import Control.Monad.IO.Class (liftIO)

main :: IO ()
main = do
    -- Simplest query: just download the information from the given URL as a
    -- lazy ByteString.
    simpleHttp "http://www.example.com/foo.txt" >>= L.writeFile "foo.txt"

    -- Use the streaming interface instead. We need to run all of this inside a
    -- ResourceT, to ensure that all our connections get properly cleaned up in
    -- the case of an exception.
    runResourceT $ do
        -- We need a Manager, which keeps track of open connections. simpleHttp
        -- creates a new manager on each run (i.e., it never reuses
        -- connections).
        manager <- liftIO $ newManager def

        -- A more efficient version of the simpleHttp query above. First we
        -- parse the URL to a request.
        req <- liftIO $ parseUrl "http://www.example.com/foo.txt"

        -- Now get the response
        res <- http req manager

        -- And finally stream the value to a file
        responseBody res $$ sinkFile "foo.txt"

        -- Make it a POST request, don't follow redirects, and accept any
        -- status code.
        let req2 = req
                { method = "POST"
                , redirectCount = 0
                , checkStatus = \_ _ -> Nothing
                }
        res2 <- http req2 manager
        responseBody res2 $$ sinkFile "post-foo.txt"
\end{lstlisting} %$

\section{Concepts}

The simplest way to make a request in http-conduit is with the
simpleHttp function. This function takes a String giving a URL and
returns a ByteString with the contents of that URL. But under the
surface, there are a few more steps:

- A new connection Manager is allocated.
- The URL is parsed to a Request. If the URL is invalid, then an
  exception is thrown.
- The HTTP request is made, following any redirects from the server.
- If the response has a status code outside the 200-range, an exception is thrown.
- The response body is read into memory and returned.
- runResourceT is called, which will free up any resources (e.g., the open
  socket to the server).

If you want more control of what's going on, then you can configure
any of the steps above (plus a few more) by explicitly creating a
Request value, allocating your Manager manually, and using the http
and httpLbs functions.

\section{Request}

The easiest way to creating a Request is with the parseUrl
function. This function will return a value in any Failure monad, such
as Maybe or IO. The last of those is the most commonly used, and
results in a runtime exception whenever an invalid URL is
provided. However, you can use a different monad if, for example, you
want to validate user input.
\begin{lstlisting}
import Network.HTTP.Conduit
import System.Environment (getArgs)
import qualified Data.ByteString.Lazy as L
import Control.Monad.IO.Class (liftIO)

main :: IO ()
main = do
    args <- getArgs
    case args of
        [urlString] ->
            case parseUrl urlString of
                Nothing -> putStrLn "Sorry, invalid URL"
                Just req -> withManager $ \manager -> do
                    Response _ _ _ lbs <- httpLbs req manager
                    liftIO $ L.putStr lbs
        _ -> putStrLn "Sorry, please provide exactly one URL"
\end{lstlisting}

The Request type is abstract so that http-conduit can add new settings
in the future without breaking the API (see the Settings Type chapter
for more information). In order to make changes to individual records,
you use record notation. For example, a modification to our program
that issues HEAD requests and prints the response headers would be:
\begin{lstlisting}
{-# LANGUAGE OverloadedStrings #-}
import Network.HTTP.Conduit
import System.Environment (getArgs)
import qualified Data.ByteString.Lazy as L
import Control.Monad.IO.Class (liftIO)

main :: IO ()
main = do
    args <- getArgs
    case args of
        [urlString] ->
            case parseUrl urlString of
                Nothing -> putStrLn "Sorry, invalid URL"
                Just req -> withManager $ \manager -> do
                    let reqHead = req { method = "HEAD" }
                    Response status _ headers _ <- http reqHead manager
                    liftIO $ do
                        print status
                        mapM_ print headers
        _ -> putStrLn "Sorry, please provide example one URL"
\end{lstlisting} %$

There are a number of different configuration settings in the API,
some noteworthy ones are:

    proxy
      Allows you to pass the request through the given proxy server.
    redirectCount
      Indicate how many redirects to follow. Default is 10.
    checkStatus
      Check the status code of the return value. By default, gives an
      exception for any non-2XX response.
    requestBody
      The request body to be sent. Be sure to also update the method. For the
      common case of url-encoded data, you can use the urlEncodedBody function.

\section{Manager}

The connection manager allows you to reuse connections. When making
multiple queries to a single server (e.g., accessing Amazon S3), this
can be critical for creating efficient code. A manager will keep track
of multiple connections to a given server (taking into account port
and SSL as well), automatically reaping unused connections as
needed. When you make a request, http-conduit first tries to check out
an existing connection. When you're finished with the connection (if
the server allows keep-alive), the connection is returned to the
manager. If anything goes wrong, the connection is closed.

To keep our code exception-safe, we use the ResourceT monad
transformer. All this means for you is that your code needs to be
wrapped inside a call to runResourceT, either implicitly or
explicitly, and that code inside that block will need to liftIO to
perform normal IO actions.

There are two ways you can get ahold of a manager. newManager will
return a manager that will not be automatically closed (you can use
closeManager to do so manually), while withManager will start a new
ResourceT block, allow you to use the manager, and then automatically
close the ResourceT when you're done. If you want to use a ResourceT
for an entire application, and have no need to close it, you should
probably use newManager.

One other thing to point out: you obviously don't want to create a new
manager for each and every request; that would defeat the whole
purpose. You should create your Manager early and then share it.

\section{Response}

The Response datatype has three pieces of information: the status
code, the response headers, and the response body. The first two are
straight-forward; let's discuss the body.

The Response type has a type variable to allow the response body to be
of multiple types. If you want to use http-conduit's streaming
interface, you want this to be a Source. For the simple interface, it
will be a lazy ByteString. One thing to note is that, even though we
use a lazy ByteString, the entire response is held in memory. In other
words, we perform no lazy I/O in this package.

\fbox{\begin{minipage}[!h]{.9\linewidth}
The conduit package does provide a lazy module which would allow you
to read this value in lazily, but like any lazy I/O, it's a bit
unsafe, and definitely non-deterministic. If you need it though, you
can use it.
\end{minipage}
}

\section{http and httpLbs}

So let's tie it together. The http function gives you access to the
streaming interface (i.e., it returns a Response using a
BufferedSource) while httpLbs returns a lazy ByteString. Both of these
return values in the ResourceT transformer so that they can access the
Manager and have connections handled properly in the case of
exceptions.

\fbox{\begin{minipage}[!h]{.9\linewidth}
If you want to ignore the remainder of a large response body, you can
connect to the sinkNull sink. The underlying connection will
automatically be closed, preventing you from having to read a large
response body over the network.
\end{minipage}
}
