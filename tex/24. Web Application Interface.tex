\chapter {Интерфейс веб-приложения}\label{chap:web_application_interface}

Практически каждый язык, используемый для веб-разработки, сталкивался с проблемой
низкоуровневого интерфейса между веб-сервером и приожением. Самым ранним примером
её решения является почтенный и потёртый в боях CGI

It is a problem almost every language used for web development has dealt with: 
the low level interface between the web server and the application. The earliest 
example of a solution is the venerable and battle-worn CGI (CGI), providing a 
language-agnostic interface using only standard input, standard output and 
environment variables.

Back when Perl was becoming the de facto web programming language, a major 
shortcoming of CGI became apparent: the process needed to be started anew for 
each request. When dealing with an interpretted language and application 
requiring database connection, this overhead became unbearable. FastCGI 
(and later SCGI) arose as a successor to CGI, but it seems that much of 
the programming world went in a different direction.

Each language began creating its own standard for interfacing with servers. mod\_perl.
mod\_python. mod\_php. mod\_ruby. Within the same language, multiple interfaces arose. In
some cases, we even had interfaces on top of interfaces. And all of this led to much
duplicated effort: a Python application designed to work with FastCGI wouldn't work with
mod\_python; mod\_python only exists for certain webservers; and these programming
language specific web server  extensions need to be written for each programming
language.

Haskell has its own history. We originally had the cgi package, which provided a monadic 
interface. The fastcgi package then provided the same interface. Meanwhile, it seemed that 
the majority of Haskell web development focused on the standalone server. The problem is 
that each server comes with its own interface, meaning that you need to target a specific 
backend. This means that it is impossible to share common features, like GZIP encoding, 
development servers, and testing frameworks.

WAI attempts to solve this, by providing a generic and efficient interface between web 
servers and applications. Any handler supporting the interface can serve any WAI 
application, while any application using the interface can run on any handler.

At the time of writing, there are various backends, including Warp, FastCGI, and
development server. There are even more esoteric backends like wai-handler-webkit 
for creating desktop apps. wai-extra provides many common middleware components 
like GZIP, JSON-P and virtual hosting. wai-test makes it easy to write unit
tests, and wai-handler-devel lets you develop your
applications without worrying about stopping to compile. Yesod targets WAI, and
Happstack is in the process of converting over as well. It's also used by some
applications that skip the framework entirely, including the new Hoogle.

\fbox{\begin{minipage}[!h]{.9\linewidth}
Yesod provides an alternate approach for a devel server, known as
yesod devel. The difference from wai-handler-devel is that yesod
devel actually compiles your code each time, respecting all settings in your 
cabal file.
This is the recommended aproach for general Yesod development.
\end{minipage}}

\section {The Interface}

The interface itself is very straight-forward: an application takes a request 
and returns a response. A response is an HTTP status, a list of headers and a 
response body. A request contains various information: the requested path, 
query string, request body, HTTP version, and so on.

\section {Response Body}

Haskell has a datatype known as a lazy bytestring. By utilizing laziness, you 
can create large values without exhausting memory. Using lazy I/O, you can do 
such tricks as having a value which represents the entire contents of a file, 
yet only occupies a small memory footprint. In theory, a lazy bytestring is the 
only representation necessary for a response body.

In practice, while lazy byte strings are wonderful for generating "pure" values, the
lazy I/O necessary to read a file introduces some non-determinism into our programs.
When serving thousands of small files a second, the limiting factor is not memory, but
file handles. Using lazy I/O, file handles may not be freed immediately, leading to
resource exhaustion. To deal with this, WAI uses conduits.

\fbox{\begin{minipage}[!h]{.9\linewidth}
Versions of WAI before 1.0 used enumerators in place of conduits. While both conduits
and enumerators solve the same basic problem, experience showed that enumerators were
too constricting in their inversion of control approach, making it difficult to
structure more complicated systems like a streaming proxy server. Conduits were designed
with the express purpose of making a better WAI.
\end{minipage}}

The data type relevant to us now is a source. A source produces a stream of
data, producing a single chunk at a time. In the case of WAI, the request body would be
a source passed to the application, and the response body would be a source returned
from the application.

There are two further optimizations: many systems provide a sendfile system call, which 
sends a file directly to a socket, bypassing a lot of the memory copying inherent in more
general I/O system calls. Additionally, there is a datatype in Haskell called Builder which 
allows efficient copying of bytes into buffers.

The WAI response body therefore has three constructors: one for pure builders
(ResponseBuilder), one for a source of builders (ResponseSource) and one for files (ResponseFile).

\section {Request Body}

In order to avoid the need to load the entire request body into memory, we use
sources here as well. Since the purpose of these values are for reading (not writing),
we use ByteStrings in place of Builders. There is a record inside Request called
requestBody, with type BufferedSource IO ByteString. We can use all of the standard 
conduit functions to interact with this source.

The request body could in theory contain any type of data, but the most common are 
URL encoded and multipart form data. The wai-extra package contains built-in support 
for parsing these in a memory-efficient manner.

\section{Hello World}

To demonstrate the simplicity of WAI, let's look at a hello world example. 
In this example, we're going to use the OverloadedStrings language extension to avoid 
explicitly packing string values into bytestrings.

\begin{lstlisting}
{-# LANGUAGE OverloadedStrings #-}
import Network.Wai
import Network.HTTP.Types (status200)
import Network.Wai.Handler.Warp (run)

application _ = return $
  responseLBS status200 [("Content-Type", "text/plain")] "Hello World"

main = run 3000 application
\end{lstlisting}%$

Lines 2 through 4 perform our imports. Warp is provided by the warp package, and is 
the premiere WAI backend. WAI is also built on top of the http-types package, which 
provides a number of datatypes and convenience values, including status200.

First we define our application. Since we don't care about the specific request 
parameters, we ignore the argument to the function. For any request, we are 
returning a response with status code 200 ("OK"), and text/plain content type 
and a body containing the words "Hello World". Pretty straight-forward.

\section {Middleware}

In addition to allowing our applications to run on multiple backends without 
code changes, the WAI allows us another benefits: middleware. Middleware is essentially 
an application transformer, taking one application and returning another one.

Middleware components can be used to provide lots of services: cleaning up URLs,
authentication, caching, JSON-P requests. But
perhaps the most useful and most intuitive
middleware is gzip compression. The middleware
works very simply: it parses the request headers
to determine if a client supports compression, and
if so compresses the response body and adds the
appropriate response header.

The great thing about middlewares is that they are unobtrusive. Let's see how we would 
apply the gzip middleware to our hello world application.

\begin{lstlisting}
{-# LANGUAGE OverloadedStrings #-}
import Network.Wai
import Network.Wai.Handler.Warp (run)
import Network.Wai.Middleware.Gzip (gzip, def)
import Network.HTTP.Types (status200)

application _ = return $ responseLBS status200 [("Content-Type", "text/plain")]
                       "Hello World"

main = run 3000 $ gzip def application
\end{lstlisting}%$

We added an import line to actually have access to the middleware, and then simply 
applied gzip to our application. You can also chain together multiple middlewares: 
a line such as gzip False \$ jsonp \$ othermiddleware \$ myapplication is perfectly valid. 
One word of warning: the order the middleware is applied can be important. For example, 
jsonp needs to work on uncompressed data, so if you apply it after you apply gzip, 
you'll have trouble.

