\chapter{REST-содержимое}\label{chap:restful-content}

Существуют истории о том, как  на заре Интернета поисковые системы уничтожали
целые веб-сайты. Когда динамическое содержимое было ещё новой концепцией,
разработчики не принимали во внимание различий между GET и POST запросами. В
результате, они создавали страницы с доступом через GET-метод, которые могли
удалить содержимое. Когда поисковики сканировали такие сайты, они могли всё
удалить.

Если бы эти разработчики точнее следовали спецификации HTTP, этого бы не
случалось.  GET-запрос подразумевает отсутствие побочных эффектов (вроде
стирания сайта, к примеру).  Не так давно, в веб-разработке произошло движение
к правильному пониманию <<передачи состояния представления>> (REpresentational
State Transfer), известного также как REST.  Эта глава описывает особенности
Yesod, связанные с поддержкой REST, и как они могут быть использованы для
создания более надёжных веб-приложений.

\section{Методы HTTP-запросов}

Во многих веб-фреймворках вы пишете для ресурса одну функцию-обработчик. В
Yesod по умолчанию предполагается отдельная функция-обработчик для каждого
метода запроса. Два самых частых метода, с которыми вы будете иметь дело при
создании веб-сайтов~--- это GET и POST. Эти методы лучше всего поддерживаются в
формате HTML, поскольку только они используются в веб-формах. Однако при
создании RESTful API, другие методы также очень полезны.

Технически, вы можете создать какие угодно методы, но настоятельно
рекомендуется придерживаться описанных в спецификации HTTP. Наиболее часто
встречаются методы:
\begin{itemize}
    \item GET. Запрос на чтение. Предполагается, что при вызове GET не
        произойдёт никаких изменений на сервере. При многократном вызове
        GET-метода, сервер должен выдавать тот же ответ, если исключить такие
        вещи как <<текущее время>> или результаты, получаемые случайно.
    \item POST. Обычный запрос на изменение. Он никогда не должен посылаться
        пользователем дважды. Обычно, в качестве примера POST-запроса приводят
        перевод денежных средств с одного банковского счёта на другой.
    \item PUT. Создание нового ресурса на сервере или изменение существующего.
        Этот метод \emph{безопасен} при многократном применении.
    \item DELETE. В точности соответствует названию. Удаляет ресурс с сервера.
        Повторные вызовы не должны приводить к проблемам.
\end{itemize}

В определённой степени, это соответствует философии Haskell: GET-запрос подобен
чистой функции, без побочных эффектов. На практике ваши GET-функции вероятно
будут производить IO-операции~--- чтение информации из базы данных,
протоколирование пользовательских действий и так далее.

В главе \hyperref[chap:routing]{Маршрутизация URL и обработчики} можно найти
более полную информацию по синтаксису определения функций-обработчиков для
каждого метода HTTP-запроса.

\section{Представления}

Пусть мы определили тип данных и значение:

\begin{lstlisting}
data Person = Person { name :: String, age :: Int }
michael = Person "Michael" 25
\end{lstlisting}

Мы можем представить эти данные в HTML:
\begin{lstlisting}[language=HTML]
 <table>
    <tr>
        <th>Name</th>
        <td>Michael</td>
    </tr>
    <tr>
        <th>Age</th>
        <td>25</td>
    </tr>
</table>
\end{lstlisting}

или в JSON:

\begin{lstlisting}
{"name":"Michael","age":25}
\end{lstlisting}

или в XML:
\begin{lstlisting}[language=XML]
<person>
    <name>Michael</name>
    <age>25</age>
</person>
\end{lstlisting}

Зачастую веб-приложения используют различные URL для получения этих
представлений: возможно \texttt{/person/michael.html},
\texttt{/person/michael.json}, и т.п. Yesod, следуя принципам RESTful,
использует один URL для каждого ресурса. Поэтому в Yesod все представления
будут доступны через \texttt{/person/michael}.

Возникает вопрос, как мы определим, \emph{какое} представление требуется. Ответ
можно найти в заголовке~\texttt{Accept} запроса: в нём находится список типов
содержимого, ожидаемых клиентом, в приоритетном порядке. Yesod предоставляет
пару функций для абстрагирования от деталей разбора заголовка напрямую и
позволяет вам работать на гораздо более высоком уровне представления.  Немного
конкретизируем последнее предложение следующим кодом:

\includecode{12/selectRep.hs}

Функция \lstinline'selectRep' говорит: <<Я верну вам некоторые возможные
представления>>. Каждый вызов функции~\lstinline'provideRep' готовит
альтернативное представление. Yesod использует типы Haskell для определения
MIME-типа для каждого представления. Так как \lstinline'shamlet' (он же
<<простой Hamlet>>) создаёт значение~\lstinline'Html', Yesod может определить,
что соответствующий MIME-тип~--- это~\texttt{text/html}. Аналогично,
\lstinline'object' создаёт значение~\lstinline'JSON', для которого мы получаем
MIME-тип~\texttt{application/json}. \lstinline'TypedContent'~--- это тип
данных, предоставляемый Yesod, для некоторых типов содержимого с привязанным
MIME-типом. Мы рассмотрим его подробнее чуть ниже.

Для проверки запустите сервер и затем попробуйте выполнить следующие команды,
использующие утилиту~\texttt{curl}:

\begin{lstlisting}[language=]
curl http://localhost:3000 --header "accept: application/json"
curl http://localhost:3000 --header "accept: text/html"
curl http://localhost:3000
\end{lstlisting}

Обратите внимание, как ответ изменяется в зависимости от значение~accept
заголовка. Кроме того, если вы опускаете заголовок, по умолчанию возвращается
HTML. Правило здесь такое: если в заголовке нет поля~accept, возвращается
первое представление. Если поле~accept в заголовке присутствует, но нет
подходящего представления, возвращается ответ 406 <<Not Acceptable>>.

По умолчанию, Yesod предоставляет удобную компоненту промежуточного слоя
(middleware), которая позволяет задать поле~accept заголовка, используя
параметры строки запроса. Она делает проще тестирование с использованием
браузера. Для пробы зайдите по ссылке
\texttt{http://localhost:3000/?\_accept=application/json}.

\subsection{Удобные средства для работы с JSON}
Так как в настоящее время JSON стал часто используемым форматом в
веб-приложениях, мы реализовали несколько встроенных функций для подготовки
JSON представлений. Эти функции построены на основе замечательной
библиотеки~\texttt{aeson}, поэтому начнём мы с краткого введения в работу этой
библиотеки.

Библиотека \texttt{aeson} основана на типе данных~\lstinline'Value', который
представляет собой любой корректное значение JSON. Также она предоставляет два
класса типов: \lstinline'ToJSON' и~\lstinline'FromJSON' для автоматизации
преобразования в JSON значений и из них, соответственно. Для наших целей, нас
сейчас интересует \lstinline'ToJSON'. Давайте рассмотрим небольшой пример
создания экземпляра класса~\lstinline'ToJSON' для нашего постоянно
используемого типа данных~\lstinline'Person'.

\includecode{12/personJson.hs}

Я не буду дальше погружаться в детали работы пакета~\texttt{aeson}, так как
\footnotehref{https://www.fpcomplete.com/haddocks/aeson}{Haddock} документация
уже предоставляет великолепное введение для библиотеки. Того, что я описал
выше, достаточно для понимания наших удобных функций.

Предположим, что у вас есть такой тип данных~\lstinline'Person', с
соответствующим значением, и вы хотите использовать его в представлении вашей
текущей страницы. Для этого вы можете воспользоваться
функцией~\lstinline'returnJson'.

\includecode{12/returnJson.hs}

На самом деле, \lstinline'returnJson' тривиальная функция, реализованная как
\lstinline'return . toJSON'. Однако, она делает жизнь слегка удобнее.
Аналогично, если вы хотите вернуть значение JSON как представление внутри
\lstinline'selectRep', вы можете воспользоваться
функцией~\lstinline'provideJson'.

\includecode{12/provideJson.hs}

\lstinline'provideJson' столь же тривиальна:
\lstinline'provideRep . returnJson'.

\subsection{Новые типы данных}
Предположим, что у меня есть некоторый новый формат данных, основанный на
использовании экземпляров класса~\lstinline'Show' Haskell. Я назову его
<<Haskell Show>>, и присвою ему MIME-тип~\texttt{text/haskell-show}. Также
предположим, что я решаю включить это новое представление в моё веб-приложение.
Как я могу это сделать? Для начала, давайте попробуем напрямую воспользоваться
типом данных~\lstinline'TypedContent'.

\includecode{12/haskell-show.hs}

Отметим несколько важных моментов:
\begin{itemize}
    \item Мы воспользовались функцией~\lstinline'toContent'. Эта функция класса
        типов, которая может преобразовать ряд типов данных в необработанные
        данные, готовые к отправке по сети. В данном случае, мы взяли экземпляр
        для типа~\lstinline'String', который использует кодировку UTF8. Другие
        общие типы данных, имеющие свои экземпляры, это: \lstinline'Text',
        \lstinline'ByteString', \lstinline'Html' и~\lstinline'Value' из пакета
        aeson.
    \item Мы использовали конструктор~\lstinline'TypedContent' напрямую. Он
        принимает два аргумента: MIME-тип и необработанное содержимое. Обратите
        внимание, что \lstinline'ContentType'~--- это просто синоним для
        строгого варианта~\lstinline'ByteString'.
\end{itemize}

Всё это хорошо, но меня беспокоит, что сигнатура функции~\lstinline'getHomeR'
совершенно неинформативна. Кроме того, её реализация выглядит достаточно
шаблонно. Я бы предпочёл иметь тип данных, представляющий данные для <<Haskell
Show>>, и предоставить некоторые простые средства для создания таких значений.
Давайте попробуем сделать так:

\includecode{12/haskell-show2.hs}

Магия здесь прячется в двух классах типов. Как мы видели выше,
\lstinline'ToContent' говорит, как преобразовать значение в необработанный
запрос. В нашем случае, мы бы хотели, используя \lstinline'show', получить
\lstinline'String' из исходного значения, а затем преобразовать значение
типа~\lstinline'String' в необработанные содержимое. Зачастую, экземпляры
класса~\lstinline'ToContent' будут строиться друг на друге таким же образом.

Класс \lstinline'ToTypedContent' используется внутри Yesod и вызывается с
выходными значениями всех функций-обработчиков. Как вы можете видеть,
реализация в целом тривиальна: устанавливается MIME-тип и вызывается
функция~\lstinline'toContent'.

И, наконец, давайте немного усложним задачу и добавим совместимость с
функцией~\lstinline'selectRep'.

\includecode{12/haskell-show3.hs}

Важное дополнение~--- экземпляр класса~\lstinline'HasContentType'. Это может
показаться излишним, но играет важную роль. Нам требуется возможность
определять MIME-тип представления \emph{до создания этого самого
    представления}. \lstinline'ToTypedContent' работает только с конкретным
значением и, соответственно, не может быть использован до того, как значение
создано. Функция~\lstinline'getContentType' принимает принимает промежуточное
значение и возвращает MIME-тип, не предоставляя какой-либо конкретики.

\begin{remark}
    Если вы хотите подготовить представление для значения, которое не имеет
    экземпляра класса~\lstinline'HasContentType', вы можете воспользоваться
    функцией~\lstinline'provideRepType', которая требует явной передачи
    MIME-типа.
\end{remark}

\section{Другие заголовки запроса}

Существует большое количество других заголовков запроса. Некоторые из них
влияют только на передачу данных между сервером и клиентом, и не должны влиять
на приложение в целом.  Например, \texttt{Accept-Encoding} сообщает серверу о
том, какие схемы сжатия понимает клиент, а \texttt{Host} информирует сервер,
какие виртуальные хосты обслуживаются.

Другие заголовки \emph{влияют} на приложение и автоматически считываются в Yesod.
Например, заголовок \texttt{Accept-Language} определяет, какой язык предпочитает
клиент (английский, испанский, немецкий и т.п.). В \hyperref[chap:i18n]{части
    про i18n} детально описывается, как используется этот заголовок.

\section{Отсутствие состояния}

Я оставил этот раздел напоследок, не от того, что он менее важен, а скорее
потому что здесь нет особенностей Yesod, обеспечивающих соответствующую
поддержку.

Протокол HTTP не поддерживает состояний (т.е. является stateless-протоколом):
каждый запрос рассматривается как начало общения. Это означает, к примеру, что
серверу не важно то, что вы уже запрашивали пять страниц~--- он будет
обрабатывать ваш шестой запрос так, как если бы он был самым первым.

С другой стороны, некоторая функциональность сайта не будет работать без
хранения состояния. Например, как реализовать корзину покупок, если не хранить
информацию о покупках между запросами?

Решением являются куки-файлы (cookies) и построенные на их основе сессии. У нас
есть целый \hyperref[chap:sessions]{раздел}, посвящённый сессиям в Yesod.
Однако я не могу не подчеркнуть, что их следует использовать с осторожностью.

Позвольте привести пример. Существует популярная система отслеживания ошибок, с
которой я работал ежедневно и которая слишком много использовала сессии. Там
есть маленький выпадающий список на каждой странице для выбора текущего
проекта. Ничто не предвещает проблем, да? Всё, что этот список делает~---
устанавливает текущий проект для вашей сессии.

В результате, щелчок на ссылке <<просмотреть ошибки>> внутренне зависит от
того, какой проект вы выбирали последним. Из-за этого нет способа сделать
закладку на ваши открытые ошибки для Yesod и отдельную ссылку на ваши ошибки
для Hamlet.

Правильный RESTful подход состоит в том, чтобы иметь один ресурс для всех
Yesod-ошибок, а другой для всех Hamlet. В Yesod это легко делается примерно
таким определением маршрута:
\begin{lstlisting}
/                      ProjectsR        GET
/projects/#ProjectID   ProjectIssuesR   GET
/issues/#IssueID       IssueR           GET
\end{lstlisting}

Будьте внимательны к своим пользователям: правильная архитектура без
использования состояний означает, что такие базовые вещи, как закладки,
постоянные ссылки и кнопки <<Вперёд>>, <<Назад>> будут всегда работать.

\section{Выводы}

Yesod придерживается следующих принципов REST:
\begin{itemize}
    \item использовать правильные методы запроса;
    \item каждый ресурс должен иметь в точности один URL;
    \item разрешать множественные представления данных для одного URL;
    \item проверять заголовки запросов для определения дополнительной
        информации о том, что ждёт клиент.
\end{itemize}

Это упрощает использование Yesod не только для построения веб-сайтов, но также
и для построения программных интерфейсов (API). В действительности, используя
технику наподобие \lstinline'selectRep'/\lstinline'provideRep', с одного URL вы
можете обслужить одновременно <<дружелюбную к пользователю>> HTML-страницу и
<<дружелюбную к машине>> JSON-страницу.
