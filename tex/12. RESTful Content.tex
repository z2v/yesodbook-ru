trRESTful содержимое

Одна из историй ранних дней развития Интернета - как поисковые системы уничтожали целые web-сайты. Когда динамические сайты были еще новой концепцией, разработчики не принимали во внимание различий между GET и POST запросами. В результате, они создавали страницы, удаляющие содержимое, с доступомчерез GET-метод. Когда поисковики начинали сканировать эти сайты, они могли удалить все содержимое.

Если бы эти разработчики точнее следовали спецификации HTTP, этого бы не случилось. GET-запрос подразумевает отсутствие побочных эффектов (вроде стирания сайта, к примеру). Не так давно, в веб-разработке произошло движение к правильному охвату “Передачи репрезентативного состояния” (Representational State Transfer), известного также как REST. Эта часть описывает особенности Yesod, связаные с поддержкой REST, и как они могут быть использованы для создания более надежных web-приложений.

Методы HTTP-запросов

Во многих веб-фреймворках, вы пишете для ресурса одну функцию-обработчик (handler). В Yesod, по умолчанию, предполагается отдельная функция-обработчик для каждого метода запроса. Два самых частых метода, с которыми вы будете иметь дело при создании веб-сайтов - это методы GET и POST. Эти методы лучше всего поддерживаются в HTML, поскольку только они поддерживаются web-формами. Однако, при создании REST-API, другие методы также очень полезны.

Технически, вы можете создать какие угодно методы, но настоятельно рекомендуется придерживаться методов, описанных в спецификации HTTP. Наиболее часто встречаются методы:

• GET
Запрос на чтение. Предполагается, что, при вызове GET, не произойдет никаких изменений на сервере. При многократном вызове GET-метода, сервер должен выдавать тот же ответ, если исключить такие вещи как “текущее время” или случайно полученные результаты.

• POST
Обычный запрос на изменение. POST-запрос никогда не должен посылаться пользователем дважды. Обычно, в качестве примера POST-запроса приводят перевод денежных средств с одного банковского счета на другой.

• PUT
Создание нового ресурса на сервере или изменение существующего. Этот метод может быть выполнен многократно с одинаковым эффектом.

• DELETE
В точности соответствует названию. Удаляет ресурс с сервера. Повторные вызовы не должны приводить к проблемам.

В определенной степени, это соответствует философии Haskell: GET-запрос подобен чистой функции, без побочных эффектов. На практике, ваши GET-функции вероятно будут производить IO-операции - чтение информации из базы данных, логирование пользовательских действий и так далее.

Смотрите часть “Routing and Handlers” для более полной информации по синтаксису определения функций-обработчиков для каждого метода HTTP-запроса.

Представления

Пусть мы определили тип данных и значение:

\begin{lstlisting}
data Person = Person { name :: String, age :: Int }
michael = Person "Michael" 25
\end{lstlisting}

Мы можем представить эти данные в HTML:
\begin{lstlisting}[language=HTML]
 <table>
    <tr>
        <th>Name</th>
        <td>Michael</td>
    </tr>
    <tr>
        <th>Age</th>
        <td>25</td>
    </tr>
</table>
\end{lstlisting}

или в JSON:

\begin{lstlisting}[language=Java]
{"name":"Michael","age":25}
\end{lstlisting}

или в XML:
\begin{lstlisting}[language=XML]
<person>
    <name>Michael</name>
    <age>25</age>
</person>
\end{lstlisting}

Зачастую, веб-приложения используют различные URL для получения этих представлений: возможно \lstinline '/person/michael.html', \lstinline '/person/michael.json', и т.п. Yesod, следуя принципам REST, использует один URL для каждого ресурса. Поэтому в Yesod все представления будут доступны через \lstinline '/person/michael'.

Возникает вопрос, как мы определим, какое представление требуется. Ответ в заголовке HTTP Accept: он получает приоритезированный список типов содержимого, ожидаемого клиентом. Yesod будет автоматически опраделять какое представление требуется, основываясь на этом заголовке.

Немного конкретизируем последнее предложение следующим кодом:

\begin{lstlisting}
type ChooseRep = [ContentType] -> IO (ContentType, Content)
class HasReps a where
    chooseRep :: a -> ChooseRep
\end{lstlisting}

Функция \lstinline `chooseRep` получает два аргумента: значение, для которого мы получаем представления и список типов содержимого, которые клиент будет получать. Список определяется чтением из заголовка Accept-запроса. \lstinline `chooseRep` возвращает пару, включающую тип содержимого ответа и непосредственно содержимое.

Этот класс типов - основа REST-подхода, принятого в Yesod для представлений. Каждая функция-обработчик должна возвращать экземпляр \lstinline `HasReps`. Когда Yesod генерирует функцию-диспетчер, он автоматически применяет \lstinline 'chooseRep' к каждому обработчику, по существу давая всем функциям тип \lstinline 'Handler \lstinline ChooseRep'. После  запуска \lstinline 'Handler' и получения результата \lstinline 'ChooseRep',  он применяет список типов содержимого, полученный из Accept-заголовка.

В Yesod определены многочисленные экземпляры класса HasReps. Например, когда мы используем \lstinline `defaultLayout`, возвращается тип \lstinline 'RepHtml', который выглядит так:

\begin{lstlisting}
newtype RepHtml = RepHtml Content
instance HasReps RepHtml where
    chooseRep (RepHtml content) _ = return ("text/html", content)
\end{lstlisting}

Заметьте, что мы полностью игнорируем список ожидаемых типов содержимого. Многочисленные встроенные представления (\lstinline 'RepHtml', \lstinline 'RepPlain', \lstinline 'RepJson', \lstinline 'RepXml'), в действительности, поддерживают только единственное представление и, следовательно, то, что содержится в Accept-заголовке клиентского запроса, является несущественным.

\lstinline 'RepHtmlJson'

Противоположный пример - \lstinline 'RepHtmlJson', который создает либо HTML либо JSON представление. Этот экземпляр класса особенно полезен в программировании AJAX приложений. Вот пример, возвращающий либо HTML либо JSON данные, в зависимости от того, что хочет клиент. 
%FIXME перевел "This instance helps greatly in programming AJAX applications that degrade nicely" как "Этот экземпляр класса особенно полезен в программировании AJAX приложений". Как перевести "that degrade nicely" (что ухудшает красиво) не придумал и просто опустил.

\begin{lstlisting}
{-# LANGUAGE QuasiQuotes, TypeFamilies, OverloadedStrings #-}
{-# LANGUAGE MultiParamTypeClasses, TemplateHaskell #-}
import Yesod
data R = R
mkYesod "R" [parseRoutes|
/ RootR GET
/#String NameR GET
|]
instance Yesod R

getRootR = defaultLayout $ do
    setTitle "Homepage"
    addScriptRemote "http://ajax.googleapis.com/ajax/libs/jquery/1.4/jquery.min.js"
    toWidget [julius|
$(function(){
    $("#ajax a").click(function(){
        jQuery.getJSON($(this).attr("href"), function(o){
            $("div").text(o.name);
        });
        return false;
    });
});
|]
    let names = words "Larry Moe Curly"
    [whamlet|
<h2>AJAX Version
<div #results>
    AJAX results will be placed here when you click #
    the names below.
<ul #ajax>
    $forall name <- names
        <li>
            <a href=@{NameR name}>#{name}

<h2>HTML Version
<p>
    Clicking the names below will redirect the page #
    to an HTML version.
<ul #html>
    $forall name <- names
        <li>
            <a href=@{NameR name}>#{name}

|]

getNameR name = do
    let widget = do
            setTitle $ toHtml name
            [whamlet|Looks like you have Javascript off. Name: #{name}|]
    let json = object ["name" .= name]
    defaultLayoutJson widget json

main = warpDebug 4000 R

\end{lstlisting}

Наш \lstinline обработчик 'getRootR' создает страницу c тремя ссылками и Javascript-кодом, который перехватывает нажатия на ссылки и производит асинхронные запросы. Если у пользователя разрешен Javascript, то нажатие на ссылку приводит к запросу, отсылаемому с Accept-заголовком application/json. В этом случае, \lstinline 'getNameR' будет возвращать JSON-представление.

Если пользователь запрещает Javascript, то нажатие на ссылку будет отсылать пользователя к соответствующему URL. Браузер укажет приоритет HTML-представления данных и, следовательно, будет возвращена страница, определенная виджетом.

Мы конечно можем расширить это поведение для работы с XML, Atom feeds, или даже двоичным представлением данных. Для развлечения можно было бы написать веб-приложение, которое готовит данные просто используя экземпляры по умолчанию класса \lstinline 'Show' типов данных. И затем написать веб-клиента, который разбирает результаты, используя экземпляры по умолчанию класса Read.
%FIXME Atom feeds

Возможно, вас беспокоит эффективность этого подхода. Разве это не означает, что мы должны генерировать и HTML и JSON ответы для каждого запроса? Благодаря ленивости этого не происходит. В функции \lstinline 'getNameR' ни \lstinline 'widget' ни \lstinline 'json' не будет вычислен, пока не будет выбран соответствующий тип ответа, и, следовательно, всегда будет выполняться только один из них.

\subsubsection News Feeds





 



