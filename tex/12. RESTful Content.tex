\chapter{REST-содержимое}

Существуют истории о том, как  на заре Интернета поисковые системы
уничтожали целые веб-сайты. Когда динамическое содержимое было ещё новой концепцией,
разработчики не принимали во внимание различий между GET и POST запросами. В результате,
они создавали страницы с доступом через GET-метод, которые могли удалить содержимое. Когда
поисковики сканировали такие сайты, они могли всё удалить.

Если бы эти разработчики точнее следовали спецификации HTTP, этого бы не случалось.
GET-запрос подразумевает отсутствие побочных эффектов (вроде стирания сайта, к примеру).
Не так давно, в веб-разработке произошло движение к правильному пониманию <<передачи
состояния представления>> (REpresentational State Transfer), известного также как REST.
Эта
глава описывает особенности Yesod, связанные с поддержкой REST, и как они могут быть
использованы для создания более надёжных веб-приложений.

\section{Методы HTTP-запросов}

Во многих веб-фреймворках вы пишете для ресурса одну функцию-обработчик. В
Yesod по умолчанию предполагается отдельная функция-обработчик для каждого метода
запроса. Два самых частых метода, с которыми вы будете иметь дело при создании веб-сайтов
--- это GET и POST. Эти методы лучше всего поддерживаются в формате HTML, поскольку только
они используются в веб-формах. Однако при создании RESTful API, другие методы также очень
полезны.

Технически, вы можете создать какие угодно методы, но настоятельно рекомендуется
придерживаться описанных в спецификации HTTP. Наиболее часто встречаются методы:
\begin{itemize}
 \item GET. Запрос на чтение. Предполагается, что при вызове GET не произойдёт никаких
 изменений на сервере. При многократном вызове GET-метода, сервер должен выдавать тот же
ответ, если исключить такие вещи как <<текущее время>> или результаты, получаемые
случайно.
 \item POST. Обычный запрос на изменение. Он никогда не должен посылаться пользователем
дважды. Обычно, в качестве примера POST-запроса приводят перевод денежных средств с
одного банковского счета на другой.
 \item PUT. Создание нового ресурса на сервере или изменение существующего. Этот метод
 может быть выполнен многократно с одинаковым эффектом.
 \item DELETE. В точности соответствует названию. Удаляет ресурс с сервера. Повторные
 вызовы не должны приводить к проблемам.
\end{itemize}

В определённой степени, это соответствует философии Haskell: GET-запрос подобен чистой
функции, без побочных эффектов. На практике ваши GET-функции вероятно будут производить
IO-операции --- чтение информации из базы данных, логирование пользовательских действий и
так далее.

В главе \hyperref[chap:routing]{Маршрутизация URL и обработчики} можно найти более
полную информацию по
синтаксису определения функций-обработчиков для каждого метода HTTP-запроса.

\section{Представления}

Пусть мы определили тип данных и значение:

\begin{lstlisting}
data Person = Person { name :: String, age :: Int }
michael = Person "Michael" 25
\end{lstlisting}

Мы можем представить эти данные в HTML:
\begin{lstlisting}[language=HTML]
 <table>
    <tr>
        <th>Name</th>
        <td>Michael</td>
    </tr>
    <tr>
        <th>Age</th>
        <td>25</td>
    </tr>
</table>
\end{lstlisting}

или в JSON:

\begin{lstlisting}
{"name":"Michael","age":25}
\end{lstlisting}

или в XML:
\begin{lstlisting}[language=XML]
<person>
    <name>Michael</name>
    <age>25</age>
</person>
\end{lstlisting}

Зачастую веб-приложения используют различные URL для получения этих представлений: возможно \lstinline'/person/michael.html, /person/michael.json', и т.п. Yesod, следуя принципам RESTful, использует один URL для каждого ресурса. Поэтому в Yesod все представления будут доступны через \lstinline'/person/michael'.

Возникает вопрос, как мы определим, какое представление требуется. Ответ можно найти в
\verb*|Accept|-заголовке запроса: он получает приоритезированный список типов содержимого,
ожидаемого клиентом. Основываясь на этом заголовке, Yesod будет автоматически определять,
какое представление требуется.

Немного конкретизируем последнее предложение следующим кодом:

\begin{lstlisting}
type ChooseRep = [ContentType] -> IO (ContentType, Content)
class HasReps a where
    chooseRep :: a -> ChooseRep
\end{lstlisting}

Функция \lstinline'chooseRep' получает два аргумента: значение, для которого мы получаем
представления, и список типов содержимого, которые клиент будет получать. Список
определяется чтением из заголовка \verb*|Accept|-запроса. \lstinline'chooseRep' возвращает
пару,
включающую тип содержимого ответа и непосредственно содержимое.

Этот класс типов --- основа RESTful-подхода, принятого в Yesod для представлений. Каждая
функция-обработчик должна возвращать экземпляр \lstinline'HasReps'. Когда Yesod генерирует
функцию-диспетчер, он автоматически применяет \lstinline'chooseRep' к каждому обработчику,
по существу давая всем функциям тип \lstinline'Handler ChooseRep'. После  запуска
\lstinline'Handler' и получения результата \lstinline'ChooseRep',  он применяет список
типов содержимого, полученный из \verb*|Accept|-заголовка.

В Yesod определены многочисленные экземпляры класса \lstinline'HasReps'. Например, когда
мы используем \lstinline'defaultLayout', возвращается тип \lstinline'RepHtml', который
выглядит так:

\begin{lstlisting}
newtype RepHtml = RepHtml Content
instance HasReps RepHtml where
    chooseRep (RepHtml content) _ = return ("text/html", content)
\end{lstlisting}

Заметьте, что мы полностью игнорируем список ожидаемых типов содержимого. Многочисленные встроенные представления (\lstinline'RepHtml, RepPlain, RepJson, RepXml'), в действительности, поддерживают только единственное представление и, следовательно, то, что содержится в Accept-заголовке клиентского запроса, является несущественным.

\section{RepHtmlJson}

Противоположный пример --- \lstinline'RepHtmlJson', который создаёт либо HTML, либо JSON
представление. Этот экземпляр класса особенно полезен в программировании AJAX приложений.
Вот пример, возвращающий либо HTML, либо JSON данные, в зависимости от того, что хочет
клиент.

\begin{lstlisting}
{-# LANGUAGE QuasiQuotes, TypeFamilies, OverloadedStrings #-}
{-# LANGUAGE MultiParamTypeClasses, TemplateHaskell #-}
import Yesod
data R = R
mkYesod "R" [parseRoutes|
/ RootR GET
/#String NameR GET
|]
instance Yesod R

getRootR = defaultLayout $ do
    setTitle "Homepage"
    addScriptRemote "http://ajax.googleapis.com/ajax/libs/jquery/1.4/jquery.min.js"
    toWidget [julius|
$(function(){
    $("#ajax a").click(function(){
        jQuery.getJSON($(this).attr("href"), function(o){
            $("div").text(o.name);
        });
        return false;
    });
});
|]
    let names = words "Larry Moe Curly"
    [whamlet|
<h2>AJAX Version
<div #results>
    AJAX results will be placed here when you click #
    the names below.
<ul #ajax>
    $forall name <- names
        <li>
            <a href=@{NameR name}>#{name}

<h2>HTML Version
<p>
    Clicking the names below will redirect the page #
    to an HTML version.
<ul #html>
    $forall name <- names
        <li>
            <a href=@{NameR name}>#{name}

|]

getNameR name = do
    let widget = do
            setTitle $ toHtml name
            [whamlet|Looks like you have Javascript off. Name: #{name}|]
    let json = object ["name" .= name]
    defaultLayoutJson widget json

main = warpDebug 4000 R

\end{lstlisting}

Наш обработчик \lstinline'getRootR' создаёт страницу с тремя ссылками и Javascript-кодом,
который перехватывает нажатия на ссылки и производит асинхронные запросы. Если у
пользователя разрешён Javascript, то нажатие на ссылку приводит к запросу, отсылаемому с
\verb*|Accept|-заголовком \verb*|application/json|. В этом случае, \lstinline'getNameR'
будет возвращать JSON-представление.

Если пользователь запрещает Javascript, то нажатие на ссылку будет отсылать пользователя к соответствующему URL. Браузер укажет приоритет HTML-представления данных и, следовательно, будет возвращена страница, определённая виджетом.

Мы, конечно, можем расширить это поведение для работы с XML, Atom-лентами, или даже
двоичным представлением данных. Для развлечения можно было бы написать веб-приложение,
которое готовит данные, просто используя для типов данных стандартные экземпляры класса
\lstinline'Show'. И затем написать веб-клиента, который разбирает результаты, используя
стандартные экземпляры класса \lstinline'Read'.

Возможно, вас беспокоит эффективность этого подхода. Разве это не означает, что мы должны
генерировать и HTML и JSON ответы для каждого запроса? Благодаря ленивости этого не
происходит. В функции \lstinline'getNameR' ни \lstinline'widget', ни \lstinline'json' не
будет вычислен, пока не будет выбран соответствующий тип ответа, и, следовательно, всегда
будет выполняться только один из них.

\section{Новостные ленты}

Отличный практический пример множественных представлений даёт пакет yesod-newsfeed. В вебе существуют два главных формата новостных лент: RSS и Atom. Они содержат почти в точности схожую информацию, но по-разному упакованную.

В пакете yesod-newsfeed определён тип данных \lstinline'Feed', который содержит такую информацию, как название, описание и время последнего изменения. Далее определено два набора функций для отображения данных: одна для RSS, другая для Atom. Для каждого набора определён собственный тип данных представления:

\begin{lstlisting}
newtype RepAtom = RepAtom Content
instance HasReps RepAtom where
    chooseRep (RepAtom c) _ = return (typeAtom, c)
newtype RepRss = RepRss Content
instance HasReps RepRss where
    chooseRep (RepRss c) _ = return (typeRss, c)
\end{lstlisting}

В третьем модуле определён другой тип данных:

\begin{lstlisting}
data RepAtomRss = RepAtomRss RepAtom RepRss
instance HasReps RepAtomRss where
    chooseRep (RepAtomRss (RepAtom a) (RepRss r)) = chooseRep
        [ (typeAtom, a)
        , (typeRss, r)
        ]
\end{lstlisting}

Этот тип данных автоматически определяет представление, предпочтительное для клиента (по умолчанию выбирается Atom). Если клиентские соединения понимают только RSS, полагая, что они обеспечивают правильные HTTP-заголовки, Yesod выдаст RSS представление.

\section{Другие заголовки запроса}

Существует большое количество других заголовков запроса. Некоторые из них влияют только на
передачу данных между сервером и клиентом, и не должны влиять на приложение в целом.
Например, \verb*|Accept-Encoding| сообщает серверу о том, какие схемы сжатия понимает
клиент, а \verb*|Host| информирует сервер, какие виртуальные хосты обслуживаются.

Другие заголовки влияют на приложение и автоматически считываются в Yesod. Например,
заголовок \verb*|Accept-Language| определяет, какой язык предпочитает клиент (Английский,
Испанский, Немецкий и т.п.). В
\hyperref[chap:i18n]{части про i18n} детально описывается, как используется этот
заголовок.

\section{Отсутствие состояния}

Я оставил этот раздел напоследок, не от того, что он менее важен, а скорее потому что
здесь нет особенностей Yesod, обеспечивающих соответствующую поддержку.

Протокол HTTP не поддерживает состояний (т.е. является stateless-протоколом): каждый
запрос рассматривается
как начало общения. Это означает, к примеру, что серверу не важно то, что вы уже
запрашивали пять страниц --- он будет обрабатывать ваш шестой запрос так, как если бы он
был самым первым.

С другой стороны, некоторая функциональность сайта не будет работать без хранения состояния. Например, как реализовать корзину покупок, если не хранить информацию о покупках между запросами?

Решением являются куки-файлы (cookies) и построенные на их основе сессии. У нас есть
целый
\hyperref[chap:sessions]{раздел}, посвящённый сессиям в Yesod. Однако я не могу не
подчеркнуть, что их следует использовать с осторожностью.

Позвольте привести пример. Существует популярная система отслеживания ошибок, с которой я
работал ежедневно и которая слишком много использовала сессии. Там есть маленький
выпадающий список на каждой странице для выбора текущего проекта. Ничто не предвещает
проблем, да? Все, что этот список делает --- устанавливает текущий проект для вашей
сессии.

В результате, щелчок на ссылке <<просмотреть ошибки>> внутренне зависит от того, какой
проект вы выбирали последним. Из-за этого нет способа сделать закладку на ваши открытые
ошибки для Yesod и отдельную ссылку на ваши ошибки для Hamlet.

Правильный RESTful подход состоит в том, чтобы иметь один ресурс для всех Yesod-ошибок,
а другой для всех Hamlet. В Yesod это легко делается примерно таким
определением маршрута:
\begin{lstlisting}
/ ProjectsR GET
/projects/#ProjectID ProjectIssuesR GET
/issues/#IssueID IssueR GET
\end{lstlisting}

Будьте внимательны к своим пользователям: правильная архитектура без использования состояний означает, что такие базовые вещи, как закладки, хранимые ссылки и кнопки <<Вперёд>>, <<Назад>> будут всегда работать.

\section{Выводы}

Yesod придерживается следующих принципов REST:
\begin{itemize}
 \item использовать правильные методы запроса;
 \item каждый ресурс должен иметь в точности один URL;
 \item разрешать множественные представления данных для одного URL;
 \item проверять заголовки запросов для определения дополнительной информации о том, что
ждёт клиент.
\end{itemize}

Это упрощает использование Yesod не только для построения веб-сайтов, но также и для
построения программных интерфейсов (API). В действительности, используя подобную
\verb*|RepHtmlJson| технику, с одного URL вы можете обслужить одновременно <<дружелюбную к
пользователю>> HTML-страницу и <<дружелюбную к машине>> JSON-страницу.
