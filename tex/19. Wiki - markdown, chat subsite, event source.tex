\chapter{Wiki: разметка, подсайт чата, источник событий}
\label{}

Этот пример свяжет воедино несколько различных идей. Мы начнём с подсайта чата,
позволяющего внедрить виджет чата в любую страницу. Для организации отправки
событий клиенту с сервера будем использовать API источника событий (event
source) HTML 5. Вы можете посмотреть полную версию проекта на
сайте~\footnotehref{https://www.fpcomplete.com/user/snoyberg/yesod/wiki-markdown-chat-subsite-event-source}{FP Haskell Center}.


\section{Подсайт чата}
\input{../hs/19/Chat/Data.lhs}

\input{../hs/19/Chat.lhs}

\section{Основной сайт}
\input{../hs/19/Wiki.lhs}

\section{Выводы}
This example demonstrated creation of a non-trivial subsite. Some important points to
notice were the usage of typeclasses to express constraints on the master site, how data
initialization was performed in the +main+ function, and how ++lift++ing allowed us to
operate in either the subsite or master site context.
If you’re looking for a way to test out your subsite skills, I’d recommend modifying this
example so that the Wiki code also lived in its own subsite.
