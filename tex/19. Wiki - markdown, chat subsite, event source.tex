\chapter{Wiki: разметка, подсайт чата, источник событий}
\label{}

Этот пример свяжет воедино несколько различных идей. Мы начнём с подсайта чата, позволяющего внедрить виджет чата в любую страницу. Для организации отправки событий клиенту с сервера будем использовать API источника событий (event source) HTML 5.

\lstinputlisting{../hs/19/Chat.hs}

Этот модуль является независимым и может использоваться в любом приложении. Ниже мы создадим такое приложение-драйвер~--- wiki. Наша wiki получит жёстко заданную домашнюю страницу, а затем и wiki-раздел сайта. Мы будем использовать множество динамических фрагментов, позволяющих строить произвольную иерархию страниц в рамках Wiki.

В качестве способа хранения мы просто используем изменяемую ссылку на Map. Для реального приложения нужно заменить её на подходящую базу данных. Содержимое будет храниться и предоставляться как Markdown. Плагин-пустышка yesod-auth предоставит нам способ выполнять (фальшивую) аутентификацию.

\lstinputlisting{../hs/19/Wiki.hs}
