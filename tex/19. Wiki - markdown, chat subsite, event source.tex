\chapter{Wiki: разметка, подсайт чата, источник событий}
\label{}

Этот пример свяжет воедино несколько различных идей. Мы начнём с подсайта чата,
позволяющего внедрить виджет чата в любую страницу. Для организации отправки
событий клиенту с сервера будем использовать API источника событий (event
source) HTML 5. Вы можете посмотреть полную версию проекта на
сайте~\footnotehref{https://www.fpcomplete.com/user/snoyberg/yesod/wiki-markdown-chat-subsite-event-source}{FP Haskell Center}.


\section{Подсайт чата}
\input{../hs/19/Chat/Data.lhs}

\input{../hs/19/Chat.lhs}

\section{Основной сайт}
\input{../hs/19/Wiki.lhs}

\section{Выводы}
Этот пример продемонстрировал создание нетривиального подсайта. Некоторые
важные изученные моменты: использование классов типов для выражения
ограничений, накладываемых на основной сайт; способ выполнения инициализации в
функции~\lstinline'main'; использование функции~\lstinline'lift', позволяющей
нам работать в контексте как подсайта, так и основного сайта.

Если вы ищете способ, как проверить свои навыки создания подсайтов, я бы
порекомендовал вам изменить пример и переместить код вики в собственный
подсайт.
