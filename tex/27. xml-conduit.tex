\chapter{Пакет xml-conduit}

%TODO: еще раз проверить орфографию

Многих разработчиков трясет от одной только мысли о работе с XML файлами. XML имеет репутацию формата, имеющего излишне усложненную модель данных, запутанные библиотеки и толстый слой сложностей, находящийся между разработчиком и его целью. Но я смею заверить, что многое из этого относятся скорее к проблемам языков программирования и библиотек, чем формата XML.

Как я уже отмечал, система типов языка Haskell позволяет нам с легкостью свести любую проблему к ее самой базовой форме. Пакет xml-types аккуратно предобразует модель данных XML (поддерживается работа как с целыми документами, так и с потоком данных) в простые абстрактные типы данных. Стандартные неизменяемые структуры данных языка Haskell упрощают преобразование документов, а простой набор функций делает их парсинг и рендеринг легкими и непренужденными. % a simple set of functions makes parsing and rendering a breeze.

Рассмотрим пакет xml-conduit. Под его капотом используется множество подходов, которые Yesod обычно применяет для высокой производительности: пакеты blaze-builder, text, conduit и attoparsec. С точки зрения пользователя он предоставляет все, начиная с простейших API (readFile/writeFile) и заканчивая полным контролем над потоками событий XML.

В дополнение к xml-conduit, в игру вступает еще несколько связанных пакетов, например, xml-hamlet и xml2html. Мы рассмотрим, как должны использоваться эти пакеты, а также когда это следует делать.

\section{Краткое содержание} % Synopsis

Входной XML файл:

\begin{lstlisting}[language=XML]
<document title="My Title">
    <para>This is a paragraph. It has <em>emphasized</em> and <strong>strong</strong> words.</para>
    <image href="myimage.png"/>
</document>
\end{lstlisting}

Код на Haskell:

\begin{lstlisting}
{-# LANGUAGE QuasiQuotes #-}
{-# LANGUAGE OverloadedStrings #-}
import Prelude hiding (readFile, writeFile)
import Text.XML
import Text.Hamlet.XML

main :: IO ()
main = do
    -- readFile will throw any parse errors as runtime exceptions
    -- def uses the default settings
    Document prologue root epilogue <- readFile def "input.xml"

    -- root is the root element of the document, let's modify it
    let root' = transform root

    -- And now we write out. Let's indent our output
    writeFile def
        { rsPretty = True
        } "output.html" $ Document prologue root' epilogue

-- We'll turn out <document> into an XHTML document
transform :: Element -> Element
transform (Element _name attrs children) = Element "html" [] [xml|
<head>
    <title>
        $maybe title <- lookup "title" attrs
            \#{title}
        $nothing
            Untitled Document
<body>
    $forall child <- children
        ^{goNode child}
|]

goNode :: Node -> [Node]
goNode (NodeElement e) = [NodeElement $ goElem e]
goNode (NodeContent t) = [NodeContent t]
goNode (NodeComment _) = [] -- hide comments
goNode (NodeInstruction _) = [] -- and hide processing instructions too

-- convert each source element to its XHTML equivalent
goElem :: Element -> Element
goElem (Element "para" attrs children) =
    Element "p" attrs $ concatMap goNode children
goElem (Element "em" attrs children) =
    Element "i" attrs $ concatMap goNode children
goElem (Element "strong" attrs children) =
    Element "b" attrs $ concatMap goNode children
goElem (Element "image" attrs _children) =
    Element "img" (map fixAttr attrs) [] -- images can't have children
  where
    fixAttr ("href", value) = ("src", value)
    fixAttr x = x
goElem (Element name attrs children) =
    -- don't know what to do, just pass it through...
    Element name attrs $ concatMap goNode children
\end{lstlisting}%$

Выходные данные в формате XHTML:

\begin{lstlisting}[language=XML] % XHTML для lstlisting неизвестен
<?xml version="1.0" encoding="UTF-8"?>
<html>
    <head>
        <title>
            My Title
        </title>
    </head>
    <body>
        <p>
            This is a paragraph. It has 
            <i>
                emphasized
            </i>
            and 
            <b>
                strong
            </b>
            words.
        </p>
        <img src="myimage.png"/>
    </body>
</html>
\end{lstlisting}

\section{Типы} % Types

Воспользуемся восходящим подходом для анализа типов. Помимо прочего, этот раздел будет служить введением в саму модель данных XML, так что не беспокойтесь, если вы не вполне знакомы с ней.

Я думаю, первое место, где Haskell по-настоящему показывает свою силу, это тип данных Name. Многие языки программирования (например, Java) сопротявляются введению выразительного типа данных, представляющего имена в XML. Проблема заключается в том, что на самом деле эти имена состоят из трех частей: локального имени, пространства имен (опционально), а также префикса (также опционально). Для наглядности рассмотрим следующий кусок XML документа:

\begin{lstlisting}[language=XML]
<no-namespace/>
<no-prefix xmlns="first-namespace" first-attr="value1"/>
<foo:with-prefix xmlns:foo="second-namespace" foo:second-attr="value2"/>
\end{lstlisting}

Первый тэг имеет локальное имя no-namespace, но не имеет префикса и не принадлежит какому-либо пространству имен. Второй тэг (с локальным именем no-prefix) также не имеет префикса, но он принадлежит пространству имен first-namespace. Однако атрибут first-attr не наследует это пространство имен: пространства имен атрибутов всегда должны точно задаваться с помощью префикса.

\begin{remark}
Пространства имен почти всегда представляют собой своего рода URI, хотя ничего подобного не требуется ни в одной спецификации.
\end{remark}

Трегий тэг имеет локальное имя with-prefix, префикс foo и принадлежит пространству имен second-namespace. Его атрибут имеет локальное имя second-attr, имеет тот же префикс и принадлежит тому же пространству имен. Атрибуты xmlns и xmlns:foo являются частью спецификации пространства имен и не рассматриваются в качестве атрибутов соответствующих элементов.

Еще раз, из чего состоит имя? Оно имеет локальное имя, а также опциональные префикс и пространство имен. Похоже на подходящий случай для применения записей:

\begin{lstlisting}
data Name = Name
    { nameLocalName :: Text
    , nameNamespace :: Maybe Text
    , namePrefix :: Maybe Text
    }
\end{lstlisting}

Согласно стандарту пространств имен в XML, два имени считаются эквивалентными, если они имеют одинаковое локальное имя и принадлежат одному пространству имен. Другими словами, префикс неважен. Потому в пакете xml-types определены экземпляры классов Eq и Ord, игнорирующие префиксы.

Последний экземпляр класса, который следует упомянуть, это IsString. Было бы очень утомительно печатать Name "p" Nothing Nothing каждый раз, когда нам нужен новый параграф. Если вы включите расширение OverloadedStrings, "p" будет преобразовываться во все это хозяйство самостоятельно! Кроме того, экземлпяр IsString распознает нечто, называемое нотацией Кларка, что позволяет использовать в именах префикс с пространством имен в фигурных скобочках. Другими словами:

\begin{lstlisting}
"{namespace}element" == Name "element" (Just "namespace") Nothing
"element" == Name "element" Nothing Nothing
\end{lstlisting}

\section{Четыре типа узлов} % The Four Types of Nodes

XML-документы представляют собой дерево вложенных узлов. В действительности существует четыре различных типа узлов --- элементы, содержание (то есть, текст), комментарии, а также инструкции по обработке (processing instructions).

\begin{remark}
Вероятно, вы не знакомы с последними, поскольку они используются довольно редко. Они обозначаются следующим образом:

\begin{lstlisting}[language=XML]
<?target data?>
\end{lstlisting}

Есть два удивительных факта об инструкциях по обработке:

\begin{itemize}
\item Инструкции по обработке не имеют атрибутов. Несмотря на то, что вам могут попастся инструкции, имеющие атрибуты, на самом деле не существует никаких правил относительно этих данных в инструкциях. % While often times you'll see processing instructions that appear to have attributes, there are in fact no rules about that data of an instruction.
\item <?xml ...?> не является инструкцией по обработке. Это просто начало документа (также известное, как объявление XML), и так получилось, что оно выглядит поразительно похожим на инструкции по обработке. Разница заключается в том, что строка <?xml ...?> не появится в разобранных данных (parsed content).
\end{itemize}

\end{remark}

Учитывая, что инструкции имеют два куска текста, связанных с ним (target и data), получается очень простой тип данных:

\begin{lstlisting}
data Instruction = Instruction
    { instructionTarget :: Text
    , instructionData :: Text
    }
\end{lstlisting}

Комментариям не соответствует специального типа, потому что они представляют собой обычный текст. Зато содержание (content) куда интереснее --- оно может состоять из простого текста и неразрешенных сущностей (например, \&copyright-statement;). Пакет xml-types оставляет эти сущности неразрешенными во всех типах данных, чтобы полностью соответствовать спецификации. Однако на практике может быть очень трудно работать с такими типами данных. И в большинстве случаев неразрешенная сущность в конечном итоге приведет к возникновению ошибки.

По этой причине модуль Text.XML определяет собственный набор типов данных для узлов, элементов и документов, в которых удаляются все неразрешенные сущности. Если вам нужно работать с неразрешенными сущностями, используйте модуль Text.XML.Unresolved. Начиная с этого момента мы сосредоточимся на типах данных модуля Text.XML, поскольку они почти идентичны версиям из пакета xml-types.

В связи с вышесказанным, содержание (content) также представляет собой обычный текст и потому у него также нет специального типа данных. Последним типом узлов является элемент, который состоит из имени, списка атрибутов и списка дочерних узлов. Каждый атрибут состоит из двух частей - имени и значения. (В пакете xml-types значение атрибута также может содержать неразрешенные сущности.) Итак, определим Element следующим образом:

\begin{lstlisting}
data Element = Element
    { elementName :: Name
    , elementAttributes :: [(Name, Text)]
    , elementNodes :: [Node]
    }
\end{lstlisting}

Возникает закономерный вопрос --- а как должен выглядеть тип данных Node? Вот где Haskell по-настоящему рулит: % This is where Haskell really shines: its sum types model the XML data model perfectly.

\begin{lstlisting}
  data Node
    = NodeElement Element
    | NodeInstruction Instruction
    | NodeContent Text
    | NodeComment Text
\end{lstlisting}

\section{Документы} % Documents

Итак, у нас есть элементы и узлы, но как на счет целых документов? Рассмотрим следующие типы данных:

\begin{lstlisting}
data Document = Document
    { documentPrologue :: Prologue
    , documentRoot :: Element
    , documentEpilogue :: [Miscellaneous]
    }

data Prologue = Prologue
    { prologueBefore :: [Miscellaneous]
    , prologueDoctype :: Maybe Doctype
    , prologueAfter :: [Miscellaneous]
    }

data Miscellaneous
    = MiscInstruction Instruction
    | MiscComment Text

data Doctype = Doctype
    { doctypeName :: Text
    , doctypeID :: Maybe ExternalID
    }

data ExternalID
    = SystemID Text
    | PublicID Text Text
\end{lstlisting}

В спецификации XML сказано, что документ может иметь только один корневой элемент (documentRoot). Он также может содержать опциональное объявление типа документа (doctype). Перед и после как типа документа, так и корневого элемента, разрешается иметь комментарии и инструкции по обработке. (Также разрешены пробелы, но они и так игнорируются во время парсинга.)

Так что там на счет типа документа? Он определяет корневой элемент документа, а затем, опционально, публичный (public) и системый (system) идентификаторы. Эти идентификаторы используются для ссылок на DTD-файлы, которые предоставляют больше информации о файле (например, правила валидации, атрибуты по-умолчанию, разрешения сущностей). Рассмотрим несколько примеров:

\begin{lstlisting}[language=HTML]
<!DOCTYPE root> <!-- no external identifier -->
<!DOCTYPE root SYSTEM "root.dtd"> <!-- a system identifier -->
<!DOCTYPE root PUBLIC "My Root Public Identifier" "root.dtd"> <!-- public identifiers have a system ID as well -->
\end{lstlisting}

Это, друзья мои, и есть вся модель данных XML. На практике, в большинстве случаев вы можете просто игнорировать тип данных Document и переходить сразу к documentRoot.

\section{События} % Events

В дополнение к API для работы с документами пакет xml-types также определяет тип данных Event. Он может быть использован для конструирования поточных инструментов, которые могут потреблять намного меньше оперативной памяти при решении множества задач (например, добавления нового атрибута всем элементам). Сейчас мы не будем рассматривать соответствующий API, однако если вы посмотрите на него после прочтения этой главы, он наверняка покажется вам очень знакомым.

\begin{remark}
Вы можете найти пример использования <<поточного>> API в главе 21, посвященой работе со Sphinx.
\end{remark}

\section{Модуль Text.XML}

Рекомендуемой точкой входа в пакет xml-conduit является модуль Text.XML. Этот модуль экспортирует все типы данных, которые могут вам потредоваться для манипулирования XML в DOM-стиле (Document Object Model, объектная модель документа), а также предоставляет несколько походов к парсингу и рендеренгу XML-документов. Начнем с простого:

\begin{lstlisting}
readFile :: ParseSettings -> FilePath -> IO Document
writeFile :: RenderSettings -> FilePath -> Document -> IO ()
\end{lstlisting}

Здесь вы видите типы данных ParseSettings и RenderSettings. Вы можете использовать их для изменения поведения парсера или рендерера, например, добавления сущностей или включения оформленного вывода (то есть, с отступами). Оба типа являются экземплярами класса типов Default, так что вы можете просто использовать функцию def в тех случаях, когда требуется предоставить значение одного из них. Собственно, так мы и собираемся поступать до конца этой главы. Дополнительную информацию вы можете найти в документации на Hackage: http://goo.gl/FPV1J.

Следует также отметить что в дополнение к API для работы с файлами, также имеются API для работы с текстом и байтовыми строками (bytestring). В функциях для работы с байтовыми строками реализовано интеллектуальное определение кодировки. Поддерживаются кодировки UTF-8, UTF-16 и UTF-32, с прямым (little endian) и обратным (big endian) порядком байт, как с BOM (Byte-Order Marker), так и без него. Весь вывод генерируется в UTF-8.

Для комплексного поиска данных в XML-документах мы рекомендуем использовать API более высокого уровня для работы с курсорами. Стандартный API модуля Text.XML не только формирует базис для этого более высокого уровня. Он также предоставляет отличный API для простого преобразования и простой генерации XML. Пример его использования вы можете найти в разделе <<Краткое содержание>> этой главы.

\section{Замечание относительно путей к файлам} % A note about file paths

В приведенных выше сигнатурах функций вы видели тип FilePath. Однако {\bf это не Prelude.FilePath}. В модуле Prelude определяется синоним type FilePath = [Char]. К сожалению, данный подход имеет множество ограничений, включая неопределенность кодировки файла, а также возможность использования различных символов в качестве разделителей в пути к файлу.

Вместо этого в xml-conduit используется пакет system-filepath, в котором определяется абстрактный тип FilePath. Я лично нахожу такой подход более удобным. Пакет очень прост в использовании, так что здесь я не буду останавливаться на деталях. Вместо этого я приведу лишь краткое пояснение относительно его использования:

\begin{itemize}
\item Поскольку FilePath является экземпляром класса IsString, вы можете вводить обычные строки и они будут интерпретированы правильно, если активировано расширение OverloadedStrings. (Я настоятельно рекомендую использовать его в любом случае, поскольку это делает работу со значениями типа Text намного приятнее.)
\item Если вам требуется преобразование в или из FilePath модуля Prelude, вы должны использовать функцию encodeString или decodeString соответсвенно. Таким образов в рассчет будет принята кодировка пути к файлу.
\item Вместо того, чтобы вручную соединять имена директорий, имена файлов и их расширения, используйте операторы из модуля Filesystem.Path.CurrentOS, например myfolder </> filename <.> extension.
\end{itemize}
  
\section{Курсоры} % Cursor

Допустим, вы хотите получить заголовок (title) из XHTML-документа. Вы можете сделать это с помощью интерфейса модуля Text.XML, который мы только что изучили, используя стандартное сопоставление с образцом потомков элементов. Но работа над программой с использованием этого подхода очень быстро станет утомительной. Вероятно, золотым стандартом для такого типа поиска является XPath, который позволяет вам обращаться к элементам, используя пути типа /html/head/title. Именно XPath послужил вдохновением для дизайна комбинаторов из модуля Text.XML.Cursor.

Кусрор представляет собой узел, который помнит свое положение в дереве; он может перемещаться вверх, в сторону или вниз. (Это реализовано путем использования "уз брака"\footnote{http://www.haskell.org/haskellwiki/Tying_the_Knot}.) Есть две функции, позволяющие преобразовывать типы модуля Text.XML в курсоры --- fromDocument и fromNode.

Также имеется концепция оси (axis), реализованная как type Axis = Cursor -> [Cursor]. Проще всего понять эту концепцию на примере. Функция child возвращает список из нуля или более курсоров, которые являются потомками текущего курсора. Функция parent возвращает одиночный родительский курсор входного курсора или пустой список для корневого элемента. И так далее.

Некоторые оси используют предикаты. Так функция element обычно используется для фильтрации элементов по имени. Например, element "title" вернет входной элемент только в том случае, если он имеет имя "title", иначе будет возвращен пустой список.

Еще одна функция, которая не вполня является осью, это content :: Cursor -> [Text]. Для всех узлов с неким содержимым она возвращает содержимый текст, иначе возвращается пустой список.

Благодаря тому, что списки являются монадами, не составляет труда объединить все вышесказанное воедино. Например, следующая программа предназначена для поиска заголовка XHTML-документа:

\begin{lstlisting}
{-# LANGUAGE OverloadedStrings #-}
import Prelude hiding (readFile)
import Text.XML
import Text.XML.Cursor
import qualified Data.Text as T

main :: IO ()
main = do
    doc <- readFile def "test.xml"
    let cursor = fromDocument doc
    print $ T.concat $
            child cursor >>= element "head" >>= child
                         >>= element "title" >>= descendant >>= content
\end{lstlisting}

В переводе на русский это значит:

\begin{itemize}
\item Найти все дочерние узлы корневого элемента.
\item Отфильтровать элементы, оставив лишь элементы с именем <<head>>.
\item Найти всех потомков (child) элементов, полученных на предыдущем шаге.
\item Отфильтровать элементы, оставив лишь элементы с именем <<title>>.
\item Найти всех наследников (descendant) полученных элементов. (Наследник - это потомок или наследник потомка. Да, это рекурсивное определение.)
\item Оставить только текстовые узлы.
\end{itemize}

Итак, для входного документа:

\begin{lstlisting}[language=HTML]
  <html>
    <head>
        <title>My <b>Title</b></title>
    </head>
    <body>
        <p>Foo bar baz</p>
    </body>
</html>
\end{lstlisting}

Мы получим вывод <<My Title>>. Это все, конечно, здорово и замечательно, но вообще-то тут намного больше кода, чем в случае использования XPath. Для борьбы с этой многословностью Aristid Breitkreuz добавил в модуль Cursor набор операторов для обработки больщинства случаев. С их помощью мы можем переписать наш пример следующим образом:

\begin{lstlisting}
{-# LANGUAGE OverloadedStrings #-}
import Prelude hiding (readFile)
import Text.XML
import Text.XML.Cursor
import qualified Data.Text as T

main :: IO ()
main = do
    doc <- readFile def "test.xml"
    let cursor = fromDocument doc
    print $ T.concat $
        cursor $/ element "head" &/ element "title" &// content
\end{lstlisting}%$

Оператор \$/ говорит применить ось справа к потомкам (children) курсора слева. Оператор \&/ практически идентичен, только используется он для комбинирования двух осей. Общее правило в модуле Text.XML.Cursor таково: операторы, начинающиеся со знака \$, напрямую применяют ось, а начинающиеся со знака \& объединяют две оси. Оператор \&// используется для применения оси ко всем наследникам (descendants).

Рассмотрим более сложный (или более надуманный?) пример. Имеется следующий документ:

\begin{lstlisting}[language=HTML]
  <html>
    <head>
        <title>Headings</title>
    </head>
    <body>
        <hgroup>
            <h1>Heading 1 foo</h1>
            <h2 class="foo">Heading 2 foo</h2>
        </hgroup>
        <hgroup>
            <h1>Heading 1 bar</h1>
            <h2 class="bar">Heading 2 bar</h2>
        </hgroup>
    </body>
</html>
\end{lstlisting}

Мы хотим получить содержимое всех тэгов h1, которые предшествуют тэгу h2 с атрибутом class, имеющим значение <<bar>>. Для выполнения этого запутанного поиска мы можем написать:

\begin{lstlisting}
{-# LANGUAGE OverloadedStrings #-}
import Prelude hiding (readFile)
import Text.XML
import Text.XML.Cursor
import qualified Data.Text as T

main :: IO ()
main = do
    doc <- readFile def "test2.xml"
    let cursor = fromDocument doc
    print $ T.concat $
        cursor $// element "h2"
               >=> attributeIs "class" "bar"
               >=> precedingSibling
               >=> element "h1"
               &// content
\end{lstlisting}%$

Давайте попробуем разобраться, что здесь происходит. Сначала мы получаем все элементы h2 в документе. (Оператор \$// получает всех наследников корневого элемента.) Из них мы оставляем только те, что имеют атрибут class со значением <<bar>>. Оператор >=> на самом деле является стандартным оператором из модула Control.Monad; вот еще одно преимущество того, что списки являются монадами. Функция precedingSibling находит все узлы, что идут перед заданным и имеют с ним общего родителя. (Имеется также предшествующая ось, с помощью которой мы ранее получили все элементы в дереве.) Затем мы просто берем все элементы h1 и получаем их содержимое.

\begin{remark}
Эквивалентный XPath, для сравнения, будет //h2[@class ='bar']/preceding-sibling::h1//text().
\end{remark}

Хоть API курсоров и уступает XPath в краткости, зато мы обращаемся к нему, используя чистый Haskell и обеспечивая безопасность типов.

\section{Пакет xml-hamlet}

Благодаря простоте системы типов языка Haskell, создание XML-документов с помощью API модуля Text.XML является крайне простым, хотя и несколько многословным. Следующий код:

\begin{lstlisting}
{-# LANGUAGE OverloadedStrings #-}
import Text.XML
import Prelude hiding (writeFile)

main :: IO ()
main =
    writeFile def "test3.xml" $ Document (Prologue [] Nothing []) root []
  where
    root = Element "html" []
        [ NodeElement $ Element "head" []
            [ NodeElement $ Element "title" []
                [ NodeContent "My "
                , NodeElement $ Element "b" []
                    [ NodeContent "Title"
                    ]
                ]
            ]
        , NodeElement $ Element "body" []
            [ NodeElement $ Element "p" []
                [ NodeContent "foo bar baz"
                ]
            ]
        ]
\end{lstlisting}

... генерирует:

\begin{lstlisting}[language=XML]
<?xml version="1.0" encoding="UTF-8"?>
<html><head><title>My <b>Title</b></title></head><body><p>foo bar baz</p></body></html>
\end{lstlisting}

Это во много раз проще, чем использовать императивный API с изменяемыми состояниями (как в Java), но все же далко от идеала и к тому же затемняет наши настоящие намерения. Для исправления ситуации имеется пакет xml-hamlet, который использует расширение QuasiQuotes для того, чтобы позволить вводить XML, используя естественный синтаксис. Например, предыдущий пример может быть переписан так:

\begin{lstlisting}
{-# LANGUAGE OverloadedStrings #-}
{-# LANGUAGE QuasiQuotes #-}
import Text.XML
import Text.Hamlet.XML
import Prelude hiding (writeFile)

main :: IO ()
main =
    writeFile def "test3.xml" $ Document (Prologue [] Nothing []) root []
  where
    root = Element "html" [] [xml|
<head>
    <title>
        My #
        <b>Title
<body>
    <p>foo bar baz
|]
\end{lstlisting}%$

Тут нужно обратить внимание на следующее:

\begin{itemize}
\item Синтаксис практически идентичен Hamlet, если не считать отсутствия интерполяции URL (@{...}). Таким образом:

\begin{itemize}
\item Никаких закрывающих тэгов.
\item Имеет место чувствительность к пробелам.
\item Если вам нужны пробелы в конце строки, используйте на конце символ \#. В начале строки используйте обратный слэш.
\end{itemize}
  
\item XML-интерполяция возвращает список узлов. В связи с этим вы все еще должны оборачивать результат в обычную конструкцию из Document и корневого Element.
\item Нет поддержки специальной формы атрибутов .class и \#id.
\end{itemize}

Как и в нормальном Hamlet, вы можете использовать интерполяцию переменных и управляющие структуры. Рассмотрим это на чуть более сложном примере:

\begin{lstlisting}
{-# LANGUAGE OverloadedStrings #-}
{-# LANGUAGE QuasiQuotes #-}
import Text.XML
import Text.Hamlet.XML
import Prelude hiding (writeFile)
import Data.Text (Text, pack)

data Person = Person
    { personName :: Text
    , personAge :: Int
    }

people :: [Person]
people =
    [ Person "Michael" 26
    , Person "Miriam" 25
    , Person "Eliezer" 3
    , Person "Gavriella" 1
    ]

main :: IO ()
main =
    writeFile def "people.xml" $ Document (Prologue [] Nothing []) root []
  where
    root = Element "html" [] [xml|
<head>
    <title>Some People
<body>
    <h1>Some People
    $if null people
        <p>There are no people.
    $else
        <dl>
            $forall person <- people
                ^{personNodes person}
|]

personNodes :: Person -> [Node]
personNodes person = [xml|
<dt>#{personName person}
<dd>#{pack $ show $ personAge person}
|]
\end{lstlisting}%$

Еще пара моментов:

\begin{itemize}
\item Caret-интерполяция (\^{...}) принимает список узлов и следовательно может с легкостью вставлять другие xml-цитаты.
\item В отличии от Hamlet, hash-интерполяции (\#{...}) не являются полиморфными и могут принимать только значения типа Text.
\end{itemize}

\section{Пакет xml2html}

До сих пор в этой главе наши примеры вращались вокруг XHTML. Я сделал это по той простой причине, что XHTML скорее всего окажется наиболее знакомой формой XML для большинства читателей. Но в этом есть и отрицательный момент, который следует признать: не всякий XHTML является корректным HTML. Существуют следующие расхождения:

\begin{itemize}
\item Некоторые <<пустые>> HTML-тэги (например, img, br) не обязаны иметь парные закрывающие тэги, и вообще-то не имеют права их иметь;
\item HTML не понимает самозакрывающиеся тэги, то есть <script></script> и <script/> означают разные вещи;
\item Объединяем два предыдущих пункта: <<пустые>> тэги могут быть самозакрывающимися, однако это ничего не значит для браузера;
\item Чтобы избежать недоразумений, HTML-документы должны начинаться с DOCTYPE-выражения;
\item XML-объявление <?xml ...?> не нужно в HTML-страницах;
\item В HTML нет никаких пространств имен, в то время, как XHTML ими полон;
\item Содержимое тэгов <style> и <script> не должно экранироваться; % В XHTML или HTML?
\end{itemize}
  
К счастью, пакет xml-conduit представляет экземпляры класса ToHtml для типов Node, Document и Element, которые уважают эти расхождения. Итак, просто используя функцию toHtml, мы можем получить корректный вывод. Следующая программа:

\begin{lstlisting}
{-# LANGUAGE OverloadedStrings, QuasiQuotes #-}
import Text.Blaze.Html (toHtml)
import Text.Blaze.Html.Renderer.String (renderHtml)
import Text.XML
import Text.Hamlet.XML
import Text.XML.Xml2Html ()

main :: IO ()
main = putStr $ renderHtml $ toHtml $ Document (Prologue [] Nothing []) root []

root :: Element
root = Element "html" [] [xml|
<head>
    <title>Test
    <script>if (5 < 6 || 8 > 9) alert("Hello World!");
    <style>body > h1 { color: red }
<body>
    <h1>Hello World!
|]
\end{lstlisting}%$

... выводит (пробелы добавлены вручную):

\begin{lstlisting}[language=HTML]
<!DOCTYPE HTML>
<html>
    <head>
        <title>Test</title>
        <script>if (5 < 6 || 8 > 9) alert("Hello World!");</script>
        <style>body > h1 { color: red }</style>
    </head>
    <body>
        <h1>Hello World!</h1>
    </body>
</html>
\end{lstlisting}%$
