\chapter{Учимся на примерах: полнотекстовый поиск с использованием Sphinx} % Case Study: Sphinx-based Search

% TODO: еще раз перечитать и проверить орфографию.

% FIXME: вставить картинку (см строку 66)
% FIXME: перевести исходники
% FIXME: в 21-4.hs помимо обработчика поискового запроса дофига другого кода, может выпилить его?
% FIXME: (?) перевести пример HTML (см строку 132)

Sphinx --- это сервер полнотекстового поиска, благодаря которому реализован поиск на многих сайтах, включая сайт самого Yesod. Код, необходимый для интеграции Sphinx и Yesod, является довольно коротким. Тем не менее, он содержит несколько непростых моментов, а потому является прекрасной иллюстрацией к использованию некоторых деталей внутреннего устройства Yesod. % a great case study in how to play with some of the under-the-surface details of Yesod.

По существу мы будем использовать три вещи: % There are essentially three different pieces at play here:

\begin{itemize}
  \item Сохранение данных, которые мы хотели бы искать. В действительности, тут непосредственно используется Persistent-код, и мы не будем подробно останавливаться на этом. % This is fairly straight-forward Persistent code, and we won't dwell on it much in this chapter.
  \item Доступ из Yesod к результатам поиска Sphinx. На самом деле, благодаря пакету sphinx, это довольно просто.
  \item Предоставление содержимого документов серверу Sphinx. Вот, где происходят интересные вещи, и где будет показано, как напрямую преобразовывать поточный контент из базы данных напрямую в XML, который затем по проводам передается клиенту. % This is where the interesting stuff happens, and will show how to deal with streaming content from a database directly to XML, which gets sent directly over the wire to the client.
\end{itemize}

\section{Установка Sphinx} % Sphinx Setup

В отличии от многих других примеров, чтобы начать, здесь нам действительно потребуется настроить и запустить внешний сервер Sphinx. Я не собираюсь затрагивать все особенности Sphinx, отчасти, потому что они не относятся к делу, но в основном --- потому что я далеко не эксперт в Sphinx.

Sphinx предоставляет три основных консольных утилиты: Демон searchd непосредственно принимает запросы от клиента (в данном случае --- нашего веб-приложения) и возвращает результат поиска. Программа indexer обрабатывает документы и создает индекс поиска. Утилита search предназначена для отладки, она посылает простые поисковые запросы серверу Sphinx.

Настройки Sphinx содержат два важных параметра --- источник (source) и индекс (index). Параметр source говорит Sphinx, откуда ему следует считывать информацию о документах. Поддерживается как чтение из MySQL и PostgreSQL напрямую, так и более общий XML-формат, известный, как xmlpipe2. Его мы и собираемся использовать. Это не только даст нам гибкость в плане выбора Persistent-бэкендов, но и продемонстрирует некоторые более мощные концепции Yesod.

Второй важный параметр --- это index. Sphinx может поддерживать несколько индексов одновременно, что позволяет организовать поиск для нескольких служб с помощью одного сервера. Каждому параметру index соответствует параметр source, указывающий на то, откуда брать данные.

Наше приложение будет предоставлять специальный URL (/search/xmlpipe), с помощью которого XML-файл, требуемый Sphinx, будет передаваться в indexer. В конфигурационный файл Sphinx следует прописать следующее:

\begin{lstlisting}
source searcher_src
{
    type = xmlpipe2
    xmlpipe_command = curl http://localhost:3000/search/xmlpipe
}

index searcher
{
    source = searcher_src
    path = /var/data/searcher
    docinfo = extern
    charset_type = utf-8
}
\end{lstlisting}

Чтобы построить индекс, необходимо запусить \lstinline'indexer'. Очевидно, это не будет работать до тех пор, пока мы не запустим наше приложение.  В окончательной версии сайта, \lstinline'indexer' следует запускать периодически с помощью crontab, чтобы индекс регулярно обновлялся.

\section{Базовая настройка Yesod} % Basic Yesod Setup

% FIXME: Let's get our basic Yesod setup going

Создадим новое Yesod-приложение. Нам понадобится одна таблица в базе данных для хранения документов, которые будут состоять из заголовка и текста. Эту таблицу мы будем хранить в базе данных SQLite. Также мы создадим маршруты для поиска, добавления и просмотра документов, а также для генерации xmlpipe-файла для Sphinx.

\lstinputlisting{../hs/21-1.hs}

К счастью, на данный момент все это уже кажется знакомым. Далее мы определим две формы --- одну для создания документов и одну для поиска:

\lstinputlisting{../hs/21-2.hs}

Передача параметра \lstinline'True' в функцию \lstinline'searchField' обеспечивает автоматическую установку фокуса на поле при загрузке страницы. Наконец, объявляем обработчики для главной страницы (на ней отображаются формы добавления и поиска документов), отображения и добавления документа.

\lstinputlisting{../hs/21-3.hs}

\section{Поиск} % Searching

Теперь, когда мы разобрались со всякой рутиной, займемся непосредственно поиском. В результатах поиска нам понадобится отображать три вещи --- идентификатор документа, его заголовок, а также выдержки. Выдержки представляют собой выделенные отрывки документа, содержащие искомое выражение.

Результаты поиска (картинка)

Давайте начнем с определения типа данных Result:

\begin{lstlisting}
data Result = Result
    { resultId :: DocId
    , resultTitle :: Text
    , resultExcerpt :: Html
    }
\end{lstlisting}

Теперь взглянем на обработчик поискового запроса:

\lstinputlisting{../hs/21-4.hs}

% FIXME: мы просто полагаемся на функцию ...

Ничего волшебного здесь не происходит, мы просто полагаемся на функции \lstinline'searchForm', объявленную выше, и еще не объявленную \lstinline'getResults'. Эта функция просто принимает строку с поисковым запросом и возвращает список результатов. Здесь мы впервые взаимодейтсвуем с API сервера Sphinx. Мы будем использовать две функции --- \lstinline'query' будет возвращать список совпадений, а \lstinline'buildExcerpts' --- выделенные отрывки. Взгляним на функцию query:

\lstinputlisting{../hs/21-5.hs}

Она принимает три аргумента --- параметры конфигурации, строку с именем индекса, по которому будет производиться поиск, и поисковый запрос. Функция возвращает список идентификаторов документов, содержащих поисковый запрос. Здесь есть небольшая хитрость, связанная с тем, что возвращаемые идентификаторы имеют тип \lstinline'Int64', когда нам нужен \lstinline'DocId'. Мы воспользуемся тем, что SQL-бэкенды \lstinline'Persistent' используют конструктор \lstinline'PersistInt64' для этих идентификаторов, после чего сможем получить требуемые значения. % and simply wrap up the values appropriately.

Если вы работаете с бэкендом, использующим нечисловые идентификаторы, например MongoDB, вам придется разработать какой-нибудь более изощренный способ.

Затем мы проходимся в цикле по этим идентификатором, получая список \lstinline'[Maybe Result]', после чего используем функцию \lstinline'catMaybes' для преобразования этого списка в \lstinline'[Result]'. В where-клозе мы определяем локальные настройки, которые замещают номер порта, используемый по умолчанию, а также изменяют режим поиска на такой, в котором возвращаются все документы, содержащие хотя бы одно слово из поискового запроса. % "set up the search to work when any term matches the document" + в исходниках S.mode = ST.Any + см http://sphinxsearch.com/docs/1.10/matching-modes.html

Наконец, взглянем на функцию \lstinline'getResult':

\begin{lstlisting}
getResult :: DocId -> Doc -> Text -> IO Result
getResult docid doc qstring = do
    excerpt' <- S.buildExcerpts
        excerptConfig
        [T.unpack $ escape $ docContent doc]
        "searcher"
        (unpack qstring)
    let excerpt =
            case excerpt' of
                ST.Ok bss -> preEscapedToMarkup $ decodeUtf8With ignore $ L.concat bss
                _ -> ""
    return Result
        { resultId = docid
        , resultTitle = docTitle doc
        , resultExcerpt = excerpt
        }
  where
    excerptConfig = E.altConfig { E.port = 9312 }

escape :: Textarea -> Text
escape =
    T.concatMap escapeChar . unTextarea
  where
    escapeChar '<' = "&lt;"
    escapeChar '>' = "&gt;"
    escapeChar '&' = "&amp;"
    escapeChar c   = T.singleton c
\end{lstlisting}

Функция \lstinline'buildExcerpts' принимает четыре аргумента --- параметры конфигурации, содержимое документа, строку с именем индекса и поисковый запрос. Обратите внимание на эранирование специальных символов в документе. Sphinx его не производит, потому этим приходится заниматься нам самим.

Аналогично, результат поиска, который мы получаем от Sphinx, представляет собой список ленивых ByteString'ов. Но мы, само собой разумеется, предпочли бы получить \lstinline'Html'. Потому мы конкатенируем элементы списка в одну ленивую \lstinline'ByteString', декодируем ее в ленивый текст (игнорируя некорректные последовательности символов UTF-8), а затем используем функцию \lstinline'preEscapedToMarkup' для того, чтобы тэги, используемые для выделения найденных совпадений, не экранировались. Вот пример полученного в итоге HTML-кода:

\begin{lstlisting}[language=HTML] % TODO: перевести?
&#8230; Departments.  The President shall have <span class='match'>Power</span> to fill up all Vacancies
&#8230;  people. Amendment 11 The Judicial <span class='match'>power</span> of the United States shall
&#8230; jurisdiction. 2. Congress shall have <span class='match'>power</span> to enforce this article by
&#8230; 5. The Congress shall have <span class='match'>power</span> to enforce, by appropriate legislation
&#8230;
\end{lstlisting}

\section{Генерируем xmlpipe} % Streaming xmlpipe output

Лучшее мы приберегли напоследок. Для большинства обработчиков в Yesod рекомендуемый подход заключается в загрузке данных из базы в оперативную память и генерации вывода, освнованного на этих данных. Так проще, но намного важнее то, что такой подход более устойчив к ошибкам.  Если возникнет проблема во время загрузки данных из базы, пользователь получит должный код ответа 500.

Что я имею ввиду, говоря "должный код ответа 500"? Если вы начнете передавать данные клиенту и в середине процесса наткнетесь на исключение, у вас не будет возможности изменить код ответа. Пользователь получит ответ с кодом 200, просто передача данных прервется посередине. Помимо того, что частично сгенерированная страница сама по себе сбивает с толку, такое поведение еще и не соответсвтует спецификации протокола HTTP.

Однако генерация xmlpipe-вывода является прекрасным примером, где следует использовать альтернативный подход. У нас может быть огромное количество документов (код сайта yesodweb.com оперирует десятками тысяч документов), а документы могут иметь размер порядка нескольких сотен килобайт. Если мы не воспользуемся поточным подходом, это может привести к использованию огромного количества оперативной памяти и большому времени ответа.

Так каким именно образом мы можем сгенерировать поточный ответ (streaming response)? Как мы рассмотрим в главе, посвященной WAI\footnote{Web Application Interface, см главу 24 --- прим. пер.}, у нас есть конструктор ResponseSource который использует поток из элементов типа Builder из библиотеки \lstinline'blaze-builder'. Со стороны Yesod вместо того, чтобы генерировать ответ, как обычно, мы можем послать WAI-ответ напрямую с помощью функции \lstinline'sendWaiResponse'. Таким образом имеется по крайней мере две части головоломки. % So there are at least two of the pieces of this puzzle.

Теперь нам известно, что из неких XML-данных требуется создать поток Builder'ов. К счастью, пакет \lstinline'xml-conduit' предоставляет непосредственный интерфейс для этого. В целом пакет \lstinline'xml-conduit' предоставляет высокоуровневый интерфейс для работы с документами в целом, однако в нашем случае понадобится низкоуровневый интерфейс Event, дабы обеспечить минимальное использование памяти. Вот функция, которая нам нужна:

\begin{lstlisting}
renderBuilder :: Resource m => RenderSettings -> Conduit Event m Builder b
\end{lstlisting}

В переводе на русский это означает, что renderBuilder принимает некие настройки (мы воспользуемся настройками по умолчанию), и преобразует поток Event'ов в поток Builder'ов. Выглядит неплохо. Все, что нам нужно --- это поток Event'ов.

Кстати говоря, а как должен выглядеть наш XML-документ? Он довольно прост, имеется дочерний элемент \lstinline'sphinx:docset', элемент \lstinline'sphinx:schema', содержащий одиночный элемент \lstinline'sphinx:field' (который определяет элемент с содержимым документа), а затем по одному элементу \lstinline'sphinx:document' для каждого документа в базе данных. Эти элементы будут содержать атрибут id и дочерний элемент \lstinline'content'.

\begin{lstlisting}[language=XML] % Пример xmlpipe-документа / Sample xmlpipe document
<sphinx:docset xmlns:sphinx="http://sphinxsearch.com/">
    <sphinx:schema>
        <sphinx:field name="content"/>
    </sphinx:schema>
    <sphinx:document id="1">
        <content>bar</content>
    </sphinx:document>
    <sphinx:document id="2">
        <content>foo bar baz</content>
    </sphinx:document>
</sphinx:docset>
\end{lstlisting}

Каждый XML-документ будет начинатся с одинаковых событий (начать docset, начать schema и т.д.) и заканчиваться одним и тем же событием (закончить docset). Вот соответствующий код: % We'll start off by defining those:

\begin{lstlisting}
toName :: Text -> X.Name
toName x = X.Name x (Just "http://sphinxsearch.com/") (Just "sphinx")

docset, schema, field, document, content :: X.Name
docset = toName "docset"
schema = toName "schema"
field = toName "field"
document = toName "document"
content = "content" -- no prefix

startEvents, endEvents :: [X.Event]
startEvents =
    [ X.EventBeginDocument
    , X.EventBeginElement docset []
    , X.EventBeginElement schema []
    , X.EventBeginElement field [("name", [X.ContentText "content"])]
    , X.EventEndElement field
    , X.EventEndElement schema
    ]

endEvents =
    [ X.EventEndElement docset
    ]
\end{lstlisting}

Теперь, когда у нас есть оболочка нашего документа, нам нужно достать Event'ы для каждого конкретного документа. Это можно сделать при помощи довольно простой функции:

\begin{lstlisting}
entityToEvents :: (Entity Doc) -> [X.Event]
entityToEvents (Entity docid doc) =
    [ X.EventBeginElement document [("id", [X.ContentText $ toPathPiece docid])]
    , X.EventBeginElement content []
    , X.EventContent $ X.ContentText $ unTextarea $ docContent doc
    , X.EventEndElement content
    , X.EventEndElement document
    ]
\end{lstlisting}

Мы начинаем с элемента document, имеющего атрибут id, заем начинаем элемент content, вставляем содержимое документа, после чего закрываем оба элемента. Для преобразования DocId в значение типа Text мы используем функцию toPathPiece. Теперь нам нужно как-то преобразовать поток этих сущностей в поток событий. Для этого воспользуемся функцией concatMap из модуля Data.Conduit.List: \lstinline`CL.concatMap entityToEvents`.

Но что мы действительно хотим, это создать поток этих событий непосредственно из базы данных. На протяжении большей части этой книги, мы использовали функцию selectList, однако Persistent также предоставляет (более мощную) функцию selectSourceConn. С ее помощью мы получаем окончательный результат: % So we end up with the function:

\begin{lstlisting}
docSource :: Connection -> C.Source (C.ResourceT IO) X.Event
docSource conn = selectSourceConn conn [] [] C.$= CL.concatMap entityToEvents
\end{lstlisting}%$

Оператор \$= соединяет источник с каналом, возвращая новый источник. Теперь, когда у нас есть источник Event'ов, все, что нам нужно --- это окружить его событиями начала и конца документа. Благодаря тому, что Source является экземпляром класса Monoid, это проще простого:

\begin{lstlisting}
fullDocSource :: Connection -> C.Source (C.ResourceT IO) X.Event
fullDocSource conn = mconcat
    [ CL.sourceList startEvents
    , docSource conn
    , CL.sourceList endEvents
    ]
\end{lstlisting}

Мы почти закончили, осталось только применить все это в функции getXmlpipeR. Нам необходимо установить соединение с базой данных. Обычно соединения с БД берутся и автоматически возвращаются с помощью функции runDB. В нашем случае требуется получить соединение и иметь к нему доступ до тех пор, пока тело ответа не будет передано полностью. Для этого воспользуемся функцией takeResource, а также регистрацией и очисткой, реализованными в монаде ResourceT.

Все WAI-приложения живут в трансформаторе монады ResourceT. Вы можете получить больше информации о монаде ResourceT в приложении, посвященном потокам (conduit).

По умолчанию ресурс не возвращается в пул. Это связано с правильной обработкой ошибок, что, впрочем, не относится к нашему случаю. Поэтому мы должны заставить соединение возвращаться в пул.

\begin{lstlisting}
getXmlpipeR :: Handler RepXml
getXmlpipeR = do
    Searcher pool <- getYesod
    let headers = [("Content-Type", "text/xml")]
    managedConn <- lift $ takeResource pool
    let conn = mrValue managedConn
    lift $ mrReuse managedConn True let source = fullDocSource conn C.$= renderBuilder def
    sendWaiResponse $ ResponseSource status200 headers source
\end{lstlisting}

Мы получаем пул соединений из базовой переменной, после чего шлем WAI-ответ. Мы используем конструктор ResponseSource, передавая ему код ответа, а также заголовки и тело ответа.

\section{Весь код} % Full code

\lstinputlisting{../hs/21-6.hs}
