\chapter{Переменные окружения для настройки}

Последнее время наметилась тенденция, вероятно, наиболее продвигаемая
приложением \footnotehref{http://12factor.net/config}{12-й фактор}, хранить все
настройки приложения в переменных окружения вместо использования файлов
настройки или жёсткой фиксации их в коде приложения (вы ведь так не делаете?).

Каркас сайта Yesod имеет частичную встроенную поддержку для такого подхода, в
частности: переменная~\lstinline'APPROOT' используется для указания, как должны
генерироваться URL; переменная~\lstinline'PORT' используется для установки
прослушиваемого порта плюс переменные для настройки соединения с базой данных.
(Кстати, всё это замечательно работает с
\footnotehref{http://github.com/snoyberg/keter}{менеджером развёртывания
    Keter}.

Техника работы таким образом достаточна проста: просто реализуйте поиск
переменных окружения в функции~\lstinline'main'. Следующий пример демонстрирует
эту технику, а также несколько специализированную обработку, необходимую для
установки корня приложения.

\sourcecode{environment-variables}{environment-variables.hs}

Единственные сложности в представленном коде:
\begin{itemize}
    \item Вам потребуется конвертировать значение~\lstinline'String',
        возвращаемое функцией~\lstinline'getEnv', в значение~\lstinline'Text',
        используя функцию~\lstinline'pack'.

    \item Мы используем для значения~\lstinline'approot'
        конструктор~\lstinline'ApprootMaster', означающий <<примени эту функцию
        к основному значению для получения фактического корня приложения>>.
\end{itemize}
